\documentclass{report}
\usepackage{graphicx, tikz-cd, float, titlepic, booktabs} % Required for inserting images
\usepackage{pgfplots}
\pgfplotsset{compat=1.15}
\usepackage{mathrsfs}
\usetikzlibrary{arrows}
\usepackage{amsmath, amssymb, amsthm, amsfonts, siunitx, physics, gensymb}
\AtBeginDocument{\RenewCommandCopy\qty\SI}
\usepackage[version=4]{mhchem}
\usepackage[most,many,breakable]{tcolorbox}
\usepackage{xcolor, fancyhdr, varwidth}
\usepackage[Glenn]{fncychap}
%Options: Sonny, Lenny, Glenn, Conny, Rejne, Bjarne, Bjornstrup
\usepackage{hyperref, cleveref}
\usepackage{icomma, enumitem} %comma as decimal and continue enumerate with [resume]
\usepackage[danish]{babel}
%%%%%%%%%%%%%%%%%%%%%%%%%%%%%%
% SELF MADE COLORS
%%%%%%%%%%%%%%%%%%%%%%%%%%%%%%
\definecolor{myg}{RGB}{56, 140, 70}
\definecolor{myb}{RGB}{45, 111, 177}
\definecolor{myr}{RGB}{199, 68, 64}
\definecolor{mytheorembg}{HTML}{F2F2F9}
\definecolor{mytheoremfr}{HTML}{00007B}
\definecolor{mylenmabg}{HTML}{FFFAF8}
\definecolor{mylenmafr}{HTML}{983b0f}
\definecolor{mypropbg}{HTML}{f2fbfc}
\definecolor{mypropfr}{HTML}{191971}
\definecolor{myexamplebg}{HTML}{F2FBF8}
\definecolor{myexamplefr}{HTML}{88D6D1}
\definecolor{myexampleti}{HTML}{2A7F7F}
\definecolor{mydefinitbg}{HTML}{E5E5FF}
\definecolor{mydefinitfr}{HTML}{3F3FA3}
\definecolor{notesgreen}{RGB}{0,162,0}
\definecolor{myp}{RGB}{197, 92, 212}
\definecolor{mygr}{HTML}{2C3338}
\definecolor{myred}{RGB}{127,0,0}
\definecolor{myyellow}{RGB}{169,121,69}
\definecolor{myexercisebg}{HTML}{F2FBF8}
\definecolor{myexercisefg}{HTML}{88D6D1}
%%%%%%%%%%%%%%%%%%%%%%%%%%%%%%%%%%%%%%%%%%%%%%%%%%%%%%%%%%%%%%%%%%%%%%
% Box environments for theorems and problems
%%%%%%%%%%%%%%%%%%%%%%%%%%%%%%%%%%%%%%%%%%%%%%%%%%%%%%%%%%%%%%%%%%%%%
\setlength{\parindent}{1cm}
%================================
% Question BOX
%================================
\makeatletter
\newtcbtheorem{question}{Opgave}{enhanced,
	breakable,
	colback=white,
	colframe=myb!80!black,
	attach boxed title to top left={yshift*=-\tcboxedtitleheight},
	fonttitle=\bfseries,
	title={#2},
	boxed title size=title,
	boxed title style={%
			sharp corners,
			rounded corners=northwest,
			colback=tcbcolframe,
			boxrule=0pt,
		},
	underlay boxed title={%
			\path[fill=tcbcolframe] (title.south west)--(title.south east)
			to[out=0, in=180] ([xshift=5mm]title.east)--
			(title.center-|frame.east)
			[rounded corners=\kvtcb@arc] |-
			(frame.north) -| cycle;
		},
	#1
}{def}
\makeatother
%================================
% DEFINITION BOX
%================================

\newtcbtheorem[]{Definition}{Definition}{enhanced,
	before skip=2mm,after skip=2mm, colback=red!5,colframe=red!80!black,boxrule=0.5mm,
	attach boxed title to top left={xshift=1cm,yshift*=1mm-\tcboxedtitleheight}, varwidth boxed title*=-3cm,
	boxed title style={frame code={
					\path[fill=tcbcolback]
					([yshift=-1mm,xshift=-1mm]frame.north west)
					arc[start angle=0,end angle=180,radius=1mm]
					([yshift=-1mm,xshift=1mm]frame.north east)
					arc[start angle=180,end angle=0,radius=1mm];
					\path[left color=tcbcolback!60!black,right color=tcbcolback!60!black,
						middle color=tcbcolback!80!black]
					([xshift=-2mm]frame.north west) -- ([xshift=2mm]frame.north east)
					[rounded corners=1mm]-- ([xshift=1mm,yshift=-1mm]frame.north east)
					-- (frame.south east) -- (frame.south west)
					-- ([xshift=-1mm,yshift=-1mm]frame.north west)
					[sharp corners]-- cycle;
				},interior engine=empty,
		},
	fonttitle=\bfseries,
	title={#2},#1}{def}
\newtcbtheorem[]{definition}{Definition}{enhanced,
	before skip=2mm,after skip=2mm, colback=red!5,colframe=red!80!black,boxrule=0.5mm,
	attach boxed title to top left={xshift=1cm,yshift*=1mm-\tcboxedtitleheight}, varwidth boxed title*=-3cm,
	boxed title style={frame code={
					\path[fill=tcbcolback]
					([yshift=-1mm,xshift=-1mm]frame.north west)
					arc[start angle=0,end angle=180,radius=1mm]
					([yshift=-1mm,xshift=1mm]frame.north east)
					arc[start angle=180,end angle=0,radius=1mm];
					\path[left color=tcbcolback!60!black,right color=tcbcolback!60!black,
						middle color=tcbcolback!80!black]
					([xshift=-2mm]frame.north west) -- ([xshift=2mm]frame.north east)
					[rounded corners=1mm]-- ([xshift=1mm,yshift=-1mm]frame.north east)
					-- (frame.south east) -- (frame.south west)
					-- ([xshift=-1mm,yshift=-1mm]frame.north west)
					[sharp corners]-- cycle;
				},interior engine=empty,
		},
	fonttitle=\bfseries,
	title={#2},#1}{def}

\newtcbtheorem{theo}%
    {Theorem}{}{theorem}
\newtcolorbox{prob}[1]{colback=red!5!white,colframe=red!50!black,fonttitle=\bfseries,title={#1}}
%================================
% NOTE BOX
%================================

\usetikzlibrary{arrows,calc,shadows.blur}
\tcbuselibrary{skins}
\newtcolorbox{note}[1][]{%
	enhanced jigsaw,
	colback=gray!20!white,%
	colframe=gray!80!black,
	size=small,
	boxrule=1pt,
	title=\textbf{Note:},
	halign title=flush center,
	coltitle=black,
	breakable,
	drop shadow=black!50!white,
	attach boxed title to top left={xshift=1cm,yshift=-\tcboxedtitleheight/2,yshifttext=-\tcboxedtitleheight/2},
	minipage boxed title=1.5cm,
	boxed title style={%
			colback=white,
			size=fbox,
			boxrule=1pt,
			boxsep=2pt,
			underlay={%
					\coordinate (dotA) at ($(interior.west) + (-0.5pt,0)$);
					\coordinate (dotB) at ($(interior.east) + (0.5pt,0)$);
					\begin{scope}
						\clip (interior.north west) rectangle ([xshift=3ex]interior.east);
						\filldraw [white, blur shadow={shadow opacity=60, shadow yshift=-.75ex}, rounded corners=2pt] (interior.north west) rectangle (interior.south east);
					\end{scope}
					\begin{scope}[gray!80!black]
						\fill (dotA) circle (2pt);
						\fill (dotB) circle (2pt);
					\end{scope}
				},
		},
	#1,
}
%================================
% EXAMPLE BOX
%================================
\newtcbtheorem[number within=section]{Example}{Example}
{%
	colback = myexamplebg
	,breakable
	,colframe = myexamplefr
	,coltitle = myexampleti
	,boxrule = 1pt
	,sharp corners
	,detach title
	,before upper=\tcbtitle\par\smallskip
	,fonttitle = \bfseries
	,description font = \mdseries
	,separator sign none
	,description delimiters parenthesis
}
{ex}
%================================
% THEOREM BOX
%================================

\tcbuselibrary{theorems,skins,hooks}
\newtcbtheorem[number within=section]{Theorem}{Theorem}
{%
	enhanced,
	breakable,
	colback = mytheorembg,
	frame hidden,
	boxrule = 0sp,
	borderline west = {2pt}{0pt}{mytheoremfr},
	sharp corners,
	detach title,
	before upper = \tcbtitle\par\smallskip,
	coltitle = mytheoremfr,
	fonttitle = \bfseries\sffamily,
	description font = \mdseries,
	separator sign none,
	segmentation style={solid, mytheoremfr},
}
{th}

%%%%%%%%%%%%%%%%%%%%%%%%%%%%%%%%%%%%%%%%%%%%%%%%%%%%%%%%%%%%%%%%%
% SELF MADE COMMANDS
%%%%%%%%%%%%%%%%%%%%%%%%%%%%%%
\newcommand{\sol}{\setlength{\parindent}{0cm}\textbf{\textit{Løsning:}}\setlength{\parindent}{1cm}}
%%%%%%%%%%%%%%%%%%%%%%%%%%%%%%%%%
\usepackage[tmargin=2cm,rmargin=1in,lmargin=1in,margin=0.85in,bmargin=2cm,footskip=.2in]{geometry}\pagestyle{fancy}
\lhead{Minrui Kevin Zhou 2.b}
\rhead{H10: Elektricitet}

\title{Hjemmeopgave H10: Elektricitet\\
{\Large \textbf{2.b fysik A}}}
\author{Kevin Zhou}
\date{\today}

\begin{document}
\maketitle
\begin{question}{Fire elektriske kredsløb}{}
  \begin{itemize}
    \item[a.] Amperemetret viser $2,0 \;\unit{A} $. Hvad viser det, når kontakten er sluttet.
    \item[b.] Hvad er den mindste værdi voltmetret viser, når den variable resistor varieres mellem $0 \;\unit{\ohm} $ og $2 \;\unit{\ohm} $. 
    \item[c.] Gennem hvilken af de viste resistorer er strømstyrken mindst?
    \item[d.] Hvad sker der med pæren, hvis kontakten $K$ sluttes? 
  \end{itemize}
\end{question}
\sol \\
\textbf{a.} For en resistor gælder Ohms lov og for to resistorer, der sidder i serieforbindelse er erstatningsresistansen blot summen af de to resistorers resistanser.
Altså ser vi når kontakten ikke er sluttet, at
\begin{equation*}
\begin{split}
  U&=R \cdot I\\ 
  &=(3 \;\unit{\ohm} +3 \;\unit{\ohm} ) \cdot 2,0 \;\unit{A} \\ 
  &=12 \;\unit{V} 
\end{split}
\end{equation*}
Når kontakten er sluttet, så ser vi, at en resistor med resistans $3 \;\unit{\ohm} $ sider i serieforbindelse med en parallelkobling af en resistor med resistans $3 \;\unit{\ohm} $ og en resistor med resistans $6 \;\unit{\ohm} $.
Siden der for to resistorer med resistanser $R_1$ og $R_2$ i parallelforbindelse gælder sammenhængen med erstatningsresistansen $R$, at $\frac{1}{R}=\frac{1}{R_1}+\frac{1}{R_2}$, så må den samlede erstatningsresistans for de tre resistorer være
\begin{equation*}
\begin{split}
  R&=3 \;\unit{\ohm} + \left(\frac{1}{3 \;\unit{\ohm} }+\frac{1}{6 \;\unit{\ohm} }\right)^{-1}\\
  &=3 \;\unit{\ohm} + 2 \;\unit{\ohm} \\ 
  &=5 \;\unit{\ohm} 
\end{split}
\end{equation*}
Vi kan da regne strømstyrken ud, når kontakten er sluttet.
\begin{equation*}
\begin{split}
  I&=\frac{U}{R}\\ 
  &=\frac{12 \;\unit{V} }{5 \;\unit{\ohm} }\\ 
  &=2,4 \;\unit{A} 
\end{split}
\end{equation*}
Altså viser amperemetret $2,4 \;\unit{A} $ når kontakten er sluttet. \\[1ex]
\textbf{b.}
Strømstyrken er konstant i en serieforbindelse.
Siden der for resistorerne gælder, at $U=R \cdot I$, så må der gælde, at spændingsfaldet er mindst, når erstatningsresistansen er størst (da strømstyrken så er så lille som mulig).
Da får vi strømstyrken til at være
\begin{equation*}
\begin{split}
  I&=\frac{U}{R}\\ 
  &=\frac{5 \;\unit{V} }{2 \;\unit{\ohm} + 3 \;\unit{\ohm} }\\ 
  &=1 \;\unit{A} 
\end{split}
\end{equation*}
Voltmetret viser spændingsfaldet over resistoren med resistans $3 \;\unit{\ohm} $, der er
\begin{equation*}
\begin{split}
  U&= R \cdot I\\ 
  &=3 \;\unit{\ohm} \cdot 1 \;\unit{A} \\ 
  &=3 \;\unit{V} 
\end{split}
\end{equation*}
Altså er den mindste værdi voltmetret viser $3 \;\unit{V} $.\\[1ex]
\textbf{c.}
For resistorer i seriekobling er strømstyrken den samme.
Altså er strømstyrken gennem resistoren med resistans $1 \;\unit{\ohm} $, parallelkoblingen af de tre resistorer samt resistoren med resistans $5 \;\unit{\ohm} $ den samme. 
I en parallelkobling er spændingen den samme for resistorerne.
Altså er det klart, at en af resistorerne i parallelforbindelsen må være den med mindst strømstyrke igennem sig.
Siden vi har $I=\frac{U}{R}$, så må det være den med den største resistans.
Altså er strømstyrken mindst gennem resistoren med en resistans på $4 \;\unit{\ohm} $.\\[1ex]
\textbf{d.}
Hvis kontakten $K$ sluttes, så må pæren slukke, idet den sidder i parallelforbindelse med en ledning med en resistans nær $0$. 
Altså vil strømmen ikke løbe gennem pæren.
\begin{question}{Kogeplade}{}
  En kogeplade indeholder to modstandstråde med resistanserne $52 \;\unit{\ohm} $ og $92 \;\unit{\ohm}$. 
  En omskifter kan lægge netspændingen på $230 \;\unit{V}$ over hver af de to tråde for sig, over de to tråde koblet i serie eller over de to tråde koblet parallelt.
  \begin{itemize}
    \item[a.] Find den effekt, hvormed kogepladen omsætter energi i hver af de fire situationer.
  \end{itemize}
\end{question}
\sol \\
Vi antager, at de to modstandstråde kan betegnes som resistorer.
Når netspændingen lægges over tråden med resistans $52 \;\unit{\ohm} $ for sig, så er effekten
\begin{equation*}
\begin{split}
  P&=\frac{U^2}{R}\\ 
  &=\frac{\left(230 \;\unit{V} \right)^2}{52 \;\unit{\ohm} }\\ 
  &\approx 1,0 \;\unit{kW} 
\end{split}
\end{equation*}
Når netspændingen lægges over tråden med resistans $92 \;\unit{\ohm} $ for sig, så er effekten
\begin{equation*}
\begin{split}
  P&=\frac{U^2}{R}\\ 
  &=\frac{\left(230 \;\unit{V} \right)^2}{92 \;\unit{\ohm} }\\ 
  &\approx 0,58 \;\unit{kW} 
\end{split}
\end{equation*}
Når netspændingen lægges over de to tråde i serie, så er effekten
\begin{equation*}
\begin{split}
P&=\frac{U^2}{R}\\ 
&=\frac{\left(230 \;\unit{V} \right)^2}{52 \;\unit{\ohm} + 92 \;\unit{\ohm} }\\ 
  &\approx 0,37 \;\unit{kW} 
\end{split}
\end{equation*}
Når netspændingen lægges over de to tråde koblet parallelt, så er effekten
\begin{equation*}
\begin{split}
P&=\frac{U^2}{R}\\ 
  &=\frac{\left(230 \;\unit{V} \right)^2 }{\left(\frac{1}{52 \;\unit{\ohm} }+\frac{1}{92 \;\unit{\ohm} }\right)^{-1}}\\ 
  &\approx 1,6 \;\unit{kW} 
\end{split}
\end{equation*}
Altså er effekten, hvormed kogepladen omsætter energi i hver af de fire situationer henholdsvis $1,0 \;\unit{kW} $, $0,58 \;\unit{kW} $, $0,37 \;\unit{kW} $ og $1,6 \;\unit{kW} $. 
\begin{question}{Varmelegeme til bilbagrude}{}
  Et elektrisk varmelegeme til bagruden af en bil består af 10 ens metaltråde, der er forbundet parallelt. 
  Hver tråd har længden $1,02 \;\unit{m} $ og tværsnitsarealet $0,038 \;\unit{mm^2}$.
Trådene er lavet af et materiale med resistiviteten $0,421 \;\unit{\ohm \cdot mm^2/m}$.
\begin{itemize}
  \item[a.] Bestem resistansen af hver af trådene og varmelegemets samlede resistans.
\end{itemize}
Varmelegemet sluttes til bilens batteri, som har hvilespændingen $12,0 \;\unit{V}$ og den indre resistans $0,17 \;\unit{\ohm} $.
\begin{itemize}
  \item[b.] Bestem strømstyrken i kredsløbet og polspændingen over batteriet.
\end{itemize}
Et batteris evne til at levere strøm angives ved dets ladning, ofte angivet i enheden amperetimer.
\begin{itemize}
  \item[c.] Hvor lang tid kan batteriet holde varmelegemet tændt, når batteriet indeholder $46 \;\unit{Ah}$?
\end{itemize}
\end{question}
\sol \\
\textbf{a.}
En af trådene har da resistansen
\begin{equation*}
\begin{split}
  R_{\text{tråd} }&=\rho \cdot \frac{l}{A}\\
  &=0,421 \;\unit{\ohm \cdot \frac{mm^2}{m}} \cdot \frac{1,02 \;\unit{m} }{0,038 \;\unit{mm^2} }\\ 
  &\approx 11 \;\unit{\ohm} 
\end{split}
\end{equation*}
Varmelegemet har ti tråde med denne resistans i parallelforbindelse.
Altså er hele varmelegemets resistans
\begin{equation*}
\begin{split}
  R_{\text{legeme} }&=\left(\frac{10}{R_{\text{tråd} }}\right)^{-1}\\
  &\approx 1,1 \;\unit{\ohm} 
\end{split}
\end{equation*}
Altså har hver af trådene en resistans på $11 \;\unit{\ohm} $, og hele varmelegemet har en resistans på $1,1 \;\unit{\ohm} $. \\[1ex]
\textbf{b.}
Vi bruger da Ohms 2. lov om en spændingskilde, hvor der gælder
\[
U_0=(R_i+R_y) \cdot I
\] 
hvor $U_0$ er hvilespændingen, $R_i$ er den indre resistans og $R_y$ er den ydre resistans. 
Strømstyrken i kredsløbet er da
\begin{equation*}
\begin{split}
  I&=\frac{U_0}{R_i+R_y}\\ 
  &=\frac{12,0 \;\unit{V} }{0,17 \;\unit{\ohm} + R_{\text{legeme} }}\\ 
  &\approx 9,2 \;\unit{A} 
\end{split}
\end{equation*}
Da vi nu både kender strømstyrken og den ydre resistans, kan vi regne polspændingen ud.
\begin{equation*}
\begin{split}
  U_{\text{pol} }&=R_y \cdot I\\ 
  &=R_{\text{legeme} } \cdot I\\ 
  &\approx 10 \;\unit{V} 
\end{split}
\end{equation*}
Altså er strømstyrken i kredsløbet $9,2 \;\unit{A} $, hvor polspændingen er $10 \;\unit{V} $.\\[1ex]
\textbf{c.}
Vi kender allerede strømstyrken i kredsløbet.
Tiden, som batteriet kan holde varmelegemet tændt i er da
\begin{equation*}
\begin{split}
  t&=\frac{46 \;\unit{Ah} }{I}\\ 
  &\approx 5,0 \;\unit{h} 
\end{split}
\end{equation*}
Altså kan batteret holde varmelegemet tændt i $5,0$ timer.
\newpage
\begin{question}{Højspændingsledning}{}
  En højspændingsledning er $19,6 \;\unit{km}$ lang og har tværsnitsarealet $712 \;\unit{mm^2} $. 
  Den består af en kerne af jern omgivet af aluminium. Jernkernen har tværsnitsarealet $85 \;\unit{mm^2} $.
Resistiviteten ved $20 \;\unit{\celsius} $ for jern er $0,101 \;\unit{\ohm \cdot mm^2/m} $, og for aluminium er den $0,0272 \;\unit{\ohm \cdot mm^2/m} $.
\begin{itemize}
  \item[a.] Beregn resistansen af jernkernen i ledningen en sommerdag, hvor temperaturen er $20 \;\unit{\celsius} $.
  \item[b.] Beregn ledningens samlede resistans samme dag.
\end{itemize}
\end{question}
\sol \\
\textbf{a.}
Resistansen af jernkernen ved $20 \;\unit{\celsius} $ er da
\begin{equation*}
\begin{split}
  R_{\text{jern} }&=\rho \cdot \frac{l}{A}\\
  &=0,101 \;\unit{\ohm} \cdot \frac{mm^2}{m} \cdot \frac{19,6 \cdot 10^3 \;\unit{m} }{85 \;\unit{mm^2} }\\ 
  &\approx 23 \;\unit{\ohm} 
\end{split}
\end{equation*}
Altså er resistansen ved $20 \;\unit{\celsius} $ af jernkernen i ledningen $23 \;\unit{\ohm} $.\\[1ex]
\textbf{b.}
Vi regner først ledningens samlede resistivitet ved at tage et vægtet gennemsnit af de to givne resistiviteter.
\begin{equation*}
\begin{split}
  \rho_{\text{samlet}}&=\frac{85}{712} \cdot 0,101 \;\unit{\ohm \cdot mm^2/m} + \frac{712-85}{712}\cdot 0,0272 \;\unit{\ohm \cdot mm^2/m} \\ 
  &=\left(\frac{85}{712} \cdot 0,101 + \frac{627}{712} \cdot 0,0272\right) \;\unit{\ohm \cdot mm^2/m} 
\end{split}
\end{equation*}
Vi kan nu regne den samlede resistans for ledningen
\begin{equation*}
\begin{split}
  R_{\text{samlet} }&=\rho_{\text{samlet} } \cdot \frac{l}{A}\\ 
  &=\rho_{\text{samlet} } \cdot \frac{19,6 \cdot 10^3 \;\unit{m} }{712 \;\unit{mm^2} }\\ 
  &\approx 0,99 \;\unit{\ohm} 
\end{split}
\end{equation*}
Altså er ledningens samlede resistans på den samme dag $0,99 \;\unit{\ohm} $.
\end{document}
