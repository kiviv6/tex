\documentclass{article}
\usepackage{graphicx, tikz-cd} % Required for inserting images
\usepackage{gensymb}
\usepackage{amsmath, amssymb, amsthm, amsfonts, siunitx, physics}
\AtBeginDocument{\RenewCommandCopy\qty\SI}
\usepackage[version=4]{mhchem}
\usepackage[most,many,breakable]{tcolorbox}
\usepackage{xcolor, fancyhdr, varwidth}
\usepackage[Glenn]{fncychap}
%Options: Sonny, Lenny, Glenn, Conny, Rejne, Bjarne, Bjornstrup
\usepackage{hyperref, cleveref}
\usepackage{icomma, enumitem} %comma as decimal and continue enumerate with [resume]
%%%%%%%%%%%%%%%%%%%%%%%%%%%%%%
% SELF MADE COLORS
%%%%%%%%%%%%%%%%%%%%%%%%%%%%%%
\definecolor{myg}{RGB}{56, 140, 70}
\definecolor{myb}{RGB}{45, 111, 177}
\definecolor{myr}{RGB}{199, 68, 64}
\definecolor{mytheorembg}{HTML}{F2F2F9}
\definecolor{mytheoremfr}{HTML}{00007B}
\definecolor{mylenmabg}{HTML}{FFFAF8}
\definecolor{mylenmafr}{HTML}{983b0f}
\definecolor{mypropbg}{HTML}{f2fbfc}
\definecolor{mypropfr}{HTML}{191971}
\definecolor{myexamplebg}{HTML}{F2FBF8}
\definecolor{myexamplefr}{HTML}{88D6D1}
\definecolor{myexampleti}{HTML}{2A7F7F}
\definecolor{mydefinitbg}{HTML}{E5E5FF}
\definecolor{mydefinitfr}{HTML}{3F3FA3}
\definecolor{notesgreen}{RGB}{0,162,0}
\definecolor{myp}{RGB}{197, 92, 212}
\definecolor{mygr}{HTML}{2C3338}
\definecolor{myred}{RGB}{127,0,0}
\definecolor{myyellow}{RGB}{169,121,69}
\definecolor{myexercisebg}{HTML}{F2FBF8}
\definecolor{myexercisefg}{HTML}{88D6D1}
%%%%%%%%%%%%%%%%%%%%%%%%%%%%%%%%%%%%%%%%%%%%%%%%%%%%%%%%%%%%%%%%%%%%%%
% Box environments for theorems and problems
%%%%%%%%%%%%%%%%%%%%%%%%%%%%%%%%%%%%%%%%%%%%%%%%%%%%%%%%%%%%%%%%%%%%%
\setlength{\parindent}{1cm}
%================================
% Question BOX
%================================
\makeatletter
\newtcbtheorem{question}{Opgave}{enhanced,
	breakable,
	colback=white,
	colframe=myb!80!black,
	attach boxed title to top left={yshift*=-\tcboxedtitleheight},
	fonttitle=\bfseries,
	title={#2},
	boxed title size=title,
	boxed title style={%
			sharp corners,
			rounded corners=northwest,
			colback=tcbcolframe,
			boxrule=0pt,
		},
	underlay boxed title={%
			\path[fill=tcbcolframe] (title.south west)--(title.south east)
			to[out=0, in=180] ([xshift=5mm]title.east)--
			(title.center-|frame.east)
			[rounded corners=\kvtcb@arc] |-
			(frame.north) -| cycle;
		},
	#1
}{def}
\makeatother
%================================
% DEFINITION BOX
%================================

\newtcbtheorem[]{Definition}{Definition}{enhanced,
	before skip=2mm,after skip=2mm, colback=red!5,colframe=red!80!black,boxrule=0.5mm,
	attach boxed title to top left={xshift=1cm,yshift*=1mm-\tcboxedtitleheight}, varwidth boxed title*=-3cm,
	boxed title style={frame code={
					\path[fill=tcbcolback]
					([yshift=-1mm,xshift=-1mm]frame.north west)
					arc[start angle=0,end angle=180,radius=1mm]
					([yshift=-1mm,xshift=1mm]frame.north east)
					arc[start angle=180,end angle=0,radius=1mm];
					\path[left color=tcbcolback!60!black,right color=tcbcolback!60!black,
						middle color=tcbcolback!80!black]
					([xshift=-2mm]frame.north west) -- ([xshift=2mm]frame.north east)
					[rounded corners=1mm]-- ([xshift=1mm,yshift=-1mm]frame.north east)
					-- (frame.south east) -- (frame.south west)
					-- ([xshift=-1mm,yshift=-1mm]frame.north west)
					[sharp corners]-- cycle;
				},interior engine=empty,
		},
	fonttitle=\bfseries,
	title={#2},#1}{def}
\newtcbtheorem[]{definition}{Definition}{enhanced,
	before skip=2mm,after skip=2mm, colback=red!5,colframe=red!80!black,boxrule=0.5mm,
	attach boxed title to top left={xshift=1cm,yshift*=1mm-\tcboxedtitleheight}, varwidth boxed title*=-3cm,
	boxed title style={frame code={
					\path[fill=tcbcolback]
					([yshift=-1mm,xshift=-1mm]frame.north west)
					arc[start angle=0,end angle=180,radius=1mm]
					([yshift=-1mm,xshift=1mm]frame.north east)
					arc[start angle=180,end angle=0,radius=1mm];
					\path[left color=tcbcolback!60!black,right color=tcbcolback!60!black,
						middle color=tcbcolback!80!black]
					([xshift=-2mm]frame.north west) -- ([xshift=2mm]frame.north east)
					[rounded corners=1mm]-- ([xshift=1mm,yshift=-1mm]frame.north east)
					-- (frame.south east) -- (frame.south west)
					-- ([xshift=-1mm,yshift=-1mm]frame.north west)
					[sharp corners]-- cycle;
				},interior engine=empty,
		},
	fonttitle=\bfseries,
	title={#2},#1}{def}

\newtcbtheorem{theo}%
    {Theorem}{}{theorem}
\newtcolorbox{prob}[1]{colback=red!5!white,colframe=red!50!black,fonttitle=\bfseries,title={#1}}
%================================
% NOTE BOX
%================================

\usetikzlibrary{arrows,calc,shadows.blur}
\tcbuselibrary{skins}
\newtcolorbox{note}[1][]{%
	enhanced jigsaw,
	colback=gray!20!white,%
	colframe=gray!80!black,
	size=small,
	boxrule=1pt,
	title=\textbf{Note:},
	halign title=flush center,
	coltitle=black,
	breakable,
	drop shadow=black!50!white,
	attach boxed title to top left={xshift=1cm,yshift=-\tcboxedtitleheight/2,yshifttext=-\tcboxedtitleheight/2},
	minipage boxed title=1.5cm,
	boxed title style={%
			colback=white,
			size=fbox,
			boxrule=1pt,
			boxsep=2pt,
			underlay={%
					\coordinate (dotA) at ($(interior.west) + (-0.5pt,0)$);
					\coordinate (dotB) at ($(interior.east) + (0.5pt,0)$);
					\begin{scope}
						\clip (interior.north west) rectangle ([xshift=3ex]interior.east);
						\filldraw [white, blur shadow={shadow opacity=60, shadow yshift=-.75ex}, rounded corners=2pt] (interior.north west) rectangle (interior.south east);
					\end{scope}
					\begin{scope}[gray!80!black]
						\fill (dotA) circle (2pt);
						\fill (dotB) circle (2pt);
					\end{scope}
				},
		},
	#1,
}

%%%%%%%%%%%%%%%%%%%%%%%%%%%%%%%%%%%%%%%%%%%%%%%%%%%%%%%%%%%%%%%%%
% SELF MADE COMMANDS
%%%%%%%%%%%%%%%%%%%%%%%%%%%%%%
\newcommand{\sol}{\setlength{\parindent}{0cm}\textbf{\textit{Løsning:}}\setlength{\parindent}{1cm}}
%%%%%%%%%%%%%%%%%%%%%%%%%%%%%%%%%
\usepackage[tmargin=2cm,rmargin=1in,lmargin=1in,margin=0.85in,bmargin=2cm,footskip=.2in]{geometry}\pagestyle{fancy}
\lhead{Minrui Kevin Zhou 2.b}
\rhead{Matematikprøve 1}

\title{Matematikprøve 1\\
{\Large \textbf{2.b mat A}}}
\author{Navn: Kevin Zhou \\ 
Nummer: 22}
\date{Oktober 2023}

\begin{document}
\maketitle
\section*{Bedømmelseskriterier:}
\begin{itemize}
    \setlength\itemsep{3cm}
    \Large
    \item  Redegørelse og dokumentation for metode
    \item Figurer, grafer og andre illustrationer
    \item Notation og layout
    \item Formidling og forklaring
\end{itemize}
\pagebreak

\begin{question}{}{}
Omsætningen hos madleverandøren Aarstiderne kan efter 2013 beskrives med modellen
\[
f(x)= 301 \cdot 1,227^x,
\] 
  hvor betegner antallet af år efter 2013, og $f(x)$ er den årlige omsætning i millioner kr. 
\begin{itemize}
  \item[a.] Hvad fortæller tallet 1,227 om udviklingen af omsætningen?
  \item[b.] Bestem fordoblingstiden for omsætningen.
\end{itemize}
\end{question}
\sol \\ 
\textbf{a.} Tallet $1,227$ fortæller, hvor meget omsætningen udvikler sig hvert år, da vækstraten, $r$, blot er
\[
r=1,227-1=0,227
\] 
Det vil sige, at omsætningen hvert år vokser med $22,7 \%$.\\[1ex]
\textbf{b.} Siden fordoblingstiden er et udtryk for, hvor lang tid det tager for omsætningen at vokse med $100\%$, kan vi benytte resultatet i \textbf{a.} og få, at fordoblingstiden er følgende.
\[
T_2=\log_{1,227}(2) \approx 3,388
\] 
Altså er fordoblingstiden for omsætningen $3,388$ år. 

\begin{question}{}{}
Funktionen $f:\mathbb{R} \to \mathbb{R}$ er givet ved
\[
f(x)= \frac{1}{2}x^2 -x + \frac{5}{2}.
\] 
\begin{itemize}
  \item[a.] Gør rede for, at punkterne $P(-1,4)$ og $Q(5,10)$ ligger på grafen for $f$.
  \item[b.] Undersøg, om linjen gennem $P$ og $Q$ er parallel med tangenten til grafen for $f$ i punktet med førstekoordinat 2. 
\end{itemize}
\end{question}
\sol \\ 
\textbf{a.} Det er klart, at punkterne $P(-1,4)$ og $Q(5,10)$ ligger på grafen for $f$, hvis og kun hvis
\[
f(-1)= 4 \land f(5)= 10
\] 
Dette vil vi nu vise. 
\begin{equation*}
\begin{split}
  f(-1)&=\frac{1}{2}(-1)^2 -(-1) + \frac{5}{2} \\ 
  &= \frac{3}{2}+\frac{5}{2}\\ 
  &= 4
\end{split}
\end{equation*}
Altså ligger $P(-1,4)$ på grafen for $f$. 
\begin{equation*}
\begin{split}
  f(5)&= \frac{1}{2}5^2 -5 + \frac{5}{2} \\
  &=\frac{6\cdot5}{2}-5 \\ 
  &= 15 - 5 \\
  &= 10
\end{split}
\end{equation*}
Altså ligger $Q(5,10)$ også på grafen for $f$.\\[1ex]
\textbf{b.} To linjer er parallelle, hvis deres hældning er ens. Linjen gennem $P$ og $Q$ har da hældningen
\[
\frac{\Delta y}{\Delta x}=\frac{10-4}{5-(-1)}=1
\] 
For at finde hældningen til tangenten til grafen for $f$, når $x=2$, kan vi benytte den geometriske betydning af differentialkvotienten $f'(2)$, der netop er denne. 
\[
f'(x)= x-1 \implies f'(2)= 1
\] 
Det ses da, at disse to linjer har samme hældning. Altså er linjen gennem $P$ og $Q$ parallel med tangenten til grafen for $f$ i punktet med førstekoordinat 2. 

\begin{question}{}{}
Linjerne $l$ og $m$ er givet ved
\[
  l: 5x-2y+1=0
\] 
\[
  m: 4x+3y-13=0
\] 
\begin{itemize}
  \item[a.] Bestem koordinatsættet til skæringspunktet mellem $l$ og $m$.
  \item[b.] Bestem den spidse vinkel $v$ mellem $l$ og $m$. 
\end{itemize}
\end{question}
\sol \\ 
\textbf{a.} Ved skæringspunktet er $x$-værdierne og $y$-værdierne ens. Da kan ligningerne ses som et ligningssystem, der skal løses. 
\begin{equation*}
\begin{split}
  5x-2y+1=0\land 4x+3y-13=0 &\implies y=\frac{5x+1}{2}\land 4x+3y-13=0 \\ 
  &\implies 4x + 3\cdot \frac{5x+1}{2} -13=0 \\ 
  &\iff (4+\frac{15}{2}\cdot x)=13-\frac{3}{2} \\ 
  &\iff \frac{23}{2}x=\frac{23}{2}\\ 
  &\iff x=1
\end{split}
\end{equation*}
$y$ kan nu findes.
\[
y=\frac{5x+1}{2}=\frac{6}{2}=3
\] 
Koordinatsættet til skæringspunktet mellem $l$ og $m$ er altså $(1,3)$. \\[1ex]
\textbf{b.} 
Hældningen for $l$ kendes fra \textbf{a.}, men skal findes for $m$.
\[
4x+3y-13=0 \iff y=-\frac{4}{3}x+\frac{13}{3}
\] 
Hældningen for $m$ er altså $-\frac{4}{3}$.
Vinklen $v$ mellem linjerne må da være følgende, grundet hældningerne for linjerne.
\[
v=180\degree - \tan^{-1}\left(\frac{5}{2}\right)-\tan^{-1}\left(-\frac{4}{3}\right) \approx 58,67\degree. 
\] 
\end{document}
