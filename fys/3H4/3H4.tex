\documentclass{report}
\usepackage{graphicx, tikz-cd, float, titlepic, booktabs} % Required for inserting images
\usepackage{pgfplots}
\pgfplotsset{compat=1.15}
\usepackage{mathrsfs}
\usetikzlibrary{arrows}
\usepackage{amsmath, amssymb, amsthm, amsfonts, siunitx, physics, gensymb}
\AtBeginDocument{\RenewCommandCopy\qty\SI}
\usepackage[version=4]{mhchem}
\usepackage[most,many,breakable]{tcolorbox}
\usepackage{xcolor, fancyhdr, varwidth}
\usepackage[Glenn]{fncychap}
%Options: Sonny, Lenny, Glenn, Conny, Rejne, Bjarne, Bjornstrup
\usepackage{hyperref, cleveref}
\usepackage{icomma, enumitem} %comma as decimal and continue enumerate with [resume]
\usepackage{plimsoll} %use standard state symbol with \stst
\usepackage[danish]{babel}
%%%%%%%%%%%%%%%%%%%%%%%%%%%%%%
% SELF MADE COLORS
%%%%%%%%%%%%%%%%%%%%%%%%%%%%%%
\definecolor{myg}{RGB}{56, 140, 70}
\definecolor{myb}{RGB}{45, 111, 177}
\definecolor{myr}{RGB}{199, 68, 64}
\definecolor{mytheorembg}{HTML}{F2F2F9}
\definecolor{mytheoremfr}{HTML}{00007B}
\definecolor{mylenmabg}{HTML}{FFFAF8}
\definecolor{mylenmafr}{HTML}{983b0f}
\definecolor{mypropbg}{HTML}{f2fbfc}
\definecolor{mypropfr}{HTML}{191971}
\definecolor{myexamplebg}{HTML}{F2FBF8}
\definecolor{myexamplefr}{HTML}{88D6D1}
\definecolor{myexampleti}{HTML}{2A7F7F}
\definecolor{mydefinitbg}{HTML}{E5E5FF}
\definecolor{mydefinitfr}{HTML}{3F3FA3}
\definecolor{notesgreen}{RGB}{0,162,0}
\definecolor{myp}{RGB}{197, 92, 212}
\definecolor{mygr}{HTML}{2C3338}
\definecolor{myred}{RGB}{127,0,0}
\definecolor{myyellow}{RGB}{169,121,69}
\definecolor{myexercisebg}{HTML}{F2FBF8}
\definecolor{myexercisefg}{HTML}{88D6D1}
%%%%%%%%%%%%%%%%%%%%%%%%%%%%%%%%%%%%%%%%%%%%%%%%%%%%%%%%%%%%%%%%%%%%%%
% Box environments for theorems and problems
%%%%%%%%%%%%%%%%%%%%%%%%%%%%%%%%%%%%%%%%%%%%%%%%%%%%%%%%%%%%%%%%%%%%%
\setlength{\parindent}{1cm}
%================================
% Question BOX
%================================
\makeatletter
\newtcbtheorem{question}{Opgave}{enhanced,
	breakable,
	colback=white,
	colframe=myb!80!black,
	attach boxed title to top left={yshift*=-\tcboxedtitleheight},
	fonttitle=\bfseries,
	title={#2},
	boxed title size=title,
	boxed title style={%
			sharp corners,
			rounded corners=northwest,
			colback=tcbcolframe,
			boxrule=0pt,
		},
	underlay boxed title={%
			\path[fill=tcbcolframe] (title.south west)--(title.south east)
			to[out=0, in=180] ([xshift=5mm]title.east)--
			(title.center-|frame.east)
			[rounded corners=\kvtcb@arc] |-
			(frame.north) -| cycle;
		},
	#1
}{def}
\makeatother
%================================
% DEFINITION BOX
%================================

\newtcbtheorem[]{Definition}{Definition}{enhanced,
	before skip=2mm,after skip=2mm, colback=red!5,colframe=red!80!black,boxrule=0.5mm,
	attach boxed title to top left={xshift=1cm,yshift*=1mm-\tcboxedtitleheight}, varwidth boxed title*=-3cm,
	boxed title style={frame code={
					\path[fill=tcbcolback]
					([yshift=-1mm,xshift=-1mm]frame.north west)
					arc[start angle=0,end angle=180,radius=1mm]
					([yshift=-1mm,xshift=1mm]frame.north east)
					arc[start angle=180,end angle=0,radius=1mm];
					\path[left color=tcbcolback!60!black,right color=tcbcolback!60!black,
						middle color=tcbcolback!80!black]
					([xshift=-2mm]frame.north west) -- ([xshift=2mm]frame.north east)
					[rounded corners=1mm]-- ([xshift=1mm,yshift=-1mm]frame.north east)
					-- (frame.south east) -- (frame.south west)
					-- ([xshift=-1mm,yshift=-1mm]frame.north west)
					[sharp corners]-- cycle;
				},interior engine=empty,
		},
	fonttitle=\bfseries,
	title={#2},#1}{def}
\newtcbtheorem[]{definition}{Definition}{enhanced,
	before skip=2mm,after skip=2mm, colback=red!5,colframe=red!80!black,boxrule=0.5mm,
	attach boxed title to top left={xshift=1cm,yshift*=1mm-\tcboxedtitleheight}, varwidth boxed title*=-3cm,
	boxed title style={frame code={
					\path[fill=tcbcolback]
					([yshift=-1mm,xshift=-1mm]frame.north west)
					arc[start angle=0,end angle=180,radius=1mm]
					([yshift=-1mm,xshift=1mm]frame.north east)
					arc[start angle=180,end angle=0,radius=1mm];
					\path[left color=tcbcolback!60!black,right color=tcbcolback!60!black,
						middle color=tcbcolback!80!black]
					([xshift=-2mm]frame.north west) -- ([xshift=2mm]frame.north east)
					[rounded corners=1mm]-- ([xshift=1mm,yshift=-1mm]frame.north east)
					-- (frame.south east) -- (frame.south west)
					-- ([xshift=-1mm,yshift=-1mm]frame.north west)
					[sharp corners]-- cycle;
				},interior engine=empty,
		},
	fonttitle=\bfseries,
	title={#2},#1}{def}

\newtcbtheorem{theo}%
    {Theorem}{}{theorem}
\newtcolorbox{prob}[1]{colback=red!5!white,colframe=red!50!black,fonttitle=\bfseries,title={#1}}
%================================
% NOTE BOX
%================================

\usetikzlibrary{arrows,calc,shadows.blur}
\tcbuselibrary{skins}
\newtcolorbox{note}[1][]{%
	enhanced jigsaw,
	colback=gray!20!white,%
	colframe=gray!80!black,
	size=small,
	boxrule=1pt,
	title=\textbf{Note:},
	halign title=flush center,
	coltitle=black,
	breakable,
	drop shadow=black!50!white,
	attach boxed title to top left={xshift=1cm,yshift=-\tcboxedtitleheight/2,yshifttext=-\tcboxedtitleheight/2},
	minipage boxed title=1.5cm,
	boxed title style={%
			colback=white,
			size=fbox,
			boxrule=1pt,
			boxsep=2pt,
			underlay={%
					\coordinate (dotA) at ($(interior.west) + (-0.5pt,0)$);
					\coordinate (dotB) at ($(interior.east) + (0.5pt,0)$);
					\begin{scope}
						\clip (interior.north west) rectangle ([xshift=3ex]interior.east);
						\filldraw [white, blur shadow={shadow opacity=60, shadow yshift=-.75ex}, rounded corners=2pt] (interior.north west) rectangle (interior.south east);
					\end{scope}
					\begin{scope}[gray!80!black]
						\fill (dotA) circle (2pt);
						\fill (dotB) circle (2pt);
					\end{scope}
				},
		},
	#1,
}
%================================
% EXAMPLE BOX
%================================
\newtcbtheorem[number within=section]{Example}{Example}
{%
	colback = myexamplebg
	,breakable
	,colframe = myexamplefr
	,coltitle = myexampleti
	,boxrule = 1pt
	,sharp corners
	,detach title
	,before upper=\tcbtitle\par\smallskip
	,fonttitle = \bfseries
	,description font = \mdseries
	,separator sign none
	,description delimiters parenthesis
}
{ex}
%================================
% THEOREM BOX
%================================

\tcbuselibrary{theorems,skins,hooks}
\newtcbtheorem[number within=section]{Theorem}{Theorem}
{%
	enhanced,
	breakable,
	colback = mytheorembg,
	frame hidden,
	boxrule = 0sp,
	borderline west = {2pt}{0pt}{mytheoremfr},
	sharp corners,
	detach title,
	before upper = \tcbtitle\par\smallskip,
	coltitle = mytheoremfr,
	fonttitle = \bfseries\sffamily,
	description font = \mdseries,
	separator sign none,
	segmentation style={solid, mytheoremfr},
}
{th}

%%%%%%%%%%%%%%%%%%%%%%%%%%%%%%%%%%%%%%%%%%%%%%%%%%%%%%%%%%%%%%%%%
% SELF MADE COMMANDS
%%%%%%%%%%%%%%%%%%%%%%%%%%%%%%
\newcommand{\sol}{\setlength{\parindent}{0cm}\textbf{\textit{Løsning:}}\setlength{\parindent}{1cm}}
%%%%%%%%%%%%%%%%%%%%%%%%%%%%%%%%%
\usepackage[tmargin=2cm,rmargin=1in,lmargin=1in,margin=0.85in,bmargin=2cm,footskip=.2in]{geometry}\pagestyle{fancy}
\lhead{Minrui Kevin Zhou 3.b}
\rhead{H4: Mekanik}

\title{H4: Mekanik – fjedre og bevægelsesmængde\\
{\Large \textbf{3.b fysik A}}}
\author{Minrui Kevin Zhou}
\date{\today}

\begin{document}
\maketitle
\begin{question}{Kortbane}{}
Massen af løber B er $72 \;\unit{kg} $, og hans fart lige før skubbet er $13,6 \;\unit{m/s} $. 
  Massen af løber A er $79 \;\unit{kg} $, og hans fart lige før skubbet er $12,3 \;\unit{m/s} $. 
  Lige efter skubbet har løber B farten $8,9 \;\unit{m/s}$. 
  Løberne bevæger sig i samme retning under skubbet.
  \begin{itemize}
    \item[a.] Beregn farten af løber A lige efter skubbet.
  \end{itemize}
\end{question}
\sol \\
\textbf{a.}
Siden der er tale om kollision uden påvirkning af kræfter udefra, så er de to løberes bevægelsesmængde bevaret.
Der gælder da 
\begin{equation*}
\begin{split}
  p _{A (\text{før} )} + p _{B (\text{før} )}=p_{A (\text{efter} )} + p _{B (\text{efter} )} &\iff m _{A} \cdot v _{A (\text{før} )} + m _{B} \cdot v _{B (\text{før} )} = m _{A} \cdot v _{A (\text{efter} )} + m _{B} \cdot v _{B (\text{efter} )}\\
  &\iff v _{A (\text{efter}) }=\frac{m _{A} \cdot v _{A (\text{før} )}+ m _{B} \cdot \left(v _{B (\text{før} )}- v _{B (\text{efter} )}\right) }{m _{A}}\\
  &\iff v _{A (\text{efter}) }=v _{A (\text{før} )}+ \frac{m _{B} }{m _{A}} \cdot \left(v _{B (\text{før} )}- v _{B (\text{efter} )}\right) 
\end{split}
\end{equation*}
Vi beregner nu farten af løber A efter skubbet
\begin{equation*}
\begin{split}
  v _{A (\text{efter}) }&=v _{A (\text{før} )}+ \frac{m _{B}}{m _{A}} \cdot \left(v _{B (\text{før} )}- v _{B (\text{efter} )}\right) \\
  &=12,3 \;\unit{m/s} + \frac{72 \;\unit{kg} }{79 \;\unit{kg} } \cdot \left(13,6 \;\unit{m/s} - 8,9 \;\unit{m/s} \right) \\
  &\approx 17 \;\unit{m/s} 
\end{split}
\end{equation*}
Altså er farten af løber A lige efter skubbet $17 \;\unit{m/s} $.

\begin{question}{Sikkerhedsnet til klipper}{}
I et eksperiment lader virksomheden en stor betonklods falde ned i et vandret udspændt sikkerhedsnet. Betonklodsen falder frit $39 \;\unit{ m}$, inden den rammer nettet. 
\begin{itemize}
  \item[a.] Hvor lang tid tager det betonklodsen at falde ned til sikkerhedsnettet?
\end{itemize}
Når betonklodsen rammer sikkerhedsnettet bremses den af en elastisk kraft, idet nettet udstrækkes $7,58 \;\unit{m} $. Betonklodsen har massen 22 ton.
\begin{itemize}
  \item[b.] Bestem fjederkonstanten for sikkerhedsnettet.
\end{itemize}
\end{question}
\sol \\
\textbf{a.}
Når vi ser bort fra luftmodstand er der tale om en jævnt accelereret bevægelse.
Derfor gælder der, at 
\begin{equation*}
\begin{split}
  s=\frac{1}{2} \cdot g \cdot t^2 &\iff t^2=\frac{2s}{g}\\
  &\iff t=\sqrt{\frac{2s}{g}} 
\end{split}
\end{equation*}
hvor $s$ er strækningen, som klodsen falder, og vi lader $t$ være en positiv størrelse. 

Vi beregner nu tiden $t$
\begin{equation*}
\begin{split}
  t&=\sqrt{\frac{2s}{g}} \\
  &=\sqrt{\frac{2 \cdot 39 \;\unit{m} }{9,8 \;\unit{m/s^2} }} \\
  &\approx 2,8 \;\unit{s} 
\end{split}
\end{equation*}
Altså tager det klodsen $2,8 \;\unit{s} $ at falde ned til sikkerhedsnettet.\\[1ex]
\textbf{b.}
Når nettet er helt udstrakt, er alt klodsens mekaniske energi i form af potentiel energi.
Da klodsen ikke påvirkes af nogen ydre krafter, så er den mekaniske energi bevaret.
Altså må der gælde, at 
\begin{equation*}
\begin{split}
  m \cdot g \cdot h = \frac{1}{2} \cdot k \cdot x^2 \iff k = \frac{2 \cdot m \cdot g \cdot h}{x^2}
\end{split}
\end{equation*}
Vi kan da beregne fjederkonstanten.
\begin{equation*}
\begin{split}
  k &= \frac{2 \cdot m \cdot g \cdot h}{x^2}\\ 
  &=\frac{2 \cdot 22 \cdot 10^3 \;\unit{kg} \cdot 9,82 \;\unit{m/s^2} \cdot \left(39 \;\unit{m} + 7,58 \;\unit{m} \right) }{\left(7,58 \;\unit{m} \right)^2}\\
  &\approx 3,5 \cdot 10^5 \;\unit{N/m} 
\end{split}
\end{equation*}
Fjederkonstanten for sikkerhedsnettet er altså $3,5 \cdot 10^5 \;\unit{N/m} $.

\begin{question}{Amerikansk fodbold}{}
En spiller accelererer fra hvile op til farten $8,1 \;\unit{ m/s}$ i løbet af $1,8 \;\unit{s} $.
\begin{itemize}
  \item[a.] Bestem størrelsen af spillerens acceleration.
\end{itemize}

Målet i amerikansk fodbold ser ud som vist på figuren. For at score point skal bolden sparkes over overliggeren i målet.

I 40 meters afstand fra målet sparker en spiller til bolden. Herved sendes bolden af sted med farten $24 \;\unit{ m/s}$ og i en vinkel på $27 \degree $ med vandret. Man kan se bort fra luftmodstanden.
\begin{itemize}
  \item[b.] Beregn den tid, det tager bolden at tilbagelægge 40 m i vandret retning. 
Bestem den højde over jorden, i hvilken bolden passerer målet.
\end{itemize}

En forsvarer kan eksempelvis stoppe en angriber ved at gribe fat i ham. En angriber med massen $82 \;\unit{ kg}$ løber frem ad banen med farten $7,6 \;\unit{ m/s}$.
En forsvarer løber fra siden med farten $6,2 \;\unit{ m/s}$ lige ind i angriberen. Se figuren på næste side. Forsvareren har massen 95 kg.

Efter sammenstødet falder begge spillere og glider hen over græsset med samme hastighed. Gnidningskoefficienten mellem spillerne og græsset er 0,70.
\begin{itemize}
  \item[c.] Vurdér, hvor langt de to spillere glider hen over græsset efter sammenstødet.
\end{itemize}
\end{question}
\sol \\
\textbf{a.}
Størrelsen spillerens gennemsnitlige acceleration i tidsrummet må være 
\begin{equation*}
\begin{split}
  a&=\frac{\Delta v}{\Delta t}\\
  &=\frac{8,1 \;\unit{m/s} }{1,8 \;\unit{s} }\\
  &=4,5 \;\unit{m/s^2} 
\end{split}
\end{equation*}
Størrelsen af spillerens acceleration er altså $4,5 \;\unit{m/s^2} $.\\[1ex]
\textbf{b.}
Lad boldens begyndelsessted være $(x,y)=(0,0)$.
Så har vi fra uafhængighedsprincippet, at
\[
x=v_0 \cdot \cos\left(\theta \right) \cdot t \iff t=\frac{x}{v_0 \cdot \cos\left(\theta \right) }
\] 
hvor $\theta $ er elevationsvinklen. 
Vi beregner nu den tid, det tager bolden at tilbagelægge $40 \;\unit{m} $ i vandret retning. 
\begin{equation*}
\begin{split}
  t&=\frac{x}{v_0 \cdot \cos\left(\theta \right) }\\
  &=\frac{40 \;\unit{m} }{24 \;\unit{m/s} \cdot \cos\left(27 \degree \right) }\\
  &\approx 1,9 \;\unit{s} 
\end{split}
\end{equation*}
For at bestemme den højde over jorden, i hvilken bolden passerer målet, bruger vi igen uafhængighedsprincippet.
\begin{equation*}
\begin{split}
  y&=-\frac{1}{2} \cdot g \cdot t^2 + v_0 \cdot \sin\left(\theta \right) \cdot t\\
  &=-\frac{1}{2} \cdot 9,8 \;\unit{m/s^2} \cdot t^2 + 24 \;\unit{m/s} \cdot \sin\left(27 \degree \right) \cdot t \\
  &\approx 3,2 \;\unit{m} 
\end{split}
\end{equation*}
Altså tager det bolden $1,9 \;\unit{s} $ at nå til målet, hvor den passerer $3,2 \;\unit{m} $ over jorden. \\[1ex]
\textbf{c.}
Da spillerne følges ad med fælles hastighed efter, er der tale om et fuldstændigt uelastisk stød.
Deres fart $v_0$ lige efter stødet må da være
\begin{equation*}
\begin{split}
  v_0&=\frac{m_1 \cdot v_1 + m_2 \cdot v_2}{m_1+m_2}\\
  &=\frac{98 \;\unit{kg} \cdot 5,9 \;\unit{m/s} + 79 \;\unit{kg} \cdot 8,1 \;\unit{m/s} }{98 \;\unit{kg} + 79 \;\unit{kg} }\\
  &=\frac{1218,1}{177} \;\unit{m/s} 
\end{split}
\end{equation*}
Gnidningskraften er modsatrettet hastigheden, hvis retning vi regner i, og siden der er vandret underlag, så må normalkraften være lig med tyngdekraften: $F_N=F_t$.
Der gælder altså
\begin{equation*}
\begin{split}
  F _{\text{res} }=-F _{\mu } &\iff m \cdot a = -\mu \cdot F_N\\
  &\iff a=-\frac{\mu \cdot F_t}{m}\\
  &\iff a=-\mu \cdot g
\end{split}
\end{equation*}
Efter sammenstødet er der så tale om en konstant accelereret bevægelse, og når spillerne er nået så langt som muligt, så er $v_{\text{slut} }=0$, og vi har da 
\begin{equation*}
\begin{split}
  \Delta s&=\frac{v _{slut}^2-v_0^2}{2 \cdot a}\\
  &=\frac{v_0^2-0}{2 \cdot \mu \cdot g}\\
  &=\frac{\left(\frac{1218,1}{177} \;\unit{m/s} \right)^2}{2 \cdot 0,70 \cdot 9,8 \;\unit{m/s^2} }\\
  &\approx 3,5 \;\unit{m} 
\end{split}
\end{equation*}
Altså glider spillerne $3,5 \;\unit{m} $ hen over græsset efter sammenstødet. 

\begin{question}{Fiskevægt}{}
Billedet viser en vægt til at veje fisk. Vægten består af en fjeder med en skala, hvor man kan aflæse fiskens masse i kg.
\begin{itemize}
  \item[a.] Tildel passende værdier til relevante fysiske størrelser, og vurdér fjederens fjederkonstant. 
  Gør herunder rede for relevante antagelser.
\end{itemize}
\end{question}
\sol \\
\textbf{a.}
Tykkelsen af min pegefinger er målt til at være $1,5 \;\unit{cm} $, hvilket vi antager, at pegefingeren på billedet også er.
Ved ligevægtstillingen må den røde del af fjederen stå ved 0.
Ved afmåling med lineal er afstanden mellem den røde streg og ligevægtsstillingen 1,5 gange tykkelsen af fingeren.
Afstanden $x$ er altså 
\[
x=1,5 \cdot 1,5 \;\unit{cm} =2,25 \;\unit{cm} 
\] 
På billedet er fjederkraften og tyngdekraften lige store og modsatrettede.
Der gælder altså
\begin{equation*}
\begin{split}
  F_t=-F_{\text{fjeder} } &\iff m \cdot g = -(-k \cdot x) \\
  &\iff k=\frac{m \cdot g}{x}
\end{split}
\end{equation*}
Massen af fisken på billedet aflæses til at være $3,4 \;\unit{kg} $, og vi kan da regne fjederkonstanten ud.
\begin{equation*}
\begin{split}
  k&=\frac{m \cdot g}{x}\\
  &=\frac{3,4 \;\unit{kg} \cdot 9,8 \;\unit{m/s^2}}{2,25 \cdot 10^{-2}\;\unit{m} }\\
  &\approx 1,5 \cdot 10^3 \;\unit{N/m} \\
  &=1,5 \;\unit{kN/m} 
\end{split}
\end{equation*}
Altså er fjederens fjederkonstant $1,5 \;\unit{kN/m} $.

\end{document}
