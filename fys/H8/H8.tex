\documentclass{report}
\usepackage{graphicx, tikz-cd, float, titlepic, booktabs} % Required for inserting images
\usepackage{amsmath, amssymb, amsthm, amsfonts, siunitx, physics, gensymb}
\AtBeginDocument{\RenewCommandCopy\qty\SI}
\usepackage[version=4]{mhchem}
\usepackage[most,many,breakable]{tcolorbox}
\usepackage{xcolor, fancyhdr, varwidth}
\usepackage[Glenn]{fncychap}
%Options: Sonny, Lenny, Glenn, Conny, Rejne, Bjarne, Bjornstrup
\usepackage{hyperref, cleveref}
\usepackage{icomma, enumitem} %comma as decimal and continue enumerate with [resume]
\usepackage[danish]{babel}
%%%%%%%%%%%%%%%%%%%%%%%%%%%%%%
% SELF MADE COLORS
%%%%%%%%%%%%%%%%%%%%%%%%%%%%%%
\definecolor{myg}{RGB}{56, 140, 70}
\definecolor{myb}{RGB}{45, 111, 177}
\definecolor{myr}{RGB}{199, 68, 64}
\definecolor{mytheorembg}{HTML}{F2F2F9}
\definecolor{mytheoremfr}{HTML}{00007B}
\definecolor{mylenmabg}{HTML}{FFFAF8}
\definecolor{mylenmafr}{HTML}{983b0f}
\definecolor{mypropbg}{HTML}{f2fbfc}
\definecolor{mypropfr}{HTML}{191971}
\definecolor{myexamplebg}{HTML}{F2FBF8}
\definecolor{myexamplefr}{HTML}{88D6D1}
\definecolor{myexampleti}{HTML}{2A7F7F}
\definecolor{mydefinitbg}{HTML}{E5E5FF}
\definecolor{mydefinitfr}{HTML}{3F3FA3}
\definecolor{notesgreen}{RGB}{0,162,0}
\definecolor{myp}{RGB}{197, 92, 212}
\definecolor{mygr}{HTML}{2C3338}
\definecolor{myred}{RGB}{127,0,0}
\definecolor{myyellow}{RGB}{169,121,69}
\definecolor{myexercisebg}{HTML}{F2FBF8}
\definecolor{myexercisefg}{HTML}{88D6D1}
%%%%%%%%%%%%%%%%%%%%%%%%%%%%%%%%%%%%%%%%%%%%%%%%%%%%%%%%%%%%%%%%%%%%%%
% Box environments for theorems and problems
%%%%%%%%%%%%%%%%%%%%%%%%%%%%%%%%%%%%%%%%%%%%%%%%%%%%%%%%%%%%%%%%%%%%%
\setlength{\parindent}{1cm}
%================================
% Question BOX
%================================
\makeatletter
\newtcbtheorem{question}{Opgave}{enhanced,
	breakable,
	colback=white,
	colframe=myb!80!black,
	attach boxed title to top left={yshift*=-\tcboxedtitleheight},
	fonttitle=\bfseries,
	title={#2},
	boxed title size=title,
	boxed title style={%
			sharp corners,
			rounded corners=northwest,
			colback=tcbcolframe,
			boxrule=0pt,
		},
	underlay boxed title={%
			\path[fill=tcbcolframe] (title.south west)--(title.south east)
			to[out=0, in=180] ([xshift=5mm]title.east)--
			(title.center-|frame.east)
			[rounded corners=\kvtcb@arc] |-
			(frame.north) -| cycle;
		},
	#1
}{def}
\makeatother
%================================
% DEFINITION BOX
%================================

\newtcbtheorem[]{Definition}{Definition}{enhanced,
	before skip=2mm,after skip=2mm, colback=red!5,colframe=red!80!black,boxrule=0.5mm,
	attach boxed title to top left={xshift=1cm,yshift*=1mm-\tcboxedtitleheight}, varwidth boxed title*=-3cm,
	boxed title style={frame code={
					\path[fill=tcbcolback]
					([yshift=-1mm,xshift=-1mm]frame.north west)
					arc[start angle=0,end angle=180,radius=1mm]
					([yshift=-1mm,xshift=1mm]frame.north east)
					arc[start angle=180,end angle=0,radius=1mm];
					\path[left color=tcbcolback!60!black,right color=tcbcolback!60!black,
						middle color=tcbcolback!80!black]
					([xshift=-2mm]frame.north west) -- ([xshift=2mm]frame.north east)
					[rounded corners=1mm]-- ([xshift=1mm,yshift=-1mm]frame.north east)
					-- (frame.south east) -- (frame.south west)
					-- ([xshift=-1mm,yshift=-1mm]frame.north west)
					[sharp corners]-- cycle;
				},interior engine=empty,
		},
	fonttitle=\bfseries,
	title={#2},#1}{def}
\newtcbtheorem[]{definition}{Definition}{enhanced,
	before skip=2mm,after skip=2mm, colback=red!5,colframe=red!80!black,boxrule=0.5mm,
	attach boxed title to top left={xshift=1cm,yshift*=1mm-\tcboxedtitleheight}, varwidth boxed title*=-3cm,
	boxed title style={frame code={
					\path[fill=tcbcolback]
					([yshift=-1mm,xshift=-1mm]frame.north west)
					arc[start angle=0,end angle=180,radius=1mm]
					([yshift=-1mm,xshift=1mm]frame.north east)
					arc[start angle=180,end angle=0,radius=1mm];
					\path[left color=tcbcolback!60!black,right color=tcbcolback!60!black,
						middle color=tcbcolback!80!black]
					([xshift=-2mm]frame.north west) -- ([xshift=2mm]frame.north east)
					[rounded corners=1mm]-- ([xshift=1mm,yshift=-1mm]frame.north east)
					-- (frame.south east) -- (frame.south west)
					-- ([xshift=-1mm,yshift=-1mm]frame.north west)
					[sharp corners]-- cycle;
				},interior engine=empty,
		},
	fonttitle=\bfseries,
	title={#2},#1}{def}

\newtcbtheorem{theo}%
    {Theorem}{}{theorem}
\newtcolorbox{prob}[1]{colback=red!5!white,colframe=red!50!black,fonttitle=\bfseries,title={#1}}
%================================
% NOTE BOX
%================================

\usetikzlibrary{arrows,calc,shadows.blur}
\tcbuselibrary{skins}
\newtcolorbox{note}[1][]{%
	enhanced jigsaw,
	colback=gray!20!white,%
	colframe=gray!80!black,
	size=small,
	boxrule=1pt,
	title=\textbf{Note:},
	halign title=flush center,
	coltitle=black,
	breakable,
	drop shadow=black!50!white,
	attach boxed title to top left={xshift=1cm,yshift=-\tcboxedtitleheight/2,yshifttext=-\tcboxedtitleheight/2},
	minipage boxed title=1.5cm,
	boxed title style={%
			colback=white,
			size=fbox,
			boxrule=1pt,
			boxsep=2pt,
			underlay={%
					\coordinate (dotA) at ($(interior.west) + (-0.5pt,0)$);
					\coordinate (dotB) at ($(interior.east) + (0.5pt,0)$);
					\begin{scope}
						\clip (interior.north west) rectangle ([xshift=3ex]interior.east);
						\filldraw [white, blur shadow={shadow opacity=60, shadow yshift=-.75ex}, rounded corners=2pt] (interior.north west) rectangle (interior.south east);
					\end{scope}
					\begin{scope}[gray!80!black]
						\fill (dotA) circle (2pt);
						\fill (dotB) circle (2pt);
					\end{scope}
				},
		},
	#1,
}

%%%%%%%%%%%%%%%%%%%%%%%%%%%%%%%%%%%%%%%%%%%%%%%%%%%%%%%%%%%%%%%%%
% SELF MADE COMMANDS
%%%%%%%%%%%%%%%%%%%%%%%%%%%%%%
\newcommand{\sol}{\setlength{\parindent}{0cm}\textbf{\textit{Løsning:}}\setlength{\parindent}{1cm}}
%%%%%%%%%%%%%%%%%%%%%%%%%%%%%%%%%
\usepackage[tmargin=2cm,rmargin=1in,lmargin=1in,margin=0.85in,bmargin=2cm,footskip=.2in]{geometry}\pagestyle{fancy}
\lhead{Minrui Kevin Zhou 2.b}
\rhead{H8: Skråplan}

\title{H8: Skråplan\\
{\Large \textbf{2.b fys B}}}
\author{Kevin Zhou}
\date{December 2023}
\begin{document}
\maketitle
\begin{question}{Bremsesvigt}{}
  En lastbil kører med farten $79 \;\unit{km/h} $ på en $6,2 \;\unit{km} $ lang strækning.
  \begin{itemize}
    \item[a.] Hvor lang tid tager det for lastbilen at køre denne strækning?
  \end{itemize}
  På toppen af en $400 \;\unit{m} $ høj bakke har lastbilen farten $10 \;\unit{ km/h}$. Da lastbilen er kommet ned i højden $200 \;\unit{m}$, er farten steget til $110 \;\unit{km/h}$. Lastbilen har massen 64 ton.
  \begin{itemize}
    \item[b.] Beregn lastbilens tab i mekanisk energi fra toppen af bakken til højden $200 \;\unit{m}$.
  \end{itemize}
  Pludseligt svigter lastbilens bremser, og chaufføren kører derfor ind på et opadgående nødspor som vist på figuren. Ved indkørslen på nødsporet er lastbilens fart $110 \;\unit{km/h}$. 
  På vej op ad nødsporet er lastbilen påvirket af en gnidningskraft, der har størrelsen $120 \;\unit{kN} $.
  \begin{itemize}
    \item[c.] Hvor langt kører lastbilen op ad nødsporet, inden den holder stille?
  \end{itemize}
\end{question}
\sol \\ 
\textbf{a.} Vi finder tiden ved at bruge definitionen af fart.
\[
t=\frac{s}{\text{fart}}=\frac{6,2 \;\unit{km} }{79 \;\unit{km/h} }\approx 7,8\cdot 10^{-2}\;\unit{h}\approx 2,8\cdot 10^2 \;\unit{s} 
\] 
Det tager altså lastbilen $7,8\cdot 10^{-2}\;\unit{h}$ at køre strækningen. \\[1ex]
\textbf{b.} Vi finder forskellen på den mekaniske energi på toppen af bakken og ved $200 \;\unit{m} $.
\begin{equation*}
\begin{split}
  E^{\text{før} }_{mek}-E^{\text{efter} }_{mek}&=m\cdot g \cdot \left(h_{\text{før} }- h_{\text{efter} }\right)  + \frac{m}{2} \cdot \left(v_{\text{før} }^2-v_{\text{efter}}^2 \right) \\ 
  &=64000 \;\unit{kg} \cdot 9,8 \;\unit{m/s^2} \cdot (400-200)\;\unit{m} + 32000 \;\unit{kg}\cdot \left(\left(\frac{10}{3,6} \;\unit{m/s} \right)^2-\left(\frac{110}{3,6} \;\unit{m/s} \right)^2\right) \\ 
  &\approx 96 \;\unit{MJ} 
\end{split}
\end{equation*}
Altså er lastbilens tab i mekanisk energi $96 \;\unit{MJ} $. \\[1ex]
\textbf{c.} Vi antager, at lastbilens acceleration er konstant.
Vi finder først accelerationen af lastbilen ned ad nødsporet mens bilen kører opad med Newtons 2. lov.
\begin{equation*}
\begin{split}
  a&= \frac{F_{res}}{m} \\ 
  &=\frac{64000 \;\unit{kg} \cdot g \cdot \sin(11\degree)+120 \;\unit{kN} }{64000 \;\unit{kg} } \\ 
  &\approx 3,7449 \;\unit{m/s^2} 
\end{split}
\end{equation*}
Vi finder nu forskellen på den kørte strækning fra indkørslen på nødsporet og når $v=0$, hvor $a$ er accelerationen ned ad nødsporet regnet ovenfor. 
\begin{equation*}
\begin{split}
  \Delta s &= \frac{\left(\frac{110}{3.6}\;\unit{m/s}\right)^2  }{2\cdot a} \\ 
  &\approx 1,2 \cdot 10^2 \;\unit{m} = 0,12 \;\unit{km}  
\end{split}
\end{equation*}
Altså kører lastbilen $0,12 \;\unit{km} $ op ad nødsporet, inden den holder stille. 
\begin{question}{Container}{}
  Billedet viser en container, der er ved at blive trukket op ad ladet på en lastvogn. Ved hjælp af en stålwire trækker en elmotor containeren med den konstante fart $0,25 \;\unit{m/s} $.
  Trækkraften $F_s$ fra stålwiren har størrelsen $78,4 \;\unit{kN} $.  
  \begin{itemize}
    \item[a.] Beregn den effekt, hvormed trækkraften fra stålwiren udfører arbejde på containeren.
  \end{itemize}
Ladet hælder $28\degree$ med vandret. Massen af containeren er $9,2$ ton. 
\begin{itemize}
  \item[c.] Beregn gnidningskoefficienten mellem lad og container under bevægelsen.
\end{itemize}
\end{question}
\sol \\ 
\textbf{a.} Siden kræften $F_s$ og vejen er ensrettede, så må $F_s$ udføre følgende arbejde på containeren når den flyttes en vejlængde på $0,25 \;\unit{m} $. 
\begin{equation*}
\begin{split}
  A&=F\cdot s \cdot \cos(\theta) \\ 
  &=78,4 \;\unit{kN} \cdot 0,25 \;\unit{m} \\ 
  &= 19,6 \;\unit{kJ} 
\end{split}
\end{equation*}
Siden containeren netop trækkes op med $0,25 \;\unit{m/s} $, så udføres ovenstående arbejde over $1 \;\unit{s} $.
Altså må effekten, hvormed trækkraften fra stålwiren udfører arbejde på containeren være
\[
P=\frac{A}{t}= \frac{19,6 \;\unit{kJ}}{1 \;\unit{s} }\approx20 \;\unit{kW} 
\] 
Altså er effekten, hvormed trækkraften fra stålwiren udfører arbejde på containeren $20 \;\unit{kW} $. \\[1ex]
\textbf{c.} Siden accelerationen af containeren er $0$, så må summen af gnidningskraftens størrelse og tyngdekraftens komposant med retning parallelt med ladets størrelse være lig med trækkraftens størrelse.
Altså vil det sige, at
\begin{equation*}
\begin{split}
  F_{\mu} + F_{t,\parallel}=78,4 \;\unit{kN}&\implies F_t \cdot \sin(28\degree) + \mu \cdot F_t \cdot \cos(28 \degree )=78,4 \;\unit{kN} \\ 
  &\implies \mu=\frac{\frac{78,4 \;\unit{kN} }{F_t}-\sin(28 \degree )}{\cos(28 \degree )} \\ 
  &\implies \mu=\frac{\frac{78,4 \;\unit{kN} }{9,2\cdot 10^3 \;\unit{kg} \cdot 9,8 \;\unit{m/s^2} }-\sin(28 \degree )}{\cos(28 \degree )} \approx 0,45
\end{split}
\end{equation*}
Altså er gnidningskoefficienten mellem lad og container under bevægelsen $0,45$.
\end{document}
