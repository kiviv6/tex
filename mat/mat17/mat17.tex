\documentclass{article}
\usepackage{graphicx, tikz-cd} % Required for inserting images
\usepackage{amsmath, amssymb, amsthm, amsfonts, siunitx, physics}
\AtBeginDocument{\RenewCommandCopy\qty\SI}
\usepackage[version=4]{mhchem}
\usepackage[most,many,breakable]{tcolorbox}
\usepackage{xcolor, fancyhdr, varwidth}
\usepackage[Glenn]{fncychap}
%Options: Sonny, Lenny, Glenn, Conny, Rejne, Bjarne, Bjornstrup
\usepackage{hyperref, cleveref}
\usepackage{icomma, enumitem} %comma as decimal and continue enumerate with [resume]
%%%%%%%%%%%%%%%%%%%%%%%%%%%%%%
% SELF MADE COLORS
%%%%%%%%%%%%%%%%%%%%%%%%%%%%%%
\definecolor{myg}{RGB}{56, 140, 70}
\definecolor{myb}{RGB}{45, 111, 177}
\definecolor{myr}{RGB}{199, 68, 64}
\definecolor{mytheorembg}{HTML}{F2F2F9}
\definecolor{mytheoremfr}{HTML}{00007B}
\definecolor{mylenmabg}{HTML}{FFFAF8}
\definecolor{mylenmafr}{HTML}{983b0f}
\definecolor{mypropbg}{HTML}{f2fbfc}
\definecolor{mypropfr}{HTML}{191971}
\definecolor{myexamplebg}{HTML}{F2FBF8}
\definecolor{myexamplefr}{HTML}{88D6D1}
\definecolor{myexampleti}{HTML}{2A7F7F}
\definecolor{mydefinitbg}{HTML}{E5E5FF}
\definecolor{mydefinitfr}{HTML}{3F3FA3}
\definecolor{notesgreen}{RGB}{0,162,0}
\definecolor{myp}{RGB}{197, 92, 212}
\definecolor{mygr}{HTML}{2C3338}
\definecolor{myred}{RGB}{127,0,0}
\definecolor{myyellow}{RGB}{169,121,69}
\definecolor{myexercisebg}{HTML}{F2FBF8}
\definecolor{myexercisefg}{HTML}{88D6D1}
%%%%%%%%%%%%%%%%%%%%%%%%%%%%%%%%%%%%%%%%%%%%%%%%%%%%%%%%%%%%%%%%%%%%%%
% Box environments for theorems and problems
%%%%%%%%%%%%%%%%%%%%%%%%%%%%%%%%%%%%%%%%%%%%%%%%%%%%%%%%%%%%%%%%%%%%%
\setlength{\parindent}{1cm}
%================================
% Question BOX
%================================
\makeatletter
\newtcbtheorem{question}{Opgave}{enhanced,
	breakable,
	colback=white,
	colframe=myb!80!black,
	attach boxed title to top left={yshift*=-\tcboxedtitleheight},
	fonttitle=\bfseries,
	title={#2},
	boxed title size=title,
	boxed title style={%
			sharp corners,
			rounded corners=northwest,
			colback=tcbcolframe,
			boxrule=0pt,
		},
	underlay boxed title={%
			\path[fill=tcbcolframe] (title.south west)--(title.south east)
			to[out=0, in=180] ([xshift=5mm]title.east)--
			(title.center-|frame.east)
			[rounded corners=\kvtcb@arc] |-
			(frame.north) -| cycle;
		},
	#1
}{def}
\makeatother
%================================
% DEFINITION BOX
%================================

\newtcbtheorem[]{Definition}{Definition}{enhanced,
	before skip=2mm,after skip=2mm, colback=red!5,colframe=red!80!black,boxrule=0.5mm,
	attach boxed title to top left={xshift=1cm,yshift*=1mm-\tcboxedtitleheight}, varwidth boxed title*=-3cm,
	boxed title style={frame code={
					\path[fill=tcbcolback]
					([yshift=-1mm,xshift=-1mm]frame.north west)
					arc[start angle=0,end angle=180,radius=1mm]
					([yshift=-1mm,xshift=1mm]frame.north east)
					arc[start angle=180,end angle=0,radius=1mm];
					\path[left color=tcbcolback!60!black,right color=tcbcolback!60!black,
						middle color=tcbcolback!80!black]
					([xshift=-2mm]frame.north west) -- ([xshift=2mm]frame.north east)
					[rounded corners=1mm]-- ([xshift=1mm,yshift=-1mm]frame.north east)
					-- (frame.south east) -- (frame.south west)
					-- ([xshift=-1mm,yshift=-1mm]frame.north west)
					[sharp corners]-- cycle;
				},interior engine=empty,
		},
	fonttitle=\bfseries,
	title={#2},#1}{def}
\newtcbtheorem[]{definition}{Definition}{enhanced,
	before skip=2mm,after skip=2mm, colback=red!5,colframe=red!80!black,boxrule=0.5mm,
	attach boxed title to top left={xshift=1cm,yshift*=1mm-\tcboxedtitleheight}, varwidth boxed title*=-3cm,
	boxed title style={frame code={
					\path[fill=tcbcolback]
					([yshift=-1mm,xshift=-1mm]frame.north west)
					arc[start angle=0,end angle=180,radius=1mm]
					([yshift=-1mm,xshift=1mm]frame.north east)
					arc[start angle=180,end angle=0,radius=1mm];
					\path[left color=tcbcolback!60!black,right color=tcbcolback!60!black,
						middle color=tcbcolback!80!black]
					([xshift=-2mm]frame.north west) -- ([xshift=2mm]frame.north east)
					[rounded corners=1mm]-- ([xshift=1mm,yshift=-1mm]frame.north east)
					-- (frame.south east) -- (frame.south west)
					-- ([xshift=-1mm,yshift=-1mm]frame.north west)
					[sharp corners]-- cycle;
				},interior engine=empty,
		},
	fonttitle=\bfseries,
	title={#2},#1}{def}

\newtcbtheorem{theo}%
    {Theorem}{}{theorem}
\newtcolorbox{prob}[1]{colback=red!5!white,colframe=red!50!black,fonttitle=\bfseries,title={#1}}
%================================
% NOTE BOX
%================================

\usetikzlibrary{arrows,calc,shadows.blur}
\tcbuselibrary{skins}
\newtcolorbox{note}[1][]{%
	enhanced jigsaw,
	colback=gray!20!white,%
	colframe=gray!80!black,
	size=small,
	boxrule=1pt,
	title=\textbf{Note:},
	halign title=flush center,
	coltitle=black,
	breakable,
	drop shadow=black!50!white,
	attach boxed title to top left={xshift=1cm,yshift=-\tcboxedtitleheight/2,yshifttext=-\tcboxedtitleheight/2},
	minipage boxed title=1.5cm,
	boxed title style={%
			colback=white,
			size=fbox,
			boxrule=1pt,
			boxsep=2pt,
			underlay={%
					\coordinate (dotA) at ($(interior.west) + (-0.5pt,0)$);
					\coordinate (dotB) at ($(interior.east) + (0.5pt,0)$);
					\begin{scope}
						\clip (interior.north west) rectangle ([xshift=3ex]interior.east);
						\filldraw [white, blur shadow={shadow opacity=60, shadow yshift=-.75ex}, rounded corners=2pt] (interior.north west) rectangle (interior.south east);
					\end{scope}
					\begin{scope}[gray!80!black]
						\fill (dotA) circle (2pt);
						\fill (dotB) circle (2pt);
					\end{scope}
				},
		},
	#1,
}

%%%%%%%%%%%%%%%%%%%%%%%%%%%%%%%%%%%%%%%%%%%%%%%%%%%%%%%%%%%%%%%%%
% SELF MADE COMMANDS
%%%%%%%%%%%%%%%%%%%%%%%%%%%%%%
\newcommand{\sol}{\setlength{\parindent}{0cm}\textbf{\textit{Løsning:}}\setlength{\parindent}{1cm}}
%%%%%%%%%%%%%%%%%%%%%%%%%%%%%%%%%
\usepackage[tmargin=2cm,rmargin=1in,lmargin=1in,margin=0.85in,bmargin=2cm,footskip=.2in]{geometry}\pagestyle{fancy}
\lhead{Minrui Kevin Zhou 2.b}
\rhead{Matematikaflevering 17}

\title{Aflevering 17\\
{\Large \textbf{2.b mat A}}}
\author{Kevin Zhou}
\date{September 2023}

\begin{document}
\maketitle
\section*{Bedømmelseskriterier:}
\begin{itemize}
    \setlength\itemsep{3cm}
    \Large
    \item  Redegørelse og dokumentation for metode
    \item Figurer, grafer og andre illustrationer
    \item Notation og layout
    \item Formidling og forklaring
\end{itemize}
\pagebreak
\begin{question}{}{}
$f:\mathbb{R} \to \mathbb{R}$ er givet ved
\[
  f(x)=(x+3) \cdot e^x
\] 
Bestem $\dv{f}{x}$   
\end{question}
\sol  
\begin{equation*}
\begin{split}
  \dv{f}{x}&=e^x\cdot \dv{x} (x+3) + e^x \cdot (x+3) \\ 
  &= e^x+(x+3)\cdot e^x \\ 
  &= (x+4)\cdot e^x
\end{split}
\end{equation*}

\begin{question}{}{}
Funktionen $f$ er givet ved
\[
  f(x)=e^{7x} \cdot \ln (x)
\] 
Bestem $\dv{f}{x}$   
\end{question}
\sol 
\begin{equation*}
\begin{split}
  \dv{f}{x} &= \ln (x) \dv{x}(e^{7x}) + e^{7x}\cdot \dv{x} (\ln (x))\\ 
  &= \ln (x)\cdot 7e^{7x} + \frac{e^{7x}}{x}\\ 
  &= (7\ln(x)+\frac{1}{x}) \cdot e^{7x}
\end{split}
\end{equation*}

\begin{question}{}{}
Funktionen $f$ er givet ved 
\[
  f(x)=\sqrt{x^2+2} 
\] 
Bestem $\dv{f}{x}$   
\end{question}
\sol 
\begin{equation*}
\begin{split}
  \dv{f}{x}&= \frac{(x^2+2)^{-\frac{1}{2}}}{2} \cdot 2x \\ 
  &= \frac{x}{\sqrt{x^2+2} }
\end{split}
\end{equation*}

\begin{question}{}{}
 Funktionen $f$ er givet ved 
\[
  f(x)=\ln(\frac{1}{x})
\] 
Bestem $\dv{f}{x}$   
\end{question}
\sol 
\begin{equation*}
\begin{split}
  \dv{f}{x}&=\frac{1}{\frac{1}{x}}\cdot - \frac{1}{x^2} \\
  &=-\frac{1}{x} 
\end{split}
\end{equation*}

\begin{question}{}{}
$f:\mathbb{R} \to \mathbb{R}$ er givet ved 
\[
f(x)= 2x^2-3x-1
\] 
  Bestem ligningen for tangenten til grafen for $f$ i punktet $(1,f(1))$. 
\end{question}
\sol \\ 
Den afledede funktion for $f$ findes.
\[
f'(x)=4x-3
\] 
Følgende må gælde for tangenten til grafen for $f$ i punktet $(1,f(1))$. 
\begin{equation*}
\begin{split}
  y&=f'(1)\cdot(x-1)+f(1)\\ 
  &= x-1 + (-2)\\ 
  &= x-3
\end{split}
\end{equation*}
Ligningen for tangenten til grafen for $f$ i punktet $(1,f(1))$ er altså $y=x-3$.

\begin{question}{}{}
  Grafen for $f(x)=-2x^2-x-4$ har netop én tangent med hældningen 7. 
  Bestem ligningen for denne tangent.
\end{question}
\sol \\ 
Siden den geometriske betydning af $\dv{f}{x}$ netop er tangentens hældning, kan vi finde $x$-værdien til tangentens røringspunkt på følgende måde.
\begin{equation*}
\begin{split}
  \dv{f}{x}=7 &\implies -4x-1=7\\ 
  &\iff x=-2
\end{split}
\end{equation*}
Følgende må da gælde for tangenten.
\begin{equation*}
\begin{split}
  y&=7\cdot x + f(-2)-7\cdot(-2)\\ 
  &=7x + (-10) +14\\ 
  &= 7x + 4
\end{split}
\end{equation*}
Ligningen for tangenten med hældningen 7 er altså $y=7x+4$.
\end{document}
