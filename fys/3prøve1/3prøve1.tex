\documentclass{report}
\usepackage{graphicx, tikz-cd, float, titlepic, booktabs} % Required for inserting images
\usepackage{pgfplots}
\pgfplotsset{compat=1.15}
\usepackage{mathrsfs}
\usetikzlibrary{arrows}
\usepackage{amsmath, amssymb, amsthm, amsfonts, siunitx, physics, gensymb}
\AtBeginDocument{\RenewCommandCopy\qty\SI}
\usepackage[version=4]{mhchem}
\usepackage[most,many,breakable]{tcolorbox}
\usepackage{xcolor, fancyhdr, varwidth}
\usepackage[Glenn]{fncychap}
%Options: Sonny, Lenny, Glenn, Conny, Rejne, Bjarne, Bjornstrup
\usepackage{hyperref, cleveref}
\usepackage{icomma, enumitem} %comma as decimal and continue enumerate with [resume]
\usepackage{plimsoll} %use standard state symbol with \stst
\usepackage[danish]{babel}
%%%%%%%%%%%%%%%%%%%%%%%%%%%%%%
% SELF MADE COLORS
%%%%%%%%%%%%%%%%%%%%%%%%%%%%%%
\definecolor{myg}{RGB}{56, 140, 70}
\definecolor{myb}{RGB}{45, 111, 177}
\definecolor{myr}{RGB}{199, 68, 64}
\definecolor{mytheorembg}{HTML}{F2F2F9}
\definecolor{mytheoremfr}{HTML}{00007B}
\definecolor{mylenmabg}{HTML}{FFFAF8}
\definecolor{mylenmafr}{HTML}{983b0f}
\definecolor{mypropbg}{HTML}{f2fbfc}
\definecolor{mypropfr}{HTML}{191971}
\definecolor{myexamplebg}{HTML}{F2FBF8}
\definecolor{myexamplefr}{HTML}{88D6D1}
\definecolor{myexampleti}{HTML}{2A7F7F}
\definecolor{mydefinitbg}{HTML}{E5E5FF}
\definecolor{mydefinitfr}{HTML}{3F3FA3}
\definecolor{notesgreen}{RGB}{0,162,0}
\definecolor{myp}{RGB}{197, 92, 212}
\definecolor{mygr}{HTML}{2C3338}
\definecolor{myred}{RGB}{127,0,0}
\definecolor{myyellow}{RGB}{169,121,69}
\definecolor{myexercisebg}{HTML}{F2FBF8}
\definecolor{myexercisefg}{HTML}{88D6D1}
%%%%%%%%%%%%%%%%%%%%%%%%%%%%%%%%%%%%%%%%%%%%%%%%%%%%%%%%%%%%%%%%%%%%%%
% Box environments for theorems and problems
%%%%%%%%%%%%%%%%%%%%%%%%%%%%%%%%%%%%%%%%%%%%%%%%%%%%%%%%%%%%%%%%%%%%%
\setlength{\parindent}{1cm}
%================================
% Question BOX
%================================
\makeatletter
\newtcbtheorem{question}{Opgave}{enhanced,
	breakable,
	colback=white,
	colframe=myb!80!black,
	attach boxed title to top left={yshift*=-\tcboxedtitleheight},
	fonttitle=\bfseries,
	title={#2},
	boxed title size=title,
	boxed title style={%
			sharp corners,
			rounded corners=northwest,
			colback=tcbcolframe,
			boxrule=0pt,
		},
	underlay boxed title={%
			\path[fill=tcbcolframe] (title.south west)--(title.south east)
			to[out=0, in=180] ([xshift=5mm]title.east)--
			(title.center-|frame.east)
			[rounded corners=\kvtcb@arc] |-
			(frame.north) -| cycle;
		},
	#1
}{def}
\makeatother
%================================
% DEFINITION BOX
%================================

\newtcbtheorem[]{Definition}{Definition}{enhanced,
	before skip=2mm,after skip=2mm, colback=red!5,colframe=red!80!black,boxrule=0.5mm,
	attach boxed title to top left={xshift=1cm,yshift*=1mm-\tcboxedtitleheight}, varwidth boxed title*=-3cm,
	boxed title style={frame code={
					\path[fill=tcbcolback]
					([yshift=-1mm,xshift=-1mm]frame.north west)
					arc[start angle=0,end angle=180,radius=1mm]
					([yshift=-1mm,xshift=1mm]frame.north east)
					arc[start angle=180,end angle=0,radius=1mm];
					\path[left color=tcbcolback!60!black,right color=tcbcolback!60!black,
						middle color=tcbcolback!80!black]
					([xshift=-2mm]frame.north west) -- ([xshift=2mm]frame.north east)
					[rounded corners=1mm]-- ([xshift=1mm,yshift=-1mm]frame.north east)
					-- (frame.south east) -- (frame.south west)
					-- ([xshift=-1mm,yshift=-1mm]frame.north west)
					[sharp corners]-- cycle;
				},interior engine=empty,
		},
	fonttitle=\bfseries,
	title={#2},#1}{def}
\newtcbtheorem[]{definition}{Definition}{enhanced,
	before skip=2mm,after skip=2mm, colback=red!5,colframe=red!80!black,boxrule=0.5mm,
	attach boxed title to top left={xshift=1cm,yshift*=1mm-\tcboxedtitleheight}, varwidth boxed title*=-3cm,
	boxed title style={frame code={
					\path[fill=tcbcolback]
					([yshift=-1mm,xshift=-1mm]frame.north west)
					arc[start angle=0,end angle=180,radius=1mm]
					([yshift=-1mm,xshift=1mm]frame.north east)
					arc[start angle=180,end angle=0,radius=1mm];
					\path[left color=tcbcolback!60!black,right color=tcbcolback!60!black,
						middle color=tcbcolback!80!black]
					([xshift=-2mm]frame.north west) -- ([xshift=2mm]frame.north east)
					[rounded corners=1mm]-- ([xshift=1mm,yshift=-1mm]frame.north east)
					-- (frame.south east) -- (frame.south west)
					-- ([xshift=-1mm,yshift=-1mm]frame.north west)
					[sharp corners]-- cycle;
				},interior engine=empty,
		},
	fonttitle=\bfseries,
	title={#2},#1}{def}

\newtcbtheorem{theo}%
    {Theorem}{}{theorem}
\newtcolorbox{prob}[1]{colback=red!5!white,colframe=red!50!black,fonttitle=\bfseries,title={#1}}
%================================
% NOTE BOX
%================================

\usetikzlibrary{arrows,calc,shadows.blur}
\tcbuselibrary{skins}
\newtcolorbox{note}[1][]{%
	enhanced jigsaw,
	colback=gray!20!white,%
	colframe=gray!80!black,
	size=small,
	boxrule=1pt,
	title=\textbf{Note:},
	halign title=flush center,
	coltitle=black,
	breakable,
	drop shadow=black!50!white,
	attach boxed title to top left={xshift=1cm,yshift=-\tcboxedtitleheight/2,yshifttext=-\tcboxedtitleheight/2},
	minipage boxed title=1.5cm,
	boxed title style={%
			colback=white,
			size=fbox,
			boxrule=1pt,
			boxsep=2pt,
			underlay={%
					\coordinate (dotA) at ($(interior.west) + (-0.5pt,0)$);
					\coordinate (dotB) at ($(interior.east) + (0.5pt,0)$);
					\begin{scope}
						\clip (interior.north west) rectangle ([xshift=3ex]interior.east);
						\filldraw [white, blur shadow={shadow opacity=60, shadow yshift=-.75ex}, rounded corners=2pt] (interior.north west) rectangle (interior.south east);
					\end{scope}
					\begin{scope}[gray!80!black]
						\fill (dotA) circle (2pt);
						\fill (dotB) circle (2pt);
					\end{scope}
				},
		},
	#1,
}
%================================
% EXAMPLE BOX
%================================
\newtcbtheorem[number within=section]{Example}{Example}
{%
	colback = myexamplebg
	,breakable
	,colframe = myexamplefr
	,coltitle = myexampleti
	,boxrule = 1pt
	,sharp corners
	,detach title
	,before upper=\tcbtitle\par\smallskip
	,fonttitle = \bfseries
	,description font = \mdseries
	,separator sign none
	,description delimiters parenthesis
}
{ex}
%================================
% THEOREM BOX
%================================

\tcbuselibrary{theorems,skins,hooks}
\newtcbtheorem[number within=section]{Theorem}{Theorem}
{%
	enhanced,
	breakable,
	colback = mytheorembg,
	frame hidden,
	boxrule = 0sp,
	borderline west = {2pt}{0pt}{mytheoremfr},
	sharp corners,
	detach title,
	before upper = \tcbtitle\par\smallskip,
	coltitle = mytheoremfr,
	fonttitle = \bfseries\sffamily,
	description font = \mdseries,
	separator sign none,
	segmentation style={solid, mytheoremfr},
}
{th}

%%%%%%%%%%%%%%%%%%%%%%%%%%%%%%%%%%%%%%%%%%%%%%%%%%%%%%%%%%%%%%%%%
% SELF MADE COMMANDS
%%%%%%%%%%%%%%%%%%%%%%%%%%%%%%
\newcommand{\sol}{\setlength{\parindent}{0cm}\textbf{\textit{Løsning:}}\setlength{\parindent}{1cm}}
%%%%%%%%%%%%%%%%%%%%%%%%%%%%%%%%%
\usepackage[tmargin=2cm,rmargin=1in,lmargin=1in,margin=0.85in,bmargin=2cm,footskip=.2in]{geometry}\pagestyle{fancy}
\lhead{Minrui Kevin Zhou 3.b}
\rhead{Prøve}

\title{Prøve\\
{\Large \textbf{3.b fys A}}}
\author{Kevin Zhou}
\date{\today}

\begin{document}
\maketitle
\section*{Tårnspring}
\sol \\
\textbf{a.}
Der er tale om "skråt kast", hvor vinklen $\theta=0$.
Hvis vi ser bort fra luftmodstand og vi betegner enden af platformen, hvor drengen hopper ud med $(0,0)$, så har vi fra uafhængighedsprincippet, at 
\[
y=-\frac{1}{2}\cdot g \cdot t^2 \iff t=\sqrt{\frac{-2y}{g}} 
\] 
da vi lader tiden $t$ være en positiv størrelse. 
Vi beregner nu tiden, når drengen rammer vandet eller bassinkanten.
\begin{equation*}
\begin{split}
  t&=\sqrt{\frac{2y}{g}} \\
  &=\sqrt{\frac{-2 \cdot \left(-10 \;\unit{m} \right) }{9,8 \;\unit{m/s^2} }} \\
  &=\sqrt{\frac{20}{9,8}}  \;\unit{s} 
\end{split}
\end{equation*}
Fra uafhængighedsprincippet har vi også, at 
\begin{equation*}
\begin{split}
  x&=v_0 \cdot t\\
  &=5,0 \;\unit{m/s} \cdot \sqrt{\frac{20}{9,8}}  \;\unit{s} \\
  &\approx 7,1 \;\unit{m} 
\end{split}
\end{equation*}
Altså skal $d$ mindst være $7,1 \;\unit{m} $, for at han ikke rammer den modsatte bassinkant. \\[1ex]
\textbf{b.}
Fra Pythagoras sætning har vi, at 
\begin{equation*}
\begin{split}
  v&=\sqrt{v_x^2 + v_y^2} \\
  &=\sqrt{v_0^2+\left(-g \cdot t\right)^2 } \\
  &=\sqrt{\left(5,0 \;\unit{m/s} \right)^2 + \left(-9,8 \;\unit{m/s} \cdot \sqrt{\frac{20}{9,8}}\;\unit{s} \right) ^2} \\
  &\approx 15 \;\unit{m/s} 
\end{split}
\end{equation*}
Altså rammer drengen vandoverfladen med farten $15 \;\unit{m/s} $.

\section*{Vanddråber i olie}
\sol \\
\textbf{a.}
For kuglerne gælder der, at
\begin{equation*}
\begin{split}
  m=\rho \cdot 100 V &\iff 100 \cdot \frac{4}{3} \pi \cdot r^3=\frac{m}{\rho}\\
  &\iff r= \sqrt[3]{\frac{3 \cdot m}{400 \cdot \pi \cdot \rho}} 
\end{split}
\end{equation*}
Vi kan da regne radius ud.
\begin{equation*}
\begin{split}
  r&= \sqrt[3]{\frac{3 \cdot m}{400 \cdot \pi \cdot \rho}} \\
  &= \sqrt[3]{\frac{3 \cdot 1,36 \;\unit{g}  }{400 \cdot \pi \cdot 1,0 \cdot 10^3 \;\unit{g/cm^3}}} \\
  &\approx 0,148 \;\unit{cm} 
\end{split}
\end{equation*}
hvilket var, hvad vi skulle vise.\\[1ex]
\textbf{c.}
Når vanddråben når den konstante fart, må den resulterende kraft være 0:
\begin{equation*}
\begin{split}
  F_t-F_\mu-F_{\text{op} }=0 \iff F_\mu=F_t-F_{\text{ op}}
\end{split}
\end{equation*}

\section*{Skilift}
\sol \\
\textbf{a.}
Dråben er påvirket af $\overrightarrow{\textbf{F}} $ og $\overrightarrow{\textbf{F}_t} $, der tilsammen må give den resulterende kraft, $\overrightarrow{\textbf{F}_c} $, der skal være rettet ind mod centrum. 
Ved at betragte en retvinklet trekant, ser vi, at 
\[
|\overrightarrow{\textbf{F}_c} | = |\overrightarrow{\textbf{F}_t}|  \cdot \tan\left(25 \degree \right) 
\] 
Da der er tale om en jævn cirkelbevægelse, så må der gælde, at 
\begin{equation*}
\begin{split}
  a=\frac{v^2}{r} &\iff v = \sqrt{a \cdot r} \\
  &\iff v=\sqrt{\frac{|\overrightarrow{\textbf{F}_c}| }{m} \cdot r} \\
  &\iff v=\sqrt{\frac{|\overrightarrow{\textbf{F}_t}| \cdot \tan\left(25 \degree \right)  }{m} \cdot r} \\
  &\iff v=\sqrt{g \cdot \tan\left(25 \degree \right) \cdot r} 
\end{split}
\end{equation*}
Vi kan nu regne dråbens fart ud.
\begin{equation*}
\begin{split}
  v&=\sqrt{g \cdot \tan\left(25 \degree \right) \cdot r} \\
  &=\sqrt{9,8 \;\unit{m/s^2} \cdot \tan\left(25 \degree \right) \cdot 0,95 \;\unit{m}  } \\
  &\approx 2,1 \;\unit{m/s} 
\end{split}
\end{equation*}
Altså er vanddråbens fart i cirkelbevægelsen $2,1 \;\unit{m/s} $.
\end{document}
