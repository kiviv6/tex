\documentclass{report}
\usepackage{graphicx, tikz-cd, float, titlepic, booktabs} % Required for inserting images
\usepackage{amsmath, amssymb, amsthm, amsfonts, siunitx, physics, gensymb}
\AtBeginDocument{\RenewCommandCopy\qty\SI}
\usepackage[version=4]{mhchem}
\usepackage[most,many,breakable]{tcolorbox}
\usepackage{xcolor, fancyhdr, varwidth}
\usepackage[Glenn]{fncychap}
%Options: Sonny, Lenny, Glenn, Conny, Rejne, Bjarne, Bjornstrup
\usepackage{hyperref, cleveref}
\usepackage{icomma, enumitem} %comma as decimal and continue enumerate with [resume]
%%%%%%%%%%%%%%%%%%%%%%%%%%%%%%
% SELF MADE COLORS
%%%%%%%%%%%%%%%%%%%%%%%%%%%%%%
\definecolor{myg}{RGB}{56, 140, 70}
\definecolor{myb}{RGB}{45, 111, 177}
\definecolor{myr}{RGB}{199, 68, 64}
\definecolor{mytheorembg}{HTML}{F2F2F9}
\definecolor{mytheoremfr}{HTML}{00007B}
\definecolor{mylenmabg}{HTML}{FFFAF8}
\definecolor{mylenmafr}{HTML}{983b0f}
\definecolor{mypropbg}{HTML}{f2fbfc}
\definecolor{mypropfr}{HTML}{191971}
\definecolor{myexamplebg}{HTML}{F2FBF8}
\definecolor{myexamplefr}{HTML}{88D6D1}
\definecolor{myexampleti}{HTML}{2A7F7F}
\definecolor{mydefinitbg}{HTML}{E5E5FF}
\definecolor{mydefinitfr}{HTML}{3F3FA3}
\definecolor{notesgreen}{RGB}{0,162,0}
\definecolor{myp}{RGB}{197, 92, 212}
\definecolor{mygr}{HTML}{2C3338}
\definecolor{myred}{RGB}{127,0,0}
\definecolor{myyellow}{RGB}{169,121,69}
\definecolor{myexercisebg}{HTML}{F2FBF8}
\definecolor{myexercisefg}{HTML}{88D6D1}
%%%%%%%%%%%%%%%%%%%%%%%%%%%%%%%%%%%%%%%%%%%%%%%%%%%%%%%%%%%%%%%%%%%%%%
% Box environments for theorems and problems
%%%%%%%%%%%%%%%%%%%%%%%%%%%%%%%%%%%%%%%%%%%%%%%%%%%%%%%%%%%%%%%%%%%%%
\setlength{\parindent}{1cm}
%================================
% Question BOX
%================================
\makeatletter
\newtcbtheorem{question}{Opgave}{enhanced,
	breakable,
	colback=white,
	colframe=myb!80!black,
	attach boxed title to top left={yshift*=-\tcboxedtitleheight},
	fonttitle=\bfseries,
	title={#2},
	boxed title size=title,
	boxed title style={%
			sharp corners,
			rounded corners=northwest,
			colback=tcbcolframe,
			boxrule=0pt,
		},
	underlay boxed title={%
			\path[fill=tcbcolframe] (title.south west)--(title.south east)
			to[out=0, in=180] ([xshift=5mm]title.east)--
			(title.center-|frame.east)
			[rounded corners=\kvtcb@arc] |-
			(frame.north) -| cycle;
		},
	#1
}{def}
\makeatother
%================================
% DEFINITION BOX
%================================

\newtcbtheorem[]{Definition}{Definition}{enhanced,
	before skip=2mm,after skip=2mm, colback=red!5,colframe=red!80!black,boxrule=0.5mm,
	attach boxed title to top left={xshift=1cm,yshift*=1mm-\tcboxedtitleheight}, varwidth boxed title*=-3cm,
	boxed title style={frame code={
					\path[fill=tcbcolback]
					([yshift=-1mm,xshift=-1mm]frame.north west)
					arc[start angle=0,end angle=180,radius=1mm]
					([yshift=-1mm,xshift=1mm]frame.north east)
					arc[start angle=180,end angle=0,radius=1mm];
					\path[left color=tcbcolback!60!black,right color=tcbcolback!60!black,
						middle color=tcbcolback!80!black]
					([xshift=-2mm]frame.north west) -- ([xshift=2mm]frame.north east)
					[rounded corners=1mm]-- ([xshift=1mm,yshift=-1mm]frame.north east)
					-- (frame.south east) -- (frame.south west)
					-- ([xshift=-1mm,yshift=-1mm]frame.north west)
					[sharp corners]-- cycle;
				},interior engine=empty,
		},
	fonttitle=\bfseries,
	title={#2},#1}{def}
\newtcbtheorem[]{definition}{Definition}{enhanced,
	before skip=2mm,after skip=2mm, colback=red!5,colframe=red!80!black,boxrule=0.5mm,
	attach boxed title to top left={xshift=1cm,yshift*=1mm-\tcboxedtitleheight}, varwidth boxed title*=-3cm,
	boxed title style={frame code={
					\path[fill=tcbcolback]
					([yshift=-1mm,xshift=-1mm]frame.north west)
					arc[start angle=0,end angle=180,radius=1mm]
					([yshift=-1mm,xshift=1mm]frame.north east)
					arc[start angle=180,end angle=0,radius=1mm];
					\path[left color=tcbcolback!60!black,right color=tcbcolback!60!black,
						middle color=tcbcolback!80!black]
					([xshift=-2mm]frame.north west) -- ([xshift=2mm]frame.north east)
					[rounded corners=1mm]-- ([xshift=1mm,yshift=-1mm]frame.north east)
					-- (frame.south east) -- (frame.south west)
					-- ([xshift=-1mm,yshift=-1mm]frame.north west)
					[sharp corners]-- cycle;
				},interior engine=empty,
		},
	fonttitle=\bfseries,
	title={#2},#1}{def}

\newtcbtheorem{theo}%
    {Theorem}{}{theorem}
\newtcolorbox{prob}[1]{colback=red!5!white,colframe=red!50!black,fonttitle=\bfseries,title={#1}}
%================================
% NOTE BOX
%================================

\usetikzlibrary{arrows,calc,shadows.blur}
\tcbuselibrary{skins}
\newtcolorbox{note}[1][]{%
	enhanced jigsaw,
	colback=gray!20!white,%
	colframe=gray!80!black,
	size=small,
	boxrule=1pt,
	title=\textbf{Note:},
	halign title=flush center,
	coltitle=black,
	breakable,
	drop shadow=black!50!white,
	attach boxed title to top left={xshift=1cm,yshift=-\tcboxedtitleheight/2,yshifttext=-\tcboxedtitleheight/2},
	minipage boxed title=1.5cm,
	boxed title style={%
			colback=white,
			size=fbox,
			boxrule=1pt,
			boxsep=2pt,
			underlay={%
					\coordinate (dotA) at ($(interior.west) + (-0.5pt,0)$);
					\coordinate (dotB) at ($(interior.east) + (0.5pt,0)$);
					\begin{scope}
						\clip (interior.north west) rectangle ([xshift=3ex]interior.east);
						\filldraw [white, blur shadow={shadow opacity=60, shadow yshift=-.75ex}, rounded corners=2pt] (interior.north west) rectangle (interior.south east);
					\end{scope}
					\begin{scope}[gray!80!black]
						\fill (dotA) circle (2pt);
						\fill (dotB) circle (2pt);
					\end{scope}
				},
		},
	#1,
}

%%%%%%%%%%%%%%%%%%%%%%%%%%%%%%%%%%%%%%%%%%%%%%%%%%%%%%%%%%%%%%%%%
% SELF MADE COMMANDS
%%%%%%%%%%%%%%%%%%%%%%%%%%%%%%
\newcommand{\sol}{\setlength{\parindent}{0cm}\textbf{\textit{Løsning:}}\setlength{\parindent}{1cm}}
%%%%%%%%%%%%%%%%%%%%%%%%%%%%%%%%%
\usepackage[tmargin=2cm,rmargin=1in,lmargin=1in,margin=0.85in,bmargin=2cm,footskip=.2in]{geometry}\pagestyle{fancy}
\lhead{Minrui Kevin Zhou 2.b}
\rhead{H7: Atomfysik}

\title{H7: Atomfysik\\
{\Large \textbf{2.b fys A}}}
\author{Kevin Zhou}
\date{November 2023}

\begin{document}
\maketitle
\begin{question}{Spektrallinjer}{}
\begin{itemize}
  \item[a.] Forklar, hvilken spektrallinje i emissionsspektret der svarer til overgangen fra energiniveau D til energiniveau B.
\end{itemize}
\begin{enumerate}
  \item Skriv farven (eller IR for infrarød eller UV for ultraviolet, hvis bølgelængden ikke er i det synlige område), og beregn  energien (målt i eV og med 4 BC) svarende til hver af de 3 viste emissionslinjer, der har bølgelængde angivet.
\item Sæt energier (i eV) på de 4 niveauer (sæt for enkelhedens skyld A’s energi til 0) – se på, hvilke overgange de 3 viste emissionslinjer, der har bølgelængder angivet, svarer til?
\item Hvilke mulige overgange er der tilbage?
\item Find energierne for disse overgange ved at se på forskelle mellem de tilhørende niveauer.
\item Find bølgelængderne, svarende til energierne i punktet ovenfor.
\item Hvilke farver svarer de til?
\item Hvilken af dem er den viste i emissionsspektret?
\end{enumerate}
\end{question}
\sol \\ 
\textbf{a.} Det er da spektrallinjen uden nogen angivet bølgelængde, som må svare til overgangen fra energiniveau D til energiniveau B.
Dette er tilfældet, da denne linje har den anden mindste bølgelængde, hvilket må svare til den anden største energi.
Ved at kigge på energiniveaudiagrammet med hensyn til emission, så ses det, at netop overgangen fra energiniveau D til energiniveau B har den næststørste energi, siden overgangen mellem energiniveau A og D ikke forekommer. \\[1ex]
\textbf{1.} Energierne for de tre bølgelængder regnes med formlen $E_{\text{foton}}=\frac{1240 \;\unit{eV}\cdot \unit{nm}}{\lambda}$. 
Energien svarende til emissionslinjen med bølgelængden $323,3 \;\unit{nm} $ er da
\[
E_{\text{foton}}=\frac{1240 \;\unit{eV}\cdot \unit{nm}}{323,3 \;\unit{nm} }
\approx 3,835 \;\unit{eV} 
\] 
Energien svarende til emissionslinjen med bølgelængden $670,8\;\unit{nm} $ er da
\[
E_{\text{foton}}=\frac{1240 \;\unit{eV}\cdot \unit{nm}}{670,8\;\unit{nm} }
\approx 1,849\;\unit{eV} 
\] 
Energien svarende til emissionslinjen med bølgelængden $2450\;\unit{nm} $ er da
\[
E_{\text{foton}}=\frac{1240 \;\unit{eV}\cdot \unit{nm}}{2450\;\unit{nm} }
\approx 0,5061\;\unit{eV} 
\]
I \cref{farve} ses disse resultater samt farverne.
\begin{table}[H]
\centering
\begin{tabular}{@{}lll@{}}
\toprule
$\lambda_{\text{emission}}/\unit{nm}$ & Farve/type & $E_{\text{foton}}/\unit{eV}$ \\ \midrule
323,3                                 & UV-A       & 3,835                        \\
670,8                                 & Rød        & 1,849                        \\
2450                                  & IR         & 0,5061                       \\ \bottomrule
\end{tabular}
\caption{Tabel med energierne til de tre emissionslinjer og tilsvarende farve}
\label{farve}
\end{table}
\noindent \textbf{2.} Siden stor bølgelængde svarer til lille energi, og absorption kun sker fra grundtilstanden, må \cref{energiniveau} og \cref{overgang} gælde.
\begin{table}[H]
\centering
\begin{tabular}{@{}ll@{}}
\toprule
$\lambda_{\text{emission}}/\unit{nm}$ & Overgang         \\ \midrule
323,3                                 & $C\rightarrow A$ \\
670,8                                 & $B\rightarrow A$ \\
2450                                  & $D\rightarrow C$ \\ \bottomrule
\end{tabular}
  \caption{Overgangene mellem energiniveauerne, som bølgelængderne svarer til}
  \label{overgang}
\end{table}
\begin{table}[H]
\centering
\begin{tabular}{@{}ll@{}}
\toprule
Energiniveau & Energi/\unit{eV} \\ \midrule
A            & 0                \\
B            & 1,849            \\
C            & 3,835            \\
D            & 4,342            \\ \bottomrule
\end{tabular}
  \caption{De fire laveste energiniveauer og tilsvarende energier}
  \label{energiniveau}
\end{table}
\noindent \textbf{3.} De mulige overgange tilbage med hensyn til emission er da $D\rightarrow B$ og $C\rightarrow B$.\\[1ex]
\textbf{4.} Ved subtraktion af energierne svarende til de forskellige energiniveauer (værdierne i \cref{energiniveau} er kun med fire betydende cifre) er energien for overgangen $D\rightarrow B$
\[
E_{D\rightarrow B} \approx 2,493 \;\unit{eV} ,
\] 
og energien for overgangen $C\rightarrow B$ er
\[
E _{C\rightarrow B}\approx 1,987 \;\unit{eV}.
\] 
\textbf{5.} Siden 
\[
\lambda = \frac{1240 \;\unit{eV}\cdot \unit{nm} }{E_{\text{foton}}},
\] 
så må følgende gælde.
\[
\lambda_{D\rightarrow B} = \frac{1240 \;\unit{eV}\cdot \unit{nm} }{E_{D\rightarrow B}} \approx 497,4 \;\unit{nm} 
\] 
\[
\lambda_{C\rightarrow B} = \frac{1240 \;\unit{eV}\cdot \unit{nm} }{E_{C\rightarrow B}} \approx 624,1\;\unit{nm} 
\] 
\textbf{6.} Disse svarer til cyan og orange.\footnote{ifølge Wikipedia}\\[1ex]
\textbf{7.} Den viste i emissionsspektret må have bølgelængden $497,4 \;\unit{nm} $. Altså er den cyan.
\begin{question}{Ultraviolet lys på hydrogenatomer}{}
Se OneNote. Jeg er for doven til at skrive ind her.
\end{question}
\sol \\ 
\textbf{a.} Energierne omregnes til \unit{eV}.
\begin{equation*}
\begin{split}
  1,80 \;\unit{aJ}&=\frac{1,80\cdot 10^{-18}\;}{1,602\cdot 10^{-19}} \;\unit{eV} \approx 11,2 \;\unit{eV} \\
  2,00 \;\unit{aJ}&=\frac{2,00\cdot 10^{-18}}{1,602\cdot 10^{-19}} \;\unit{eV} \approx 12,5 \;\unit{eV} 
\end{split}
\end{equation*}
\textbf{b.} Den største bølgelængde vil da være
\[
\lambda_{max}=\frac{1240 \;\unit{eV} \cdot \;\unit{nm} }{\frac{18,0}{1,602}\;\unit{eV} } \approx 110 \;\unit{nm}
\] 
Den mindste bølgelængde vil da være
\[
\lambda_{min}=\frac{1240 \;\unit{eV} \cdot \;\unit{nm} }{\frac{20,0}{1,602}\;\unit{eV} } \approx 99,3\;\unit{nm}
\] 
\textbf{c.} I position 1 vil intensitetsgrafen være for et absorptionsspektrum. Altså kan graf 1,4 og 6 nemt udelukkes. 
Graf 2 passer heller ikke, da absorption af fotoner med en bestemt bølgelængde ikke vil øge intensiteten der, men derimod gøre den mindre.
Ved absorption af en foton i vores interval, er den eneste mulighed, at overgangen er fra energiniveau O til B (se fig. 3). 
Derefter emitteres for det enkelte hydrogenatom enten en foton med energien $1,94 \;\unit{aJ} $ eller en foton med energien $0,3 \;\unit{aJ} $ samt en foton med energien $1,64 \;\unit{aJ} $. 
Derfor må graf 3 bedst repræsentere målingerne i position 1. \\[1ex]
\textbf{d.} I position 2 vil intensitetsgrafen være for et emissionsspektrum. Altså kan graf 2, 3 og 5 udelukkes. 
Af årsagerne også nævnt i \textbf{c.}, må graf 6 da repræsentere målingerne i position 2 bedst. 
\end{document}
