\documentclass{article}
\usepackage{graphicx, tikz-cd, float, titlepic, booktabs} % Required for inserting images
\usepackage{pgfplots}
\pgfplotsset{compat=1.15}
\usepackage{mathrsfs}
\usetikzlibrary{arrows}
\usepackage{amsmath, amssymb, amsthm, amsfonts, siunitx, physics, gensymb}
\AtBeginDocument{\RenewCommandCopy\qty\SI}
\usepackage[version=4]{mhchem}
\usepackage[most,many,breakable]{tcolorbox}
\usepackage{xcolor, fancyhdr, varwidth}
\usepackage[Glenn]{fncychap}
%Options: Sonny, Lenny, Glenn, Conny, Rejne, Bjarne, Bjornstrup
\usepackage{hyperref, cleveref}
\usepackage{icomma, enumitem} %comma as decimal and continue enumerate with [resume]
\usepackage[danish]{babel}
%%%%%%%%%%%%%%%%%%%%%%%%%%%%%%
% SELF MADE COLORS
%%%%%%%%%%%%%%%%%%%%%%%%%%%%%%
\definecolor{myg}{RGB}{56, 140, 70}
\definecolor{myb}{RGB}{45, 111, 177}
\definecolor{myr}{RGB}{199, 68, 64}
\definecolor{mytheorembg}{HTML}{F2F2F9}
\definecolor{mytheoremfr}{HTML}{00007B}
\definecolor{mylenmabg}{HTML}{FFFAF8}
\definecolor{mylenmafr}{HTML}{983b0f}
\definecolor{mypropbg}{HTML}{f2fbfc}
\definecolor{mypropfr}{HTML}{191971}
\definecolor{myexamplebg}{HTML}{F2FBF8}
\definecolor{myexamplefr}{HTML}{88D6D1}
\definecolor{myexampleti}{HTML}{2A7F7F}
\definecolor{mydefinitbg}{HTML}{E5E5FF}
\definecolor{mydefinitfr}{HTML}{3F3FA3}
\definecolor{notesgreen}{RGB}{0,162,0}
\definecolor{myp}{RGB}{197, 92, 212}
\definecolor{mygr}{HTML}{2C3338}
\definecolor{myred}{RGB}{127,0,0}
\definecolor{myyellow}{RGB}{169,121,69}
\definecolor{myexercisebg}{HTML}{F2FBF8}
\definecolor{myexercisefg}{HTML}{88D6D1}
%%%%%%%%%%%%%%%%%%%%%%%%%%%%%%%%%%%%%%%%%%%%%%%%%%%%%%%%%%%%%%%%%%%%%%
% Box environments for theorems and problems
%%%%%%%%%%%%%%%%%%%%%%%%%%%%%%%%%%%%%%%%%%%%%%%%%%%%%%%%%%%%%%%%%%%%%
\setlength{\parindent}{1cm}
%================================
% Question BOX
%================================
\makeatletter
\newtcbtheorem{question}{Opgave}{enhanced,
	breakable,
	colback=white,
	colframe=myb!80!black,
	attach boxed title to top left={yshift*=-\tcboxedtitleheight},
	fonttitle=\bfseries,
	title={#2},
	boxed title size=title,
	boxed title style={%
			sharp corners,
			rounded corners=northwest,
			colback=tcbcolframe,
			boxrule=0pt,
		},
	underlay boxed title={%
			\path[fill=tcbcolframe] (title.south west)--(title.south east)
			to[out=0, in=180] ([xshift=5mm]title.east)--
			(title.center-|frame.east)
			[rounded corners=\kvtcb@arc] |-
			(frame.north) -| cycle;
		},
	#1
}{def}
\makeatother
%================================
% DEFINITION BOX
%================================

\newtcbtheorem[]{Definition}{Definition}{enhanced,
	before skip=2mm,after skip=2mm, colback=red!5,colframe=red!80!black,boxrule=0.5mm,
	attach boxed title to top left={xshift=1cm,yshift*=1mm-\tcboxedtitleheight}, varwidth boxed title*=-3cm,
	boxed title style={frame code={
					\path[fill=tcbcolback]
					([yshift=-1mm,xshift=-1mm]frame.north west)
					arc[start angle=0,end angle=180,radius=1mm]
					([yshift=-1mm,xshift=1mm]frame.north east)
					arc[start angle=180,end angle=0,radius=1mm];
					\path[left color=tcbcolback!60!black,right color=tcbcolback!60!black,
						middle color=tcbcolback!80!black]
					([xshift=-2mm]frame.north west) -- ([xshift=2mm]frame.north east)
					[rounded corners=1mm]-- ([xshift=1mm,yshift=-1mm]frame.north east)
					-- (frame.south east) -- (frame.south west)
					-- ([xshift=-1mm,yshift=-1mm]frame.north west)
					[sharp corners]-- cycle;
				},interior engine=empty,
		},
	fonttitle=\bfseries,
	title={#2},#1}{def}
\newtcbtheorem[]{definition}{Definition}{enhanced,
	before skip=2mm,after skip=2mm, colback=red!5,colframe=red!80!black,boxrule=0.5mm,
	attach boxed title to top left={xshift=1cm,yshift*=1mm-\tcboxedtitleheight}, varwidth boxed title*=-3cm,
	boxed title style={frame code={
					\path[fill=tcbcolback]
					([yshift=-1mm,xshift=-1mm]frame.north west)
					arc[start angle=0,end angle=180,radius=1mm]
					([yshift=-1mm,xshift=1mm]frame.north east)
					arc[start angle=180,end angle=0,radius=1mm];
					\path[left color=tcbcolback!60!black,right color=tcbcolback!60!black,
						middle color=tcbcolback!80!black]
					([xshift=-2mm]frame.north west) -- ([xshift=2mm]frame.north east)
					[rounded corners=1mm]-- ([xshift=1mm,yshift=-1mm]frame.north east)
					-- (frame.south east) -- (frame.south west)
					-- ([xshift=-1mm,yshift=-1mm]frame.north west)
					[sharp corners]-- cycle;
				},interior engine=empty,
		},
	fonttitle=\bfseries,
	title={#2},#1}{def}

\newtcbtheorem{theo}%
    {Theorem}{}{theorem}
\newtcolorbox{prob}[1]{colback=red!5!white,colframe=red!50!black,fonttitle=\bfseries,title={#1}}
%================================
% NOTE BOX
%================================

\usetikzlibrary{arrows,calc,shadows.blur}
\tcbuselibrary{skins}
\newtcolorbox{note}[1][]{%
	enhanced jigsaw,
	colback=gray!20!white,%
	colframe=gray!80!black,
	size=small,
	boxrule=1pt,
	title=\textbf{Note:},
	halign title=flush center,
	coltitle=black,
	breakable,
	drop shadow=black!50!white,
	attach boxed title to top left={xshift=1cm,yshift=-\tcboxedtitleheight/2,yshifttext=-\tcboxedtitleheight/2},
	minipage boxed title=1.5cm,
	boxed title style={%
			colback=white,
			size=fbox,
			boxrule=1pt,
			boxsep=2pt,
			underlay={%
					\coordinate (dotA) at ($(interior.west) + (-0.5pt,0)$);
					\coordinate (dotB) at ($(interior.east) + (0.5pt,0)$);
					\begin{scope}
						\clip (interior.north west) rectangle ([xshift=3ex]interior.east);
						\filldraw [white, blur shadow={shadow opacity=60, shadow yshift=-.75ex}, rounded corners=2pt] (interior.north west) rectangle (interior.south east);
					\end{scope}
					\begin{scope}[gray!80!black]
						\fill (dotA) circle (2pt);
						\fill (dotB) circle (2pt);
					\end{scope}
				},
		},
	#1,
}

%%%%%%%%%%%%%%%%%%%%%%%%%%%%%%%%%%%%%%%%%%%%%%%%%%%%%%%%%%%%%%%%%
% SELF MADE COMMANDS
%%%%%%%%%%%%%%%%%%%%%%%%%%%%%%
\newcommand{\sol}{\setlength{\parindent}{0cm}\textbf{\textit{Løsning:}}\setlength{\parindent}{1cm}}
%%%%%%%%%%%%%%%%%%%%%%%%%%%%%%%%%
\usepackage[tmargin=2cm,rmargin=1in,lmargin=1in,margin=0.85in,bmargin=2cm,footskip=.2in]{geometry}\pagestyle{fancy}
\lhead{Minrui Kevin Zhou 2.b}
\rhead{Aflevering 25}

\title{Aflevering 25\\
{\Large \textbf{2.b mat A}}}
\author{Kevin Zhou}
\date{\today}

\begin{document}
\maketitle
\section*{Bedømmelseskriterier:}
\begin{itemize}
    \setlength\itemsep{3cm}
    \Large
    \item  Redegørelse og dokumentation for metode
    \item Figurer, grafer og andre illustrationer
    \item Notation og layout
    \item Formidling og forklaring
\end{itemize}
\pagebreak
\begin{question}{}{}
  I en model kan dugen på en drage beskrives ved parallelogrammet udspændt af vektorerne
  \[
  \va{a} =\mqty(2\\ 3) \text{ og } \va{b} =\mqty(2\\ -4).
  \] 
  Enheden i modellen er \unit{dm }.  
\begin{itemize}
  \item[a.] Tegn en model af dragen i et koordinatsystem med enheden \unit{dm } på begge akser.
  \item[b.] Benyt modellen til at bestemme arealet af dugen.
\end{itemize}
\end{question}
\sol \\
\textbf{a.}
En model af dragen ses i \cref{fig:drage} med enheden \unit{dm } på begge akser.
\begin{figure}[H]
\begin{center}
\definecolor{zzttqq}{rgb}{0.6,0.2,0.}
\begin{tikzpicture}[line cap=round,line join=round,>=triangle 45,x=1.0cm,y=1.0cm]
\begin{axis}[
x=1.0cm,y=1.0cm,
axis lines=middle,
ymajorgrids=true,
xmajorgrids=true,
xmin=-1.0,
xmax=5.0,
ymin=-4.5,
ymax=3.5,
xtick={-1.0,0.0,...,5.0},
ytick={-4.0,-3.0,...,3.0},]
\clip(-1.,-4.5) rectangle (5.,3.5);
\fill[line width=2.pt,color=zzttqq,fill=zzttqq,fill opacity=0.10000000149011612] (0.,0.) -- (2.,3.) -- (4.,-1.) -- (2.,-4.) -- cycle;
\draw [->,line width=2.pt] (0.,0.) -- (2.,3.);
\draw [->,line width=2.pt] (0.,0.) -- (2.,-4.);
\draw [line width=2.pt,color=zzttqq] (2.,3.)-- (4.,-1.);
\draw [line width=2.pt,color=zzttqq] (4.,-1.)-- (2.,-4.);
\draw (0.5598802395209581,2.51946107784432) node[anchor=north west] {$\vec{a}$};
\draw (0.5763473053892216,-2.1077844311377163) node[anchor=north west] {$\vec{b}$};
\end{axis}
\end{tikzpicture}
\end{center}
  \caption{Model tegnet med GeoGebra og TikZ, hvor enheden på akserne er \unit{dm}}
\label{fig:drage}
\end{figure}
\textbf{b.}
Arealet af parallelogrammet udspændt af de to vektorer er da
\begin{equation*}
\begin{split}
  A&=\left| \det \left(\va{a}, \va{b} \right)  \right| \\ 
  &= \left| \mdet{2 & 2 \\ 3 & -4}  \right| \\ 
  &=14
\end{split}
\end{equation*}
Altså er arealet af dugen $14 \;\unit{dm^2} $.
\begin{question}{}{}
  En ret linje i planen går gennem punktet $P(2,5)$, og en normalvektor til linjen er givet ved $\va{n} =\mqty(7\\ 2) $. 
  \begin{itemize}
    \item[a.] Bestem en ligning for linjen.
    \item[b.] Bestem en parameterfremstilling af linjen.
  \end{itemize}
\end{question}
\sol \\
\textbf{a.}
For en linje gennem $P_0(x_0,y_0)$ med normalvektor $\va{n} =\mqty(a\\ b) $ har den ligningen 
\begin{equation*}
\begin{split}
  a \cdot \left(x-x_0\right) + b \cdot \left(y-y_0\right) =0 &\implies 7 \cdot \left(x-2\right) + 2 \cdot \left(y-5\right) =0 \\ 
  &\iff 7x + 2y -24 = 0
\end{split}
\end{equation*}
\textbf{b.}
Stedvektoren for punktet $P(2,5)$ på linjen er $\mqty(2\\ 5) $. Tværvektoren for normalvektoren til linjen må være en retningsvektor for linjen, og er $\mqty(-2\\ 7) $.
Altså er en parameterfremstilling af linjen
\[
\mqty(x\\ y) =\mqty(2\\ 5) + t \cdot \mqty(-2\\ 7) 
\] 
\begin{question}{}{}
  Lad vektorerne $\va{a} \text{ og } \va{b} $ være givet ved 
  \[
  \va{a} =\mqty(-3\\ 2) \text{ og } \va{b} = \mqty(5\\ 4) 
  \] 
  \begin{itemize}
    \item[a.] Bestem $t \in \mathbb{R}$, således at $\left(\va{a} + t \cdot \va{b} \right) \perp \va{b} $.
    \item[b.] Undersøg, om der findes $t \in \mathbb{R}$, således at $\left(\va{a} + t \cdot \va{b} \right) \parallel \va{b} $.
    \item[c.] Bestem projektionen af $\va{a} \text{ på } \va{b} $.
  \end{itemize}
\end{question}
\sol \\
\textbf{a.}
Skalarproduktet af de to ortogonale vektorer må være 0.
\begin{equation*}
\begin{split}
  \left(\va{a} + t \cdot \va{b} \right) \cdot \va{b} =0 &\implies \mqty(-3+ t \cdot 5\\ 2+t \cdot 4) \cdot \mqty(5\\ 4) =0 \\ 
  &\iff 5 \cdot \left(-3 + 5t\right) + 4 \cdot \left(2+4t\right) =0 \\ 
  &\iff 41t=7 \\ 
  &\iff t=\frac{7}{41}
\end{split}
\end{equation*}
\textbf{b.}
Når to vektorer er parallele, så må determinanten for dem være 0.
\begin{equation*}
\begin{split}
  \left(\va{a} +t \cdot \va{b} \right) \parallel \va{b} &\implies \mdet{-3+5t & 5 \\ 2+4t & 4} =0 \\ 
  &\iff 4 \cdot \left(-3+5t\right) - 5 \cdot \left(2+4t\right) =0 \\ 
  &\iff -24=0
\end{split}
\end{equation*}
Altså vil det sige at
\[
\nexists t \in \mathbb{R} \text{ sådan at }  \left(\va{a} + t \cdot \va{b} \right) \parallel \va{b} 
\] 
\textbf{c.}
Projektionen af $\va{a} $ på $\va{b} $ er da
\begin{equation*}
\begin{split}
  \va{a}_b &=\frac{\va{a} \cdot \va{b} }{\abs{\va{b}}^2} \cdot \va{b}  \\ 
  &= \frac{-3 \cdot 5 + 2 \cdot 4}{5^2+4^2} \cdot \mqty(5\\ 4) \\ 
  &=-\frac{7}{41} \cdot \mqty(5\\ 4) \\ 
  &=\mqty(-\frac{35}{41}\\ -\frac{28}{41}) 
\end{split}
\end{equation*}
\begin{question}{}{}
  To linjer $m \text{ og } l$ i planen er givet ved
  \begin{equation*}
  \begin{split}
    &l:\mqty(x\\ y) =\mqty(7\\ 0) +t \cdot \mqty(3\\ 4) \\ 
    &m: \mqty(x\\ y) =\mqty(3\\ 2) + t \cdot \mqty(2\\ 7) 
  \end{split}
  \end{equation*}
Undersøg, om vinklen mellem linjerne $l$ og $m$ er spids.
\end{question}
\sol \\
Vi finder den lille vinkel mellem de to retningsvektorer i parameterfremstillingerne for linjerne.
\begin{equation*}
\begin{split}
  \theta &= \cos^{-1}\left(\frac{\mqty(3\\ 4) \cdot \mqty(2\\ 7) }{\sqrt{(3^2+4^2)\cdot (2^2+7^2)} }\right) \\ 
  &=\cos^{-1}\left(\frac{34}{5 \sqrt{53} }\right) \\ 
  &\approx 20,92 \degree 
\end{split}
\end{equation*}
Altså er der en spids og en stump vinkel mellem linjerne $l$ og $m  $. 
\begin{question}{}{}
 En linje $l$ er givet ved ligningen
  \[
  2x+3y+1=0
  \] 
  Linjen $m  $ står vinkelret på $l$ og går gennem punktet $P_0(5,-7)$. \\ 
  Bestem en parameterfremstilling for $m  $.
\end{question}
\sol \\
Ligningen for $l$ kan omskrives til 
\[
2x+3y+1=0 \iff 2 \left(x-1\right) +3 \left(y+1\right) =0
\] 
Altså er $\mqty(2\\ 3) $ en normalvektor til linjen $l$ og derfor en retningsvektor for linjen $m$. 
Vi kan da bestemme en parameterfremstilling for $m$, da vi også kender et punkt, der tilhører $m$.
\[
m: \mqty(x\\ y) =\mqty(5\\ -7) +t \cdot \mqty(2\\ 3) 
\] 
\end{document}
