\documentclass{report}
\usepackage{graphicx, tikz-cd, float, titlepic, booktabs} % Required for inserting images
\usepackage{pgfplots}
\usepackage{multicol}
\usepackage{makecell}
\pgfplotsset{compat=1.15}
\usepackage{mathrsfs}
\usetikzlibrary{arrows}
\usepackage{amsmath, amssymb, amsthm, amsfonts, siunitx, physics, gensymb}
\AtBeginDocument{\RenewCommandCopy\qty\SI}
\usepackage[version=4]{mhchem}
\usepackage[most,many,breakable]{tcolorbox}
\usepackage{xcolor, fancyhdr, varwidth}
\usepackage[Glenn]{fncychap}
%Options: Sonny, Lenny, Glenn, Conny, Rejne, Bjarne, Bjornstrup
\usepackage{hyperref, cleveref}
\usepackage{icomma, enumitem} %comma as decimal and continue enumerate with [resume]
\usepackage{plimsoll} %use standard state symbol with \stst
\usepackage[danish]{babel}
\renewcommand{\cellalign}{cl}
\renewcommand{\theadalign}{cl}
\renewcommand\theadfont{\bfseries}
%%%%%%%%%%%%%%%%%%%%%%%%%%%%%%
% SELF MADE COLORS
%%%%%%%%%%%%%%%%%%%%%%%%%%%%%%
\definecolor{myg}{RGB}{56, 140, 70}
\definecolor{myb}{RGB}{45, 111, 177}
\definecolor{myr}{RGB}{199, 68, 64}
\definecolor{mytheorembg}{HTML}{F2F2F9}
\definecolor{mytheoremfr}{HTML}{00007B}
\definecolor{mylenmabg}{HTML}{FFFAF8}
\definecolor{mylenmafr}{HTML}{983b0f}
\definecolor{mypropbg}{HTML}{f2fbfc}
\definecolor{mypropfr}{HTML}{191971}
\definecolor{myexamplebg}{HTML}{F2FBF8}
\definecolor{myexamplefr}{HTML}{88D6D1}
\definecolor{myexampleti}{HTML}{2A7F7F}
\definecolor{mydefinitbg}{HTML}{E5E5FF}
\definecolor{mydefinitfr}{HTML}{3F3FA3}
\definecolor{notesgreen}{RGB}{0,162,0}
\definecolor{myp}{RGB}{197, 92, 212}
\definecolor{mygr}{HTML}{2C3338}
\definecolor{myred}{RGB}{127,0,0}
\definecolor{myyellow}{RGB}{169,121,69}
\definecolor{myexercisebg}{HTML}{F2FBF8}
\definecolor{myexercisefg}{HTML}{88D6D1}
%%%%%%%%%%%%%%%%%%%%%%%%%%%%%%%%%%%%%%%%%%%%%%%%%%%%%%%%%%%%%%%%%%%%%%
% Box environments for theorems and problems
%%%%%%%%%%%%%%%%%%%%%%%%%%%%%%%%%%%%%%%%%%%%%%%%%%%%%%%%%%%%%%%%%%%%%
\setlength{\parindent}{1cm}
%================================
% Question BOX
%================================
\makeatletter
\newtcbtheorem{question}{Opgave}{enhanced,
	breakable,
	colback=white,
	colframe=myb!80!black,
	attach boxed title to top left={yshift*=-\tcboxedtitleheight},
	fonttitle=\bfseries,
	title={#2},
	boxed title size=title,
	boxed title style={%
			sharp corners,
			rounded corners=northwest,
			colback=tcbcolframe,
			boxrule=0pt,
		},
	underlay boxed title={%
			\path[fill=tcbcolframe] (title.south west)--(title.south east)
			to[out=0, in=180] ([xshift=5mm]title.east)--
			(title.center-|frame.east)
			[rounded corners=\kvtcb@arc] |-
			(frame.north) -| cycle;
		},
	#1
}{def}
\makeatother
%================================
% DEFINITION BOX
%================================

\newtcbtheorem[]{Definition}{Definition}{enhanced,
	before skip=2mm,after skip=2mm, colback=red!5,colframe=red!80!black,boxrule=0.5mm,
	attach boxed title to top left={xshift=1cm,yshift*=1mm-\tcboxedtitleheight}, varwidth boxed title*=-3cm,
	boxed title style={frame code={
					\path[fill=tcbcolback]
					([yshift=-1mm,xshift=-1mm]frame.north west)
					arc[start angle=0,end angle=180,radius=1mm]
					([yshift=-1mm,xshift=1mm]frame.north east)
					arc[start angle=180,end angle=0,radius=1mm];
					\path[left color=tcbcolback!60!black,right color=tcbcolback!60!black,
						middle color=tcbcolback!80!black]
					([xshift=-2mm]frame.north west) -- ([xshift=2mm]frame.north east)
					[rounded corners=1mm]-- ([xshift=1mm,yshift=-1mm]frame.north east)
					-- (frame.south east) -- (frame.south west)
					-- ([xshift=-1mm,yshift=-1mm]frame.north west)
					[sharp corners]-- cycle;
				},interior engine=empty,
		},
	fonttitle=\bfseries,
	title={#2},#1}{def}
\newtcbtheorem[]{definition}{Definition}{enhanced,
	before skip=2mm,after skip=2mm, colback=red!5,colframe=red!80!black,boxrule=0.5mm,
	attach boxed title to top left={xshift=1cm,yshift*=1mm-\tcboxedtitleheight}, varwidth boxed title*=-3cm,
	boxed title style={frame code={
					\path[fill=tcbcolback]
					([yshift=-1mm,xshift=-1mm]frame.north west)
					arc[start angle=0,end angle=180,radius=1mm]
					([yshift=-1mm,xshift=1mm]frame.north east)
					arc[start angle=180,end angle=0,radius=1mm];
					\path[left color=tcbcolback!60!black,right color=tcbcolback!60!black,
						middle color=tcbcolback!80!black]
					([xshift=-2mm]frame.north west) -- ([xshift=2mm]frame.north east)
					[rounded corners=1mm]-- ([xshift=1mm,yshift=-1mm]frame.north east)
					-- (frame.south east) -- (frame.south west)
					-- ([xshift=-1mm,yshift=-1mm]frame.north west)
					[sharp corners]-- cycle;
				},interior engine=empty,
		},
	fonttitle=\bfseries,
	title={#2},#1}{def}

\newtcbtheorem{theo}%
    {Theorem}{}{theorem}
\newtcolorbox{prob}[1]{colback=red!5!white,colframe=red!50!black,fonttitle=\bfseries,title={#1}}
%================================
% NOTE BOX
%================================

\usetikzlibrary{arrows,calc,shadows.blur}
\tcbuselibrary{skins}
\newtcolorbox{note}[1][]{%
	enhanced jigsaw,
	colback=gray!20!white,%
	colframe=gray!80!black,
	size=small,
	boxrule=1pt,
	title=\textbf{Note:},
	halign title=flush center,
	coltitle=black,
	breakable,
	drop shadow=black!50!white,
	attach boxed title to top left={xshift=1cm,yshift=-\tcboxedtitleheight/2,yshifttext=-\tcboxedtitleheight/2},
	minipage boxed title=1.5cm,
	boxed title style={%
			colback=white,
			size=fbox,
			boxrule=1pt,
			boxsep=2pt,
			underlay={%
					\coordinate (dotA) at ($(interior.west) + (-0.5pt,0)$);
					\coordinate (dotB) at ($(interior.east) + (0.5pt,0)$);
					\begin{scope}
						\clip (interior.north west) rectangle ([xshift=3ex]interior.east);
						\filldraw [white, blur shadow={shadow opacity=60, shadow yshift=-.75ex}, rounded corners=2pt] (interior.north west) rectangle (interior.south east);
					\end{scope}
					\begin{scope}[gray!80!black]
						\fill (dotA) circle (2pt);
						\fill (dotB) circle (2pt);
					\end{scope}
				},
		},
	#1,
}
%================================
% EXAMPLE BOX
%================================
\newtcbtheorem[number within=section]{Example}{Example}
{%
	colback = myexamplebg
	,breakable
	,colframe = myexamplefr
	,coltitle = myexampleti
	,boxrule = 1pt
	,sharp corners
	,detach title
	,before upper=\tcbtitle\par\smallskip
	,fonttitle = \bfseries
	,description font = \mdseries
	,separator sign none
	,description delimiters parenthesis
}
{ex}
%================================
% THEOREM BOX
%================================

\tcbuselibrary{theorems,skins,hooks}
\newtcbtheorem[number within=section]{Theorem}{Theorem}
{%
	enhanced,
	breakable,
	colback = mytheorembg,
	frame hidden,
	boxrule = 0sp,
	borderline west = {2pt}{0pt}{mytheoremfr},
	sharp corners,
	detach title,
	before upper = \tcbtitle\par\smallskip,
	coltitle = mytheoremfr,
	fonttitle = \bfseries\sffamily,
	description font = \mdseries,
	separator sign none,
	segmentation style={solid, mytheoremfr},
}
{th}

%%%%%%%%%%%%%%%%%%%%%%%%%%%%%%%%%%%%%%%%%%%%%%%%%%%%%%%%%%%%%%%%%
% SELF MADE COMMANDS
%%%%%%%%%%%%%%%%%%%%%%%%%%%%%%
\newcommand{\sol}{\setlength{\parindent}{0cm}\textbf{\textit{Løsning:}}\setlength{\parindent}{1cm}}
%%%%%%%%%%%%%%%%%%%%%%%%%%%%%%%%%
\usepackage[tmargin=2cm,rmargin=1in,lmargin=1in,margin=0.85in,bmargin=2cm,footskip=.2in]{geometry}\pagestyle{fancy}
\lhead{Minrui Kevin Zhou 3.b}
\rhead{Aflevering}

\title{Fysik prøve 2\\
{\Large \textbf{3.b fysik A}}}
\author{Kevin Zhou}
\date{\today}

\begin{document}
\maketitle
\section*{Opgave 1 - Tennis}
\sol \\
\textbf{a.}
Fra impulssætningen har vi, at gennemsnittet af størrelsen af den samlede kraft, som bolden påvirkes af må være
\begin{equation*}
\begin{split}
  F _{\text{res} }&=\frac{\Delta p}{\Delta t}\\
  &=\frac{m_{\text{bold} } \cdot v_2 - m_{\text{bold} } \cdot v_1}{\Delta t}\\
  &=\frac{60 \cdot 10 ^{-3} \;\unit{kg} \cdot 25 \;\unit{m/s} }{2,2 \cdot 10 ^{-3} \;\unit{s} }\\
  &\approx 6,8 \cdot 10^3 \;\unit{N} 
\end{split}
\end{equation*}
Størrelsen af den samlede kraft, som bolden i gennemsnit er påvirket af under slaget er altså $6,8 \cdot 10^3 \;\unit{N} $.
\section*{Opgave 2 - Rumsonden Sojus 12}
\sol \\
\textbf{a.}
Vi antager, at den givne afstand fra Sojus 12 til plutos overflade er målt fra Sojus 12's massemidtpunkt, og betegner denne afstand for $a$.
Størrelsen af gravitationskraften mellem dem bliver så
\begin{equation*}
\begin{split}
  F_G &= G \cdot \frac{m _{\text{Sojus} } \cdot m _{\text{Pluto} }}{r^2}\\
  &= G \cdot \frac{m _{\text{Sojus} } \cdot m _{\text{Pluto} }}{(d_{\text{Pluto} } + a)^2}\\
  &=6,674 \cdot 10 ^{-11} \;\unit{\frac{N \cdot m^2}{kg^2}} \cdot \frac{470 \;\unit{kg} \cdot 1,31 \cdot 10 ^{22} \;\unit{kg} }{\left(2,37 \cdot 10^6 \;\unit{m} + 1,25 \cdot 10^7 \;\unit{m} \right)^2 }\\
  &\approx 1,86 \;\unit{N} 
\end{split}
\end{equation*}
Størrelsen af gravitationskraften mellem Sojus 12 og Pluto på det tidspunkt de er tættest på hinanden er altså $1,86 \;\unit{N} $. \\[1ex]
\textbf{b.}
Sojus 12 vil fortsætte med at bevæge sig væk fra solen netop hvis summen af dens potentielle og kinetiske energi er ikke negativ:
\begin{equation*}
\begin{split}
  \frac{1}{2} \cdot m \cdot v^2 - G \cdot \frac{m \cdot M _{\text{sol} }}{r } \geq 0 &\iff v \geq \sqrt{2 \cdot G \cdot \frac{M _{\text{sol}}}{r }}
\end{split}
\end{equation*}
Denne størrelse udregner vi
\begin{equation*}
\begin{split}
  \sqrt{2 \cdot G \cdot \frac{M _{\text{sol}}}{r}} &=\sqrt{2 \cdot 6,674 \cdot 10 ^{-11} \;\unit{\frac{N \cdot m^2}{kg^2}} \cdot \frac{1,989 \cdot 10 ^{30} \;\unit{kg} }{4,92 \cdot 10 ^{12} \;\unit{m} }} \\
  &\approx 7,35 \cdot 10^3 \;\unit{m/s} \\
  &=7,35 \;\unit{km/s} 
\end{split}
\end{equation*}
Siden Sojus 12 bevæger sig med farten
\[
14,5 \;\unit{km/s} > 7,35 \;\unit{km/s} 
\] 
så vil den fortsætte med at bevæge sig væk fra solen.

Da der ikke er nogen ydre kræfter, så gælder der, at
\[
\Delta E _{\text{mek} }=0
\] 
Når Sojus 12 ikke er påvirket af solen, er den uendeligt langt væk, og den potentielle energi er 0.
Vi har da 
\begin{equation*}
\begin{split}
  \frac{1}{2} \cdot m \cdot v_2^2=\frac{1}{2} \cdot m \cdot v_1^2 - G \cdot \frac{m \cdot M _{\text{sol} }}{r_1 } \iff v_2=\sqrt{v_1^2-2 \cdot G \cdot \frac{M _{\text{sol} }}{r_1}} 
\end{split}
\end{equation*}
Vi beregner da farten
\begin{equation*}
\begin{split}
  v_2&=\sqrt{v_1^2-2 \cdot G \cdot \frac{M _{\text{sol} }}{r_1}}\\
  &=\sqrt{\left(14,5 \cdot 10^3 \;\unit{m/s} \right)^2 - 2 \cdot 6,674 \cdot 10 ^{-11} \;\unit{\frac{N \cdot m^2}{kg^2}} \cdot \frac{1,989 \cdot 10 ^{30} \;\unit{kg} }{4,92 \cdot 10 ^{12} \;\unit{m} }} \\
  &\approx 12,5 \;\unit{km/s} 
\end{split}
\end{equation*}
Altså er Sojus 12's fart $12,5 \;\unit{km/s} $, når den ikke længere er påvirket af solen.
\section*{Opgave 3 - Kørekort}
\sol \\
\textbf{a.}
Da der er tale om en konstant accelereret bevægelse, så gælder der 
\begin{equation*}
\begin{split}
  v= a \cdot t + v_0 \iff t=\frac{v-v_0}{a}
\end{split}
\end{equation*}
Vi beregner nu tiden det tager, for at bremse op, og vi regner med størrelsen af kræfterne i bagudrettet retning.
\begin{equation*}
\begin{split}
  t&=\frac{v-v_0}{a}\\
  &=\frac{0 \;\unit{m/s} - \left(- 50 \cdot \frac{1000}{60^2} \;\unit{m/s} \right) }{2,9 \;\unit{m/s^2} }\\
  &\approx 4,8 \;\unit{s} 
\end{split}
\end{equation*}
Det tager altså bilen $4,8 \;\unit{s} $ at bremse helt op. 


\end{document}
