\documentclass{report}
\usepackage{graphicx, tikz-cd, float, titlepic, booktabs} % Required for inserting images
\usepackage{amsmath, amssymb, amsthm, amsfonts, siunitx, physics, gensymb}
\AtBeginDocument{\RenewCommandCopy\qty\SI}
\usepackage[version=4]{mhchem}
\usepackage[most,many,breakable]{tcolorbox}
\usepackage{xcolor, fancyhdr, varwidth}
\usepackage[Glenn]{fncychap}
%Options: Sonny, Lenny, Glenn, Conny, Rejne, Bjarne, Bjornstrup
\usepackage{hyperref, cleveref}
\usepackage{icomma, enumitem} %comma as decimal and continue enumerate with [resume]
\usepackage[danish]{babel}
%%%%%%%%%%%%%%%%%%%%%%%%%%%%%%
% SELF MADE COLORS
%%%%%%%%%%%%%%%%%%%%%%%%%%%%%%
\definecolor{myg}{RGB}{56, 140, 70}
\definecolor{myb}{RGB}{45, 111, 177}
\definecolor{myr}{RGB}{199, 68, 64}
\definecolor{mytheorembg}{HTML}{F2F2F9}
\definecolor{mytheoremfr}{HTML}{00007B}
\definecolor{mylenmabg}{HTML}{FFFAF8}
\definecolor{mylenmafr}{HTML}{983b0f}
\definecolor{mypropbg}{HTML}{f2fbfc}
\definecolor{mypropfr}{HTML}{191971}
\definecolor{myexamplebg}{HTML}{F2FBF8}
\definecolor{myexamplefr}{HTML}{88D6D1}
\definecolor{myexampleti}{HTML}{2A7F7F}
\definecolor{mydefinitbg}{HTML}{E5E5FF}
\definecolor{mydefinitfr}{HTML}{3F3FA3}
\definecolor{notesgreen}{RGB}{0,162,0}
\definecolor{myp}{RGB}{197, 92, 212}
\definecolor{mygr}{HTML}{2C3338}
\definecolor{myred}{RGB}{127,0,0}
\definecolor{myyellow}{RGB}{169,121,69}
\definecolor{myexercisebg}{HTML}{F2FBF8}
\definecolor{myexercisefg}{HTML}{88D6D1}
%%%%%%%%%%%%%%%%%%%%%%%%%%%%%%%%%%%%%%%%%%%%%%%%%%%%%%%%%%%%%%%%%%%%%%
% Box environments for theorems and problems
%%%%%%%%%%%%%%%%%%%%%%%%%%%%%%%%%%%%%%%%%%%%%%%%%%%%%%%%%%%%%%%%%%%%%
\setlength{\parindent}{1cm}
%================================
% Question BOX
%================================
\makeatletter
\newtcbtheorem{question}{Opgave}{enhanced,
	breakable,
	colback=white,
	colframe=myb!80!black,
	attach boxed title to top left={yshift*=-\tcboxedtitleheight},
	fonttitle=\bfseries,
	title={#2},
	boxed title size=title,
	boxed title style={%
			sharp corners,
			rounded corners=northwest,
			colback=tcbcolframe,
			boxrule=0pt,
		},
	underlay boxed title={%
			\path[fill=tcbcolframe] (title.south west)--(title.south east)
			to[out=0, in=180] ([xshift=5mm]title.east)--
			(title.center-|frame.east)
			[rounded corners=\kvtcb@arc] |-
			(frame.north) -| cycle;
		},
	#1
}{def}
\makeatother
%================================
% DEFINITION BOX
%================================

\newtcbtheorem[]{Definition}{Definition}{enhanced,
	before skip=2mm,after skip=2mm, colback=red!5,colframe=red!80!black,boxrule=0.5mm,
	attach boxed title to top left={xshift=1cm,yshift*=1mm-\tcboxedtitleheight}, varwidth boxed title*=-3cm,
	boxed title style={frame code={
					\path[fill=tcbcolback]
					([yshift=-1mm,xshift=-1mm]frame.north west)
					arc[start angle=0,end angle=180,radius=1mm]
					([yshift=-1mm,xshift=1mm]frame.north east)
					arc[start angle=180,end angle=0,radius=1mm];
					\path[left color=tcbcolback!60!black,right color=tcbcolback!60!black,
						middle color=tcbcolback!80!black]
					([xshift=-2mm]frame.north west) -- ([xshift=2mm]frame.north east)
					[rounded corners=1mm]-- ([xshift=1mm,yshift=-1mm]frame.north east)
					-- (frame.south east) -- (frame.south west)
					-- ([xshift=-1mm,yshift=-1mm]frame.north west)
					[sharp corners]-- cycle;
				},interior engine=empty,
		},
	fonttitle=\bfseries,
	title={#2},#1}{def}
\newtcbtheorem[]{definition}{Definition}{enhanced,
	before skip=2mm,after skip=2mm, colback=red!5,colframe=red!80!black,boxrule=0.5mm,
	attach boxed title to top left={xshift=1cm,yshift*=1mm-\tcboxedtitleheight}, varwidth boxed title*=-3cm,
	boxed title style={frame code={
					\path[fill=tcbcolback]
					([yshift=-1mm,xshift=-1mm]frame.north west)
					arc[start angle=0,end angle=180,radius=1mm]
					([yshift=-1mm,xshift=1mm]frame.north east)
					arc[start angle=180,end angle=0,radius=1mm];
					\path[left color=tcbcolback!60!black,right color=tcbcolback!60!black,
						middle color=tcbcolback!80!black]
					([xshift=-2mm]frame.north west) -- ([xshift=2mm]frame.north east)
					[rounded corners=1mm]-- ([xshift=1mm,yshift=-1mm]frame.north east)
					-- (frame.south east) -- (frame.south west)
					-- ([xshift=-1mm,yshift=-1mm]frame.north west)
					[sharp corners]-- cycle;
				},interior engine=empty,
		},
	fonttitle=\bfseries,
	title={#2},#1}{def}

\newtcbtheorem{theo}%
    {Theorem}{}{theorem}
\newtcolorbox{prob}[1]{colback=red!5!white,colframe=red!50!black,fonttitle=\bfseries,title={#1}}
%================================
% NOTE BOX
%================================

\usetikzlibrary{arrows,calc,shadows.blur}
\tcbuselibrary{skins}
\newtcolorbox{note}[1][]{%
	enhanced jigsaw,
	colback=gray!20!white,%
	colframe=gray!80!black,
	size=small,
	boxrule=1pt,
	title=\textbf{Note:},
	halign title=flush center,
	coltitle=black,
	breakable,
	drop shadow=black!50!white,
	attach boxed title to top left={xshift=1cm,yshift=-\tcboxedtitleheight/2,yshifttext=-\tcboxedtitleheight/2},
	minipage boxed title=1.5cm,
	boxed title style={%
			colback=white,
			size=fbox,
			boxrule=1pt,
			boxsep=2pt,
			underlay={%
					\coordinate (dotA) at ($(interior.west) + (-0.5pt,0)$);
					\coordinate (dotB) at ($(interior.east) + (0.5pt,0)$);
					\begin{scope}
						\clip (interior.north west) rectangle ([xshift=3ex]interior.east);
						\filldraw [white, blur shadow={shadow opacity=60, shadow yshift=-.75ex}, rounded corners=2pt] (interior.north west) rectangle (interior.south east);
					\end{scope}
					\begin{scope}[gray!80!black]
						\fill (dotA) circle (2pt);
						\fill (dotB) circle (2pt);
					\end{scope}
				},
		},
	#1,
}

%%%%%%%%%%%%%%%%%%%%%%%%%%%%%%%%%%%%%%%%%%%%%%%%%%%%%%%%%%%%%%%%%
% SELF MADE COMMANDS
%%%%%%%%%%%%%%%%%%%%%%%%%%%%%%
\newcommand{\sol}{\setlength{\parindent}{0cm}\textbf{\textit{Løsning:}}\setlength{\parindent}{1cm}}
%%%%%%%%%%%%%%%%%%%%%%%%%%%%%%%%%
\usepackage[tmargin=2cm,rmargin=1in,lmargin=1in,margin=0.85in,bmargin=2cm,footskip=.2in]{geometry}\pagestyle{fancy}
\lhead{Minrui Kevin Zhou 2.b}
\rhead{H9}

\title{H9\\
{\Large \textbf{2.b fys A}}}
\author{Kevin Zhou}
\date{Januar 2024}

\begin{document}
\maketitle
\begin{question}{Luftskib}{}
  Et luftskib med rumfanget $1390 \;\unit{m^3} $ er fyldt med helium. 
  Trykket både inden- og udenfor er $102,1 \;\unit{kPa}$, og temperaturen er $18,0 \;\unit{\celsius} $ 
  \begin{itemize}
    \item[a.] Hvad er stofmængden af helium i luftskibet?
  \end{itemize}
Heliums molare masse er $4,00 \;\unit{ g/mol}$.
\begin{itemize}
  \item[b.] Hvad er massen af helium i luftskibet?
\end{itemize}
Den molare masse af atmosfærisk luft er $28,9 \;\unit{ g/mol}$.
\begin{itemize}
  \item[c.] Hvor stor er opdriften på luftskibet?
\end{itemize}
Grafen i opgaven viser, hvordan densiteten af den atmosfæriske luft afhænger af højden over overfladen.
Luftskibet stiger nu op til en vis højde, hvor det holder sig svævende. Luftskibets masse uden helium er $1280 \;\unit{ kg}$
\begin{itemize}
  \item[d.] Find svævehøjden.
\end{itemize}
\end{question}
\sol \\ 
\textbf{a.}
Vi bruger idealgasloven.
\begin{equation*}
\begin{split}
  n&=\frac{p\cdot V}{R\cdot T}\\ 
  &=\frac{102,3\cdot 10^3 \;\unit{Pa} \cdot 1390 \;\unit{m^3} }{8,31 \;\unit{\frac{Pa \cdot m^3}{mol\cdot K}}\cdot 291 \;\unit{K} }\\ 
  &\approx 58,8 \;\unit{kmol} 
\end{split}
\end{equation*}
Altså er stofmængden af helium i luftskibet $58,8 \;\unit{kmol} $. \\[1ex]
\textbf{b.}
Vi bruger stofmængden af helium fra opgave \textbf{a.} til at udregne massen.
\begin{equation*}
\begin{split}
  m_{\text{helium} }&=M_{\text{helium} }\cdot n_{\text{helium} }\\ 
  &=4,00 \;\unit{g/mol} \cdot \frac{102,3\cdot 10^3  \cdot 1390 }{{8,31 }\cdot 291 }\;\unit{mol} \\ 
  &\approx 235 \;\unit{kg} 
\end{split}
\end{equation*}
Altså er massen af helium i luftskibet $235 \;\unit{kg} $.\\[1ex]
\textbf{c.}
Kraften fra luften på luftskibet er lige så stor som tyngdekraften på den fortrængte luft.
\begin{equation*}
\begin{split}
  F_{op}&=m_{\text{luft} }\cdot g\\ 
  &=\frac{102,3\cdot 10^3  \cdot 1390 }{{8,31 }\cdot 291 }\;\unit{mol} \cdot 0,0289 \;\unit{kg/mol} \cdot 9,82 \;\unit{m/s^2} \\ 
  &\approx 16,7 \;\unit{kN}
\end{split}
\end{equation*}
Altså er opdriften på luftskibet $16,7 \;\unit{kN} $.\\[1ex]
\textbf{d.}
Skibet svæver, når gennemsnitsdensiteten af luftskibet er lig med luftens densitet.
Skibets densitet (med heliums masse fra \textbf{b}) er da
\begin{equation*}
\begin{split}
  \rho_{\text{skib} }=\frac{1280 \;\unit{kg} +m_{\text{helium} }}{1390 \;\unit{m^3} } \approx 1,09 \;\unit{kg/m^3} 
\end{split}
\end{equation*}
Vi aflæser på grafen, og får svævehøjden til at være $1,19\;\unit{km}$.
\begin{question}{Dykkerflaske}{}
 En dykker bruger dykkerflasker til sin luftforsyning under vandet.
Hver dykkerflaske er en stålbeholder, som rummer $7,4 \;\unit{ L}$.
Før brug pumpes atmosfærisk luft ind i flasken, så trykket i flasken er $20,6 \;\unit{ MPa}$ ved 21,0 °C.
  Lufts molar masse er $29,0 \;\unit{ g/mol}$.  
\begin{itemize}
  \item[a.] Hvad er massen af luften i en dykkerflaske, når den er fyldt med luft og klar til brug?
\end{itemize}
Under dykningen forbruger dykkeren på et minut $33 \;\unit{ L}$ luft ved trykket $0,10 \;\unit{ MPa}$ og temperaturen $18,1 \;\unit{\celsius} $.
\begin{itemize}
  \item[b.] Hvor længe kan dykkeren opholde sig under vand, når hun er forsynet med to dykkerflasker?
\end{itemize}
Massen af en tom dykkerflaske er $7,6 \;\unit{ kg}$, og ståls densitet er $7,9 \;\unit{ kg/L}$. En dykker befinder sig helt nede i vandet.
\begin{itemize}
  \item[c.] Bestem størrelse og retning af den kraft, hvormed en dykkerflaske påvirker dykkeren, når flasken er tom.
  \item[d.] Bestem størrelse og retning af den kraft, hvormed en dykkerflaske påvirker dykkeren, når flasken er fuld.
\end{itemize}
\end{question}
\sol \\
\textbf{a.}
Vi finder stofmængden af luft i beholderen og ganger så med den molare masse for luft.
\begin{equation*}
\begin{split}
  m&=\frac{p\cdot V}{R\cdot T} \cdot M \\ 
  &=\frac{20,6 \cdot 10^6 \;\unit{Pa} \cdot 0,0074 \;\unit{m^3} }{8,31 \;\unit{\frac{Pa \cdot m^3}{mol \cdot K}} \cdot 294 \;\unit{K} } \cdot 29,0 \;\unit{g/mol} \\ 
  &\approx 1,8 \;\unit{kg} 
\end{split}
\end{equation*}
Altså er massen af luft i dykkerflasken $1,8 \;\unit{kg} $. \\[1ex]
\textbf{b.}
Vi finder da forholdet mellem stofmængden, der forbruges per minut og stofmængden af luften i to dykkerflasker.
\begin{equation*}
\begin{split}
  n_{\text{per min} }&=\frac{0,10 \cdot 10^6 \;\unit{Pa} \cdot 0,033 \;\unit{m^3} }{8,31 \;\unit{\frac{Pa \cdot m^3}{mol \cdot K}} \cdot 291,1 \;\unit{K} }\\ 
  & \frac{n_{ \text{flasker} }}{n_{\text{per min}}}=\frac{2 \cdot 20,6 \cdot 10^6 \cdot 0,0074}{8,31 \cdot 294}\cdot \frac{8,31 \cdot 291,1}{0,033 \cdot 10^5} \approx 91
\end{split}
\end{equation*}
Altså kan dykkeren være under vand i $91$ minutter. \\[1ex]
\textbf{c.}
Volumenet af flaskens stål må være
\[
V_{\text{stål} }=\frac{m_{\text{tom flaske} }}{\rho_{\text{stål} }}= \frac{7,6 }{7,9\cdot 10^3} \;\unit{m^3} 
\] 
Vi bestemmer størrelsen af kræften nedad, der er længden af tyngdekraften minus længden af opdriften.
\begin{equation*}
\begin{split}
  F_{\text{ned} }&=\left(m_{\text{flaske}}- \rho_{\text{vand} } \cdot V_{\text{flaske} }\right) \cdot g \\ 
  &=\left(7,6 \;\unit{kg} - 1000 \;\unit{kg/m^3} \cdot \left(0,0074 \;\unit{m^3} +\frac{7,6}{7,9 \cdot 10^3} \;\unit{m^3} \right)\right) \cdot 9,82 \;\unit{m/s^2} \\ 
  &\approx -7,5 \;\unit{N} 
\end{split}
\end{equation*}
Altså er er størrelsen på kraften $7,5 \;\unit{N} $ og retningen er opad. \\[1ex]
\textbf{d.}
Vi bestemmer igen kræften nedad. 
Vi bruger massen af luften fra \textbf{a.}
\begin{equation*}
\begin{split}
  F_{\text{ned}} &=\left(m_{\text{flaske}}+m_{\text{luft} }- \rho_{\text{vand} } \cdot V_{\text{flaske} }\right) \cdot g \\ 
  &= \left(7,6 \;\unit{kg} + m_{\text{luft} }- 1000 \;\unit{kg/m^3} \cdot \left(0,0074 \;\unit{m^3} +\frac{7,6}{7,9 \cdot 10^3} \;\unit{m^3} \right)\right) \cdot 9,82 \;\unit{m/s^2} \\ 
  &\approx 10 \;\unit{N} 
\end{split}
\end{equation*}
Altså er kraften $10 \;\unit{N} $ og retningen er nedad. 
\end{document}
