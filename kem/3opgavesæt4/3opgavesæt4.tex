\documentclass{report}
\usepackage{graphicx, tikz-cd, float, titlepic, booktabs} % Required for inserting images
\usepackage{pgfplots}
\pgfplotsset{compat=1.15}
\usepackage{mathrsfs}
\usetikzlibrary{arrows}
\usepackage{amsmath, amssymb, amsthm, amsfonts, siunitx, physics, gensymb}
\AtBeginDocument{\RenewCommandCopy\qty\SI}
\usepackage[version=4]{mhchem}
\usepackage[most,many,breakable]{tcolorbox}
\usepackage{xcolor, fancyhdr, varwidth}
\usepackage[Glenn]{fncychap}
%Options: Sonny, Lenny, Glenn, Conny, Rejne, Bjarne, Bjornstrup
\usepackage{hyperref, cleveref}
\usepackage{icomma, enumitem} %comma as decimal and continue enumerate with [resume]
\usepackage{plimsoll} %use standard state symbol with \stst
\usepackage[danish]{babel}
%%%%%%%%%%%%%%%%%%%%%%%%%%%%%%
% SELF MADE COLORS
%%%%%%%%%%%%%%%%%%%%%%%%%%%%%%
\definecolor{myg}{RGB}{56, 140, 70}
\definecolor{myb}{RGB}{45, 111, 177}
\definecolor{myr}{RGB}{199, 68, 64}
\definecolor{mytheorembg}{HTML}{F2F2F9}
\definecolor{mytheoremfr}{HTML}{00007B}
\definecolor{mylenmabg}{HTML}{FFFAF8}
\definecolor{mylenmafr}{HTML}{983b0f}
\definecolor{mypropbg}{HTML}{f2fbfc}
\definecolor{mypropfr}{HTML}{191971}
\definecolor{myexamplebg}{HTML}{F2FBF8}
\definecolor{myexamplefr}{HTML}{88D6D1}
\definecolor{myexampleti}{HTML}{2A7F7F}
\definecolor{mydefinitbg}{HTML}{E5E5FF}
\definecolor{mydefinitfr}{HTML}{3F3FA3}
\definecolor{notesgreen}{RGB}{0,162,0}
\definecolor{myp}{RGB}{197, 92, 212}
\definecolor{mygr}{HTML}{2C3338}
\definecolor{myred}{RGB}{127,0,0}
\definecolor{myyellow}{RGB}{169,121,69}
\definecolor{myexercisebg}{HTML}{F2FBF8}
\definecolor{myexercisefg}{HTML}{88D6D1}
%%%%%%%%%%%%%%%%%%%%%%%%%%%%%%%%%%%%%%%%%%%%%%%%%%%%%%%%%%%%%%%%%%%%%%
% Box environments for theorems and problems
%%%%%%%%%%%%%%%%%%%%%%%%%%%%%%%%%%%%%%%%%%%%%%%%%%%%%%%%%%%%%%%%%%%%%
\setlength{\parindent}{1cm}
%================================
% Question BOX
%================================
\makeatletter
\newtcbtheorem{question}{Opgave}{enhanced,
	breakable,
	colback=white,
	colframe=myb!80!black,
	attach boxed title to top left={yshift*=-\tcboxedtitleheight},
	fonttitle=\bfseries,
	title={#2},
	boxed title size=title,
	boxed title style={%
			sharp corners,
			rounded corners=northwest,
			colback=tcbcolframe,
			boxrule=0pt,
		},
	underlay boxed title={%
			\path[fill=tcbcolframe] (title.south west)--(title.south east)
			to[out=0, in=180] ([xshift=5mm]title.east)--
			(title.center-|frame.east)
			[rounded corners=\kvtcb@arc] |-
			(frame.north) -| cycle;
		},
	#1
}{def}
\makeatother
%================================
% DEFINITION BOX
%================================

\newtcbtheorem[]{Definition}{Definition}{enhanced,
	before skip=2mm,after skip=2mm, colback=red!5,colframe=red!80!black,boxrule=0.5mm,
	attach boxed title to top left={xshift=1cm,yshift*=1mm-\tcboxedtitleheight}, varwidth boxed title*=-3cm,
	boxed title style={frame code={
					\path[fill=tcbcolback]
					([yshift=-1mm,xshift=-1mm]frame.north west)
					arc[start angle=0,end angle=180,radius=1mm]
					([yshift=-1mm,xshift=1mm]frame.north east)
					arc[start angle=180,end angle=0,radius=1mm];
					\path[left color=tcbcolback!60!black,right color=tcbcolback!60!black,
						middle color=tcbcolback!80!black]
					([xshift=-2mm]frame.north west) -- ([xshift=2mm]frame.north east)
					[rounded corners=1mm]-- ([xshift=1mm,yshift=-1mm]frame.north east)
					-- (frame.south east) -- (frame.south west)
					-- ([xshift=-1mm,yshift=-1mm]frame.north west)
					[sharp corners]-- cycle;
				},interior engine=empty,
		},
	fonttitle=\bfseries,
	title={#2},#1}{def}
\newtcbtheorem[]{definition}{Definition}{enhanced,
	before skip=2mm,after skip=2mm, colback=red!5,colframe=red!80!black,boxrule=0.5mm,
	attach boxed title to top left={xshift=1cm,yshift*=1mm-\tcboxedtitleheight}, varwidth boxed title*=-3cm,
	boxed title style={frame code={
					\path[fill=tcbcolback]
					([yshift=-1mm,xshift=-1mm]frame.north west)
					arc[start angle=0,end angle=180,radius=1mm]
					([yshift=-1mm,xshift=1mm]frame.north east)
					arc[start angle=180,end angle=0,radius=1mm];
					\path[left color=tcbcolback!60!black,right color=tcbcolback!60!black,
						middle color=tcbcolback!80!black]
					([xshift=-2mm]frame.north west) -- ([xshift=2mm]frame.north east)
					[rounded corners=1mm]-- ([xshift=1mm,yshift=-1mm]frame.north east)
					-- (frame.south east) -- (frame.south west)
					-- ([xshift=-1mm,yshift=-1mm]frame.north west)
					[sharp corners]-- cycle;
				},interior engine=empty,
		},
	fonttitle=\bfseries,
	title={#2},#1}{def}

\newtcbtheorem{theo}%
    {Theorem}{}{theorem}
\newtcolorbox{prob}[1]{colback=red!5!white,colframe=red!50!black,fonttitle=\bfseries,title={#1}}
%================================
% NOTE BOX
%================================

\usetikzlibrary{arrows,calc,shadows.blur}
\tcbuselibrary{skins}
\newtcolorbox{note}[1][]{%
	enhanced jigsaw,
	colback=gray!20!white,%
	colframe=gray!80!black,
	size=small,
	boxrule=1pt,
	title=\textbf{Note:},
	halign title=flush center,
	coltitle=black,
	breakable,
	drop shadow=black!50!white,
	attach boxed title to top left={xshift=1cm,yshift=-\tcboxedtitleheight/2,yshifttext=-\tcboxedtitleheight/2},
	minipage boxed title=1.5cm,
	boxed title style={%
			colback=white,
			size=fbox,
			boxrule=1pt,
			boxsep=2pt,
			underlay={%
					\coordinate (dotA) at ($(interior.west) + (-0.5pt,0)$);
					\coordinate (dotB) at ($(interior.east) + (0.5pt,0)$);
					\begin{scope}
						\clip (interior.north west) rectangle ([xshift=3ex]interior.east);
						\filldraw [white, blur shadow={shadow opacity=60, shadow yshift=-.75ex}, rounded corners=2pt] (interior.north west) rectangle (interior.south east);
					\end{scope}
					\begin{scope}[gray!80!black]
						\fill (dotA) circle (2pt);
						\fill (dotB) circle (2pt);
					\end{scope}
				},
		},
	#1,
}
%================================
% EXAMPLE BOX
%================================
\newtcbtheorem[number within=section]{Example}{Example}
{%
	colback = myexamplebg
	,breakable
	,colframe = myexamplefr
	,coltitle = myexampleti
	,boxrule = 1pt
	,sharp corners
	,detach title
	,before upper=\tcbtitle\par\smallskip
	,fonttitle = \bfseries
	,description font = \mdseries
	,separator sign none
	,description delimiters parenthesis
}
{ex}
%================================
% THEOREM BOX
%================================

\tcbuselibrary{theorems,skins,hooks}
\newtcbtheorem[number within=section]{Theorem}{Theorem}
{%
	enhanced,
	breakable,
	colback = mytheorembg,
	frame hidden,
	boxrule = 0sp,
	borderline west = {2pt}{0pt}{mytheoremfr},
	sharp corners,
	detach title,
	before upper = \tcbtitle\par\smallskip,
	coltitle = mytheoremfr,
	fonttitle = \bfseries\sffamily,
	description font = \mdseries,
	separator sign none,
	segmentation style={solid, mytheoremfr},
}
{th}

%%%%%%%%%%%%%%%%%%%%%%%%%%%%%%%%%%%%%%%%%%%%%%%%%%%%%%%%%%%%%%%%%
% SELF MADE COMMANDS
%%%%%%%%%%%%%%%%%%%%%%%%%%%%%%
\newcommand{\sol}{\setlength{\parindent}{0cm}\textbf{\textit{Løsning:}}\setlength{\parindent}{1cm}}
%%%%%%%%%%%%%%%%%%%%%%%%%%%%%%%%%
\usepackage[tmargin=2cm,rmargin=1in,lmargin=1in,margin=0.85in,bmargin=2cm,footskip=.2in]{geometry}\pagestyle{fancy}
\lhead{Minrui Kevin Zhou 3.b}
\rhead{Opgavesæt 4}

\title{Opgavesæt 4\\
{\Large \textbf{3.b kemi A}}}
\author{Kevin Zhou}
\date{\today}

\begin{document}
\maketitle
\begin{note}
  Databog fysik kemi (2007) er benyttet ved beregningerne.
\end{note}
\section*{Brezel}
\sol \\
\textbf{a.}
Reaktionsforholdet mellem \ce{Na2CO3.10H2O} og \ce{Na2CO3} er 1:1.
Der gælder altså 
\begin{equation*}
\begin{split}
  c(\ce{NaCO3} )&=\frac{n(\ce{NaCO3} )}{V}\\
  &=\frac{m(\ce{NaCO3.10H2O} )}{V \cdot M(\ce{NaCO3.10H2O})}\\
  &=\frac{54 \;\unit{g} }{1,0 \;\unit{L} \cdot 286,14 \;\unit{g/mol} }\\
  &\approx 0,19 \;\unit{\textsc{m}} 
\end{split}
\end{equation*}
I opløsningen er altså $c(\ce{NaCO3} )=0,19 \;\unit{\textsc{m}} $.\\[1ex]
\textbf{b.}
\ce{CO3^2-} er en middelstærk base med styrkekonstanten $K_b=2,09 \cdot 10^{-4} \;\unit{\textsc{m}} $.
Siden $c(\ce{CO3^2-} )=c(\ce{NaCO3} )$ og $pH=14,00 + \log\left(\frac{\left[\ce{OH-} \right]}{\textsc{m}}\right) $ , så har vi 
\begin{equation*}
\begin{split}
  K_b=\frac{\left[\ce{OH-} \right]^2}{c_b-[\ce{OH-} ]} &\implies pH=14,00 + \log\left(\frac{-K_b + \sqrt{K_b^2 - 4 \cdot \left(-K_b\right) \cdot c_b } }{2 \;\unit{\textsc{m}} }\right)\\
  &\implies pH=14,00 + \log\left(\frac{-2,09 \cdot 10 ^{-4} \;\unit{\textsc{m}} +\sqrt{\left(2,09 \cdot 10 ^{-4}\;\unit{\textsc{m}} \right)^2-4 \cdot \left(-2,09 \cdot 10 ^{-4} \;\unit{\textsc{m}}\right) \cdot 0,1887 \;\unit{\textsc{m}} } }{2 \;\unit{\textsc{m}} }\right) \approx 12 \\
\end{split}
\end{equation*}
I natriumcarbonatopløsningen er altså $pH=12$.\\[1ex]
\textbf{c.}
Titreringsreaktionen er 
\[
\ce{HCl(aq) + OH-(aq) -> Cl-(aq) + H2O(l)} 
\] 
Altså er reaktionsforholdet mellem syren og basen 1:1.
Ved ækvivalenspunktet må $n(\ce{NaOH} )=n(\ce{HCl} )$.
Det aflæses ved ækvivalenspunktet, at $V(\ce{HCl} )=33,0 \;\unit{mL} $.
Altså må $n(\ce{NaOH} )$ være 
\begin{equation*}
\begin{split}
  n(\ce{NaOH} )&=n(\ce{HCl} )\\
  &=c(\ce{HCl} ) \cdot V(\ce{HCl} )\\
  &=0,097 \;\unit{\textsc{m}} \cdot 33,0 \;\unit{mL} \\
  &=3,201 \;\unit{mmol} 
\end{split}
\end{equation*}
Stofmængdekoncentrationen af natriumhydroxid i bagerens opløsning må da være
\begin{equation*}
\begin{split}
  c(\ce{NaOH} )&=\frac{n(\ce{NaOH} )}{V}\\
  &=\frac{3,201 \;\unit{mmol}}{5,0 \;\unit{mL} }\\
  &\approx 0,64 \;\unit{\textsc{m}}  < 1,0 \;\unit{\textsc{m}}  
\end{split}
\end{equation*}
Altså overholder bagerens natriumhydroxidopløsning reglerne.

\section*{Wolfram i elektroniske komponenter}
\sol \\
\textbf{a.}
Reaktionsbrøken for reaktionen fra venstre mod højre må være
\begin{equation*}
\begin{split}
  \frac{p(\ce{HF} )^6}{p(\ce{ H2})^3 \cdot p(\ce{WF6} )}
\end{split}
\end{equation*}
Siden enheden for partialtrykkene er $\unit{bar}$, så må reaktionsbrøkens enhed være
\[
\frac{\unit{bar}^6}{\unit{bar}^3 \cdot \unit{bar}}=\unit{bar}^2
\] 
\textbf{b.}
Vi beregner tilvæksten i molar standard entropi.
\begin{equation*}
\begin{split}
  \Delta S \stst &=\left(S \stst \left(\ce{W(s)} \right) + 6 \cdot S \stst \left(\ce{HF(g)} \right) \right) - \left(S \stst \left(\ce{WF6(g)} \right) +3 \cdot S \stst \left(\ce{H2(g)} \right) \right) \\
  &=\left(32,6 \;\unit{\frac{J}{mol \cdot K}} + 6 \cdot 173,8 \;\unit{\frac{J}{mol \cdot K}} \right) - \left(354 \;\unit{\frac{J}{mol \cdot K}} +3 \cdot 130,68 \;\unit{\frac{J}{mol \cdot K}} \right) \\
  &\approx 329 \;\unit{\frac{J}{mol \cdot K}} >0
\end{split}
\end{equation*}
Altså øges uordenen i reaktionen fra venstre mod højre.
Det er i overensstemmelse med, at antallet af molekyler stiger i den retning, hvilket fremgår af reaktionsskemaet.\\[1ex]
\textbf{c.}
Vi beregner først den molare tilvækst i standardentalpi.
\begin{equation*}
\begin{split}
  \Delta H \stst &=\left(H \stst \left(\ce{W(s)} \right) + 6 \cdot H \stst \left(\ce{HF(g)} \right) \right) - \left(H \stst \left(\ce{WF6(g)} \right) +3 \cdot H \stst \left(\ce{H2(g)} \right) \right) \\
  &=\left(0+6 \cdot (-273,3 \;\unit{kJ/mol}) \right) -\left(-1721,5 \;\unit{kJ/mol} +0\right) \\
  &=81,7 \;\unit{kJ/mol} 
\end{split}
\end{equation*}
Fra van't Hoffs ligning har vi, at
\begin{equation*}
\begin{split}
  K&=e^{-\frac{\Delta H \stst }{R \cdot T} + \frac{\Delta S \stst }{R}} \;\unit{bar^2} \\
  &=e^{-\frac{81,7 \cdot 10^3 \;\unit{J/mol} }{8,314 \;\unit{J/(mol \cdot K)} \cdot 573 \;\unit{K} }+\frac{329,36 \;\unit{J/(mol \cdot K)} }{8,314 \;\unit{J/(mol \cdot K)} }} \;\unit{bar^2} \\
  &\approx 5,71 \cdot 10^9 \;\unit{bar^2} 
\end{split}
\end{equation*}
Ved $300 \;\unit{\celsius} $ er altså $K=5,71 \cdot 10^9 \;\unit{bar^2} $. \\[1ex]
\textbf{d.}
Vi udregner reaktionsbrøken.
\begin{equation*}
\begin{split}
  \frac{p(\ce{HF} )^6}{p(\ce{ H2})^3 \cdot p(\ce{WF6} )}&=\frac{\left(0,0011 \;\unit{bar} \right)^6}{\left(0,0439 \;\unit{bar} \right)^3 \cdot 0,0018 \;\unit{bar} }\\
  &\approx 1,2 \cdot 10 ^{-11} \;\unit{bar^2} <K 
\end{split}
\end{equation*}
Altså er ligevægten forskudt mod venstre, hvilket betyder, at reaktionen mod højre må forløbe spontant i beholderen.

\section*{Antimon – en miljøgift i det antikke Rom}
\sol \\
\textbf{a.}
På venstre side har \ce{SB} oxidationstallet $0$, hvor det på højre side har oxidationstallet +III.
Siden oxidationstallet stiger, sker der en oxidation.

På venstre side har \ce{O} oxidationstallet $0$, hvor det på højre side har oxidationstallet -II. 
Siden oxidationstallet falder, sker der en reduktion.
Da reaktionen både indeholder en oxidation og en reduktion, så må reaktionen være en redoxreaktion.\\[1ex]
\textbf{b.}
Da reaktionsforholdet mellem \ce{K2Sb2(C4H2O6)2} og \ce{Sb} er 1:2 ved opløsningen, så må der gælde, at
\begin{equation*}
\begin{split}
  c(\ce{Sb} )&=2 \cdot c(\ce{K2Sb2(C4H2O6)2})\\
  &=2 \cdot \frac{m(\ce{K2Sb2(C4H2O6)2})}{M(\ce{K2Sb2(C4H2O6)2}) \cdot V}\\
  &=2 \cdot \frac{15,4\;\unit{mg} }{613,83 \;\unit{g/mol} \cdot 500 \;\unit{mL} }\\
  &\approx 1,00 \cdot 10 ^{-4} \;\unit{\textsc{m}} \\
  &=0,100 \;\unit{m \textsc{m}} 
\end{split}
\end{equation*}
hvilket var, hvad vi skulle vise.\\[1ex]
\textbf{c.}
Fra standardkurven har vi, at
\begin{equation*}
\begin{split}
  A=73,671 \cdot \frac{c_{\text{tm}}(\ce{Sb} )}{\unit{m \textsc{m}}}+0,001 &\iff c_{\text{tm}}(\ce{Sb} )=\frac{A-0,001}{73,671} \;\unit{m \textsc{m}}\\
  &\iff c_{\text{vand}}(\ce{Sb} )=\frac{10,0 \;\unit{mL} }{250 \;\unit{mL} } \cdot \frac{A-0,001}{73,671} \;\unit{m \textsc{m}}\\
  &\iff c_{\text{vand}}(\ce{Sb} )=\frac{A-0,001}{25 \cdot 73,671} \;\unit{m \textsc{m}}
\end{split}
\end{equation*}
hvor $c_{\text{tm} }(\ce{Sb} )$ er den formelle koncentration af antimon i trichlormethanfasen, og $c_{\text{vand}}(\ce{Sb} )$ er den formelle koncentration af antimon i vandprøven.
Indholdet af antimon i vandprøven må da være
\begin{equation*}
\begin{split}
  c_{\text{vand} }(\ce{Sb} ) \cdot M(\ce{Sb})&=\frac{0,573-0,001}{25 \cdot 73,671} \;\unit{m \textsc{m}} \cdot 121,76 \;\unit{g/mol} \\
  &\approx 0,0378 \;\unit{mg/L} 
\end{split}
\end{equation*}
Altså er indholdet af antimon i vandprøven $0,0378 \;\unit{mg/L} $.

\end{document}
