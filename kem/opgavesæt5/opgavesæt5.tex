\documentclass{report}
\usepackage{graphicx, tikz-cd, float, titlepic, booktabs} % Required for inserting images
\usepackage{pgfplots}
\pgfplotsset{compat=1.15}
\usepackage{mathrsfs}
\usetikzlibrary{arrows}
\usepackage{amsmath, amssymb, amsthm, amsfonts, siunitx, physics, gensymb}
\AtBeginDocument{\RenewCommandCopy\qty\SI}
\usepackage[version=4]{mhchem}
\usepackage[most,many,breakable]{tcolorbox}
\usepackage{xcolor, fancyhdr, varwidth}
\usepackage[Glenn]{fncychap}
%Options: Sonny, Lenny, Glenn, Conny, Rejne, Bjarne, Bjornstrup
\usepackage{hyperref, cleveref}
\usepackage{icomma, enumitem} %comma as decimal and continue enumerate with [resume]
\usepackage[danish]{babel}
%%%%%%%%%%%%%%%%%%%%%%%%%%%%%%
% SELF MADE COLORS
%%%%%%%%%%%%%%%%%%%%%%%%%%%%%%
\definecolor{myg}{RGB}{56, 140, 70}
\definecolor{myb}{RGB}{45, 111, 177}
\definecolor{myr}{RGB}{199, 68, 64}
\definecolor{mytheorembg}{HTML}{F2F2F9}
\definecolor{mytheoremfr}{HTML}{00007B}
\definecolor{mylenmabg}{HTML}{FFFAF8}
\definecolor{mylenmafr}{HTML}{983b0f}
\definecolor{mypropbg}{HTML}{f2fbfc}
\definecolor{mypropfr}{HTML}{191971}
\definecolor{myexamplebg}{HTML}{F2FBF8}
\definecolor{myexamplefr}{HTML}{88D6D1}
\definecolor{myexampleti}{HTML}{2A7F7F}
\definecolor{mydefinitbg}{HTML}{E5E5FF}
\definecolor{mydefinitfr}{HTML}{3F3FA3}
\definecolor{notesgreen}{RGB}{0,162,0}
\definecolor{myp}{RGB}{197, 92, 212}
\definecolor{mygr}{HTML}{2C3338}
\definecolor{myred}{RGB}{127,0,0}
\definecolor{myyellow}{RGB}{169,121,69}
\definecolor{myexercisebg}{HTML}{F2FBF8}
\definecolor{myexercisefg}{HTML}{88D6D1}
%%%%%%%%%%%%%%%%%%%%%%%%%%%%%%%%%%%%%%%%%%%%%%%%%%%%%%%%%%%%%%%%%%%%%%
% Box environments for theorems and problems
%%%%%%%%%%%%%%%%%%%%%%%%%%%%%%%%%%%%%%%%%%%%%%%%%%%%%%%%%%%%%%%%%%%%%
\setlength{\parindent}{1cm}
%================================
% Question BOX
%================================
\makeatletter
\newtcbtheorem{question}{Opgave}{enhanced,
	breakable,
	colback=white,
	colframe=myb!80!black,
	attach boxed title to top left={yshift*=-\tcboxedtitleheight},
	fonttitle=\bfseries,
	title={#2},
	boxed title size=title,
	boxed title style={%
			sharp corners,
			rounded corners=northwest,
			colback=tcbcolframe,
			boxrule=0pt,
		},
	underlay boxed title={%
			\path[fill=tcbcolframe] (title.south west)--(title.south east)
			to[out=0, in=180] ([xshift=5mm]title.east)--
			(title.center-|frame.east)
			[rounded corners=\kvtcb@arc] |-
			(frame.north) -| cycle;
		},
	#1
}{def}
\makeatother
%================================
% DEFINITION BOX
%================================

\newtcbtheorem[]{Definition}{Definition}{enhanced,
	before skip=2mm,after skip=2mm, colback=red!5,colframe=red!80!black,boxrule=0.5mm,
	attach boxed title to top left={xshift=1cm,yshift*=1mm-\tcboxedtitleheight}, varwidth boxed title*=-3cm,
	boxed title style={frame code={
					\path[fill=tcbcolback]
					([yshift=-1mm,xshift=-1mm]frame.north west)
					arc[start angle=0,end angle=180,radius=1mm]
					([yshift=-1mm,xshift=1mm]frame.north east)
					arc[start angle=180,end angle=0,radius=1mm];
					\path[left color=tcbcolback!60!black,right color=tcbcolback!60!black,
						middle color=tcbcolback!80!black]
					([xshift=-2mm]frame.north west) -- ([xshift=2mm]frame.north east)
					[rounded corners=1mm]-- ([xshift=1mm,yshift=-1mm]frame.north east)
					-- (frame.south east) -- (frame.south west)
					-- ([xshift=-1mm,yshift=-1mm]frame.north west)
					[sharp corners]-- cycle;
				},interior engine=empty,
		},
	fonttitle=\bfseries,
	title={#2},#1}{def}
\newtcbtheorem[]{definition}{Definition}{enhanced,
	before skip=2mm,after skip=2mm, colback=red!5,colframe=red!80!black,boxrule=0.5mm,
	attach boxed title to top left={xshift=1cm,yshift*=1mm-\tcboxedtitleheight}, varwidth boxed title*=-3cm,
	boxed title style={frame code={
					\path[fill=tcbcolback]
					([yshift=-1mm,xshift=-1mm]frame.north west)
					arc[start angle=0,end angle=180,radius=1mm]
					([yshift=-1mm,xshift=1mm]frame.north east)
					arc[start angle=180,end angle=0,radius=1mm];
					\path[left color=tcbcolback!60!black,right color=tcbcolback!60!black,
						middle color=tcbcolback!80!black]
					([xshift=-2mm]frame.north west) -- ([xshift=2mm]frame.north east)
					[rounded corners=1mm]-- ([xshift=1mm,yshift=-1mm]frame.north east)
					-- (frame.south east) -- (frame.south west)
					-- ([xshift=-1mm,yshift=-1mm]frame.north west)
					[sharp corners]-- cycle;
				},interior engine=empty,
		},
	fonttitle=\bfseries,
	title={#2},#1}{def}

\newtcbtheorem{theo}%
    {Theorem}{}{theorem}
\newtcolorbox{prob}[1]{colback=red!5!white,colframe=red!50!black,fonttitle=\bfseries,title={#1}}
%================================
% NOTE BOX
%================================

\usetikzlibrary{arrows,calc,shadows.blur}
\tcbuselibrary{skins}
\newtcolorbox{note}[1][]{%
	enhanced jigsaw,
	colback=gray!20!white,%
	colframe=gray!80!black,
	size=small,
	boxrule=1pt,
	title=\textbf{Note:},
	halign title=flush center,
	coltitle=black,
	breakable,
	drop shadow=black!50!white,
	attach boxed title to top left={xshift=1cm,yshift=-\tcboxedtitleheight/2,yshifttext=-\tcboxedtitleheight/2},
	minipage boxed title=1.5cm,
	boxed title style={%
			colback=white,
			size=fbox,
			boxrule=1pt,
			boxsep=2pt,
			underlay={%
					\coordinate (dotA) at ($(interior.west) + (-0.5pt,0)$);
					\coordinate (dotB) at ($(interior.east) + (0.5pt,0)$);
					\begin{scope}
						\clip (interior.north west) rectangle ([xshift=3ex]interior.east);
						\filldraw [white, blur shadow={shadow opacity=60, shadow yshift=-.75ex}, rounded corners=2pt] (interior.north west) rectangle (interior.south east);
					\end{scope}
					\begin{scope}[gray!80!black]
						\fill (dotA) circle (2pt);
						\fill (dotB) circle (2pt);
					\end{scope}
				},
		},
	#1,
}
%================================
% EXAMPLE BOX
%================================
\newtcbtheorem[number within=section]{Example}{Example}
{%
	colback = myexamplebg
	,breakable
	,colframe = myexamplefr
	,coltitle = myexampleti
	,boxrule = 1pt
	,sharp corners
	,detach title
	,before upper=\tcbtitle\par\smallskip
	,fonttitle = \bfseries
	,description font = \mdseries
	,separator sign none
	,description delimiters parenthesis
}
{ex}
%================================
% THEOREM BOX
%================================

\tcbuselibrary{theorems,skins,hooks}
\newtcbtheorem[number within=section]{Theorem}{Theorem}
{%
	enhanced,
	breakable,
	colback = mytheorembg,
	frame hidden,
	boxrule = 0sp,
	borderline west = {2pt}{0pt}{mytheoremfr},
	sharp corners,
	detach title,
	before upper = \tcbtitle\par\smallskip,
	coltitle = mytheoremfr,
	fonttitle = \bfseries\sffamily,
	description font = \mdseries,
	separator sign none,
	segmentation style={solid, mytheoremfr},
}
{th}

%%%%%%%%%%%%%%%%%%%%%%%%%%%%%%%%%%%%%%%%%%%%%%%%%%%%%%%%%%%%%%%%%
% SELF MADE COMMANDS
%%%%%%%%%%%%%%%%%%%%%%%%%%%%%%
\newcommand{\sol}{\setlength{\parindent}{0cm}\textbf{\textit{Løsning:}}\setlength{\parindent}{1cm}}
%%%%%%%%%%%%%%%%%%%%%%%%%%%%%%%%%
\usepackage[tmargin=2cm,rmargin=1in,lmargin=1in,margin=0.85in,bmargin=2cm,footskip=.2in]{geometry}\pagestyle{fancy}
\lhead{Minrui Kevin Zhou 2.b}
\rhead{Opgavesæt 5}

\title{Opgavesæt 5\\
{\Large \textbf{2.b kemi A}}}
\author{Kevin Zhou}
\date{\today}

\begin{document}
\maketitle
\section*{Opgave 3.2}
\sol \\
\textbf{a.}
Da opløsningsmidlet indgår i reaktionsbrøken med stofmængdebrøken, der tilnærmelsesvist er $1$, så vil reaktionsbrøken for reaktionen være
\[
\frac{[\ce{F-} ] \cdot [\ce{H3O+} ]}{[\ce{HF} ]}
\] 
\textbf{b.}
Ved ligevægt må reaktionsbrøken være lig med ligevægtskonstanten. 
Altså har vi 
\begin{equation*}
\begin{split}
  K&=\frac{[\ce{F-} ] \cdot [\ce{H3O+} ]}{[\ce{HF} ]}\\ 
  &=\frac{0,0063 \;\unit{\textsc{m}} \cdot 0,0063 \;\unit{\textsc{m}} }{0,114 \;\unit{\textsc{m}} }\\ 
  &\approx 3,5 \cdot 10^{-4} \;\unit{\textsc{m}} 
\end{split}
\end{equation*}
Værdien for ligevægtskonstanten ved $25 \;\unit{\celsius} $ er altså $3,5 \cdot 10^{-4} \;\unit{\textsc{m}} $. \\[1ex]
\textbf{c.}
Antag, at opløsningen fortyndes sådan, at volumenet bliver $x$ gange det forhenværende volumen, hvor $x \in  ]1;\infty)$.
Reaktionsbrøkens tæller bliver derved $x \cdot x=x^2$ gange mindre, mens nævneren kun bliver $x$ gange mindre. 
Fortyndingen af opløsningen har altså gjort reaktionsbrøken $x$ gange mindre.
Imidlertid ser vi, at siden $x>1$, så må reaktionsbrøken være mindre end $K$ og der sker en forskydning mod højre: 
\[
\frac{[\ce{F-} ] \cdot [\ce{H3O+} ]}{[\ce{HF} ]}<K \quad \ce{->} 
\] 
\section*{Opgave 3.5 B}
\sol \\
\textbf{a.}
Reaktionsbrøken ved stuetemperatur må da være
\begin{equation*}
\begin{split}
  \frac{[\ce{CO} ] \cdot [\ce{H2O} ]}{[\ce{H2} ] \cdot [\ce{CO2}]}&=\frac{0,0256 \;\unit{\textsc{m}} \cdot 0,0181 \;\unit{\textsc{m}} }{0,0065 \;\unit{\textsc{m}} \cdot 0,0130 \;\unit{\textsc{m}} }\\ 
  &\approx 5,5
\end{split}
\end{equation*}
Altså er reaktionsbrøken ved stuetemperatur $5,5$.\\[1ex]
\textbf{b.}
Reaktionsbrøken ved stuetemperatur er da større end ligevægtskonstanten ved $720 \;\unit{\celsius} $, og der sker derfor en forskydning mod venstre:
\[
\frac{[\ce{CO} ] \cdot [\ce{H2O} ]}{[\ce{H2} ] \cdot [\ce{CO2}]}>K \quad \ce{<-} 
\] 
Altså forløber nettoreaktionen mod venstre, når beholderen opvarmes til $720 \;\unit{\celsius} $, hvor ligevægten indstiller sig.\\[1ex]
\textbf{c.}
Siden reaktionen mod højre er endoterm, så er reaktionen mod venstre exoterm.
Der gælder generelt, at hvis temperaturen sænkes, sker der en forskydning i den exoterme reaktions retning.
Altså vil ligevægtens forskydes mod venstre.
\section*{Opgave 3.20}
\sol \\
\textbf{a.}
Vi finder først stofmængden af \ce{Ba(OH)2.8H2O}.
\begin{equation*}
\begin{split}
  n(\ce{Ba(OH)2.8H2O})&=\frac{m(\ce{Ba(OH)2.8H2O}) }{M(\ce{Ba(OH)2.8H2O}) }\\ 
  &=\frac{5,10 \;\unit{g} }{315,46 \;\unit{g/mol} }\\ 
  &=0,0161669 \;\unit{mol} 
\end{split}
\end{equation*}
Reaktionsskemaet for opløsningen af \ce{Ba(OH)2.8H2O} i vand ses nedenfor.
\[
\ce{Ba(OH)2.8H2O(s) -> Ba^2+(aq) + 2OH-(aq) + 8H2O(l)} 
\] 
Vi ser, at reaktionsforholdet mellem \ce{Ba(OH)2.8H2O} og \ce{Ba^2+} er 1:1, hvor reaktionsforholdet mellem \ce{Ba(OH)2.8H2O} og \ce{OH-} er 1:2.
Stofmængderne må være tilsvarende.
\begin{equation*}
\begin{split}
  [\ce{Ba^2+}]&=\frac{n(\ce{Ba^2+} )}{V}\\ 
  &=\frac{n(\ce{Ba(OH)2.8H2O})}{0,250 \;\unit{L} }\\ 
  &\approx 0,0647\;\unit{\textsc{m}} 
\end{split}
\end{equation*}
Siden $n(\ce{OH-} )=2 \cdot n(\ce{Ba^2+}) $, så må der gælde, at 
\begin{equation*}
\begin{split}
  [\ce{OH-} ]&=2 \cdot [\ce{ Ba^2+}]\\ 
  &\approx 0,129 \;\unit{\textsc{m}} 
\end{split}
\end{equation*}
Altså har vi beregnet $[\ce{Ba^2+} ]$ til at være $0,0647 \;\unit{\textsc{m}} $ og $[\ce{OH-} ]$ til at være $0,129 \;\unit{\textsc{m}} $.\\[1ex]
\textbf{b.}
\ce{HNO3} er en stærk syre.
Der må da gælde for stofmængden af \ce{H3O+} i opløsningen af \ce{HNO3}, at 
\begin{equation*}
\begin{split}
  n_{\text{før}}(\ce{H3O+} )&=c(\ce{HNO3} ) \cdot V\\ 
  &= 0,350 \;\unit{\textsc{m}} \cdot 150 \;\unit{mL} 
\end{split}
\end{equation*}
Siden vi blander en opløsning af en stærk syre med en opløsning af en stærk base sker en neutralisationsreaktion, der ses nedenfor.
\[
\ce{H3O+(aq) + OH-(aq) -> 2H2O(l)} 
\] 
Imidlertid ser vi også, at stofmængden af \ce{H3O+} i opløsningen af \ce{HNO3} er større end stofmængden af \ce{OH-} i opløsningen af \ce{Ba(OH)2.8H2O}:
\begin{equation*}
\begin{split}
  0,350 \;\unit{\textsc{m}} \cdot 150 \;\unit{mL} > 0,129335 \;\unit{\textsc{m}} \cdot 250 \;\unit{mL} \iff n_{\text{før} }(\ce{H3O+})>n_{\text{før} }(\ce{OH-} )
\end{split}
\end{equation*}
Vi kan nu regne pH for den færdige blanding (hvis volumen vi antager er $400 \;\unit{mL} $) ud fra den aktuelle stofmængdekoncentration af den overskydende \ce{H3O+}. 
\begin{equation*}
\begin{split}
  pH&=-\log \left(\frac{[\ce{H3O+} ]}{\unit{\textsc{m}}}\right) \\ 
  &=-\log \left(\frac{n_{\text{efter} }(\ce{H3O+} )}{V_{\text{efter} } \cdot \textsc{m}}\right) \\
  &= -\log \left(\frac{n_{\text{før}}(\ce{H3O+} )-n_{\text{før} }(\ce{OH-}) }{400 \;\unit{mL \cdot \textsc{m}} }\right) \\ 
  &=-\log \left(\frac{0,350 \;\unit{\textsc{m}} \cdot 150 \;\unit{mL}-0,129335 \;\unit{\textsc{m}} \cdot 250 \;\unit{mL}}{400 \;\unit{mL \cdot \textsc{m}} }\right) \\ 
  &\approx 1,30
\end{split}
\end{equation*}
Altså vil pH for den færdige blanding være $1,30$.\\[1ex]
\textbf{c.}
Både $n(\ce{Ba^2+} )$ og $n(\ce{NO3-}) $ forbliver den samme som før opblandingen.
Vi antager igen, at blandingens volumen er $400 \;\unit{mL} $.
\begin{equation*}
\begin{split}
  [\ce{Ba^2+} ]&=\frac{0,06466746 \;\unit{\textsc{m}}  \cdot 250 \;\unit{mL} }{400 \;\unit{mL} } \approx 0,0404 \;\unit{\textsc{m}} \\ 
  [\ce{NO3-} ]&=\frac{0,350 \;\unit{\textsc{m}} \cdot 150 \;\unit{mL} }{400 \;\unit{mL} } \approx 0,131 \;\unit{\textsc{m}} 
\end{split}
\end{equation*}
Siden der generelt gælder, at $[\ce{H3O+} ] \cdot [\ce{OH-} ]=K_v$, så har vi
\begin{equation*}
\begin{split}
  [\ce{OH-} ]&= \frac{K_v}{[\ce{H3O+} ]}\\ 
  &=\frac{1,0 \cdot 10^{-14} \;\unit{\textsc{m}^2} }{\frac{0,350 \;\unit{\textsc{m}} \cdot 150 \;\unit{mL}-0,129335 \;\unit{\textsc{m}} \cdot 250 \;\unit{mL}}{400 \;\unit{mL}}}\\ 
  &\approx 1,98 \cdot 10^{-13} \;\unit{\textsc{m}} 
\end{split}
\end{equation*}
I blandingen har vi altså beregnet $[\ce{Ba^2+} ]$ til at være $0,0404 \;\unit{\textsc{m}} $, $[\ce{NO3-} ]$ til at være $0,131 \;\unit{\textsc{m}} $ og $[\ce{ OH-}]$ til at være $1,98 \cdot 10^{-13} \;\unit{\textsc{m}} $.
\section*{Opgave 3.24}
\sol \\
\textbf{a.}
Ved opslag i databogen får vi $pK_s=1,35$.
Derfor har vi 
\begin{equation*}
\begin{split}
K_s&=10^{-pK_s} \;\unit{\textsc{m}} \\ 
&=10^{-1,35} \;\unit{\textsc{m}} \\ 
&\approx 0,0447 \;\unit{\textsc{m}} 
\end{split}
\end{equation*}
\\[1ex]
\textbf{b.}
Reaktionsskemaet for hydronolyse af dichlorethansyre i vand ses nedenfor.
\[
\ce{CHCl2COOH(aq) + H2O(l) <=> CHCl2COO-(aq) + H3O+(aq)} 
\] 
\textbf{c.}
Dichlorethansyre er altså ikke en stærk syre.
Siden der gælder, at 
\begin{equation*}
\begin{split}
  K_s=\frac{[\ce{H3O+} ]^2}{c_s-[\ce{H3O+} ]} \implies 10^{-1,35} \;\unit{\textsc{m}} =\frac{[\ce{H3O+} ]^2}{0,100 \;\unit{\textsc{m}} - [\ce{H3O+} ]}
\end{split}
\end{equation*}
Vi løser andengradsligningen med hensyn til den positive løsning, da koncentrationen kun kan være positiv.
\begin{equation*}
\begin{split}
  [\ce{ H3O+}] &= \frac{-10^{-1,35} \;\unit{\textsc{m}} + \sqrt{(10^{-1,35}\;\unit{\textsc{m}} )^2-4 \cdot 1 \cdot (-0,100 \;\unit{\textsc{m}} \cdot 10^{-1,35} \;\unit{\textsc{m}} )} }{2 \cdot 1}\\ 
  &=0,048133199 \;\unit{\textsc{m}} \\ 
  &\approx 0,0481 \;\unit{\textsc{m}} 
\end{split}
\end{equation*}
Vi regner nu pH ud for opløsningen.
\begin{equation*}
\begin{split}
  pH&=-\log \left(\frac{[\ce{ H3O+}]}{\textsc{m}}\right) \\ 
  &=-\log \left(\frac{0,048133199 \;\unit{\textsc{m}} }{\textsc{m}}\right) \\ 
  &\approx 1,32
\end{split}
\end{equation*}
Altså har vi for en $0,100 \;\unit{\textsc{m}} $ opløsning af dichlorethansyre beregnet $[\ce{H3O+} ]$ til at være $0,0481 \;\unit{\textsc{m}} $ og pH til at være $1,03$.\\[1ex]
\textbf{d.}
Fra reaktionsskeamet for hydronolyse af dichlorethansyre i vand ses det, at $[\ce{CH3COO-} ]=[\ce{H3O+} ]$ når ligevægten har indfundet sig, da $[\ce{H3O+} ]\approx 0$ ved start. 
Altså har vi
\begin{equation*}
\begin{split}
  \alpha&=\frac{[\ce{CHCl2COO-} ]}{c_s}\\ 
  &= \frac{[\ce{H3O+} ]}{0,100 \;\unit{\textsc{m}} } \\
  &=\frac{0,048133199 \;\unit{\textsc{m}} }{0,100 \;\unit{\textsc{m}} }\\ 
  &\approx 0,481
\end{split}
\end{equation*}
Altså er hydronolysegraden for reaktionen $0,481$ og $48,1 \%$ af det oprindelige antal dichlorethansyremolekyler er omdannet til dichlorethanoat. 
\end{document}
