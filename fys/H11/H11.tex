\documentclass{report}
\usepackage{graphicx, tikz-cd, float, titlepic, booktabs} % Required for inserting images
\usepackage{pgfplots}
\pgfplotsset{compat=1.15}
\usepackage{mathrsfs}
\usetikzlibrary{arrows}
\usepackage{amsmath, amssymb, amsthm, amsfonts, siunitx, physics, gensymb}
\AtBeginDocument{\RenewCommandCopy\qty\SI}
\usepackage[version=4]{mhchem}
\usepackage[most,many,breakable]{tcolorbox}
\usepackage{xcolor, fancyhdr, varwidth}
\usepackage[Glenn]{fncychap}
%Options: Sonny, Lenny, Glenn, Conny, Rejne, Bjarne, Bjornstrup
\usepackage{hyperref, cleveref}
\usepackage{icomma, enumitem} %comma as decimal and continue enumerate with [resume]
\usepackage[danish]{babel}
%%%%%%%%%%%%%%%%%%%%%%%%%%%%%%
% SELF MADE COLORS
%%%%%%%%%%%%%%%%%%%%%%%%%%%%%%
\definecolor{myg}{RGB}{56, 140, 70}
\definecolor{myb}{RGB}{45, 111, 177}
\definecolor{myr}{RGB}{199, 68, 64}
\definecolor{mytheorembg}{HTML}{F2F2F9}
\definecolor{mytheoremfr}{HTML}{00007B}
\definecolor{mylenmabg}{HTML}{FFFAF8}
\definecolor{mylenmafr}{HTML}{983b0f}
\definecolor{mypropbg}{HTML}{f2fbfc}
\definecolor{mypropfr}{HTML}{191971}
\definecolor{myexamplebg}{HTML}{F2FBF8}
\definecolor{myexamplefr}{HTML}{88D6D1}
\definecolor{myexampleti}{HTML}{2A7F7F}
\definecolor{mydefinitbg}{HTML}{E5E5FF}
\definecolor{mydefinitfr}{HTML}{3F3FA3}
\definecolor{notesgreen}{RGB}{0,162,0}
\definecolor{myp}{RGB}{197, 92, 212}
\definecolor{mygr}{HTML}{2C3338}
\definecolor{myred}{RGB}{127,0,0}
\definecolor{myyellow}{RGB}{169,121,69}
\definecolor{myexercisebg}{HTML}{F2FBF8}
\definecolor{myexercisefg}{HTML}{88D6D1}
%%%%%%%%%%%%%%%%%%%%%%%%%%%%%%%%%%%%%%%%%%%%%%%%%%%%%%%%%%%%%%%%%%%%%%
% Box environments for theorems and problems
%%%%%%%%%%%%%%%%%%%%%%%%%%%%%%%%%%%%%%%%%%%%%%%%%%%%%%%%%%%%%%%%%%%%%
\setlength{\parindent}{1cm}
%================================
% Question BOX
%================================
\makeatletter
\newtcbtheorem{question}{Opgave}{enhanced,
	breakable,
	colback=white,
	colframe=myb!80!black,
	attach boxed title to top left={yshift*=-\tcboxedtitleheight},
	fonttitle=\bfseries,
	title={#2},
	boxed title size=title,
	boxed title style={%
			sharp corners,
			rounded corners=northwest,
			colback=tcbcolframe,
			boxrule=0pt,
		},
	underlay boxed title={%
			\path[fill=tcbcolframe] (title.south west)--(title.south east)
			to[out=0, in=180] ([xshift=5mm]title.east)--
			(title.center-|frame.east)
			[rounded corners=\kvtcb@arc] |-
			(frame.north) -| cycle;
		},
	#1
}{def}
\makeatother
%================================
% DEFINITION BOX
%================================

\newtcbtheorem[]{Definition}{Definition}{enhanced,
	before skip=2mm,after skip=2mm, colback=red!5,colframe=red!80!black,boxrule=0.5mm,
	attach boxed title to top left={xshift=1cm,yshift*=1mm-\tcboxedtitleheight}, varwidth boxed title*=-3cm,
	boxed title style={frame code={
					\path[fill=tcbcolback]
					([yshift=-1mm,xshift=-1mm]frame.north west)
					arc[start angle=0,end angle=180,radius=1mm]
					([yshift=-1mm,xshift=1mm]frame.north east)
					arc[start angle=180,end angle=0,radius=1mm];
					\path[left color=tcbcolback!60!black,right color=tcbcolback!60!black,
						middle color=tcbcolback!80!black]
					([xshift=-2mm]frame.north west) -- ([xshift=2mm]frame.north east)
					[rounded corners=1mm]-- ([xshift=1mm,yshift=-1mm]frame.north east)
					-- (frame.south east) -- (frame.south west)
					-- ([xshift=-1mm,yshift=-1mm]frame.north west)
					[sharp corners]-- cycle;
				},interior engine=empty,
		},
	fonttitle=\bfseries,
	title={#2},#1}{def}
\newtcbtheorem[]{definition}{Definition}{enhanced,
	before skip=2mm,after skip=2mm, colback=red!5,colframe=red!80!black,boxrule=0.5mm,
	attach boxed title to top left={xshift=1cm,yshift*=1mm-\tcboxedtitleheight}, varwidth boxed title*=-3cm,
	boxed title style={frame code={
					\path[fill=tcbcolback]
					([yshift=-1mm,xshift=-1mm]frame.north west)
					arc[start angle=0,end angle=180,radius=1mm]
					([yshift=-1mm,xshift=1mm]frame.north east)
					arc[start angle=180,end angle=0,radius=1mm];
					\path[left color=tcbcolback!60!black,right color=tcbcolback!60!black,
						middle color=tcbcolback!80!black]
					([xshift=-2mm]frame.north west) -- ([xshift=2mm]frame.north east)
					[rounded corners=1mm]-- ([xshift=1mm,yshift=-1mm]frame.north east)
					-- (frame.south east) -- (frame.south west)
					-- ([xshift=-1mm,yshift=-1mm]frame.north west)
					[sharp corners]-- cycle;
				},interior engine=empty,
		},
	fonttitle=\bfseries,
	title={#2},#1}{def}

\newtcbtheorem{theo}%
    {Theorem}{}{theorem}
\newtcolorbox{prob}[1]{colback=red!5!white,colframe=red!50!black,fonttitle=\bfseries,title={#1}}
%================================
% NOTE BOX
%================================

\usetikzlibrary{arrows,calc,shadows.blur}
\tcbuselibrary{skins}
\newtcolorbox{note}[1][]{%
	enhanced jigsaw,
	colback=gray!20!white,%
	colframe=gray!80!black,
	size=small,
	boxrule=1pt,
	title=\textbf{Note:},
	halign title=flush center,
	coltitle=black,
	breakable,
	drop shadow=black!50!white,
	attach boxed title to top left={xshift=1cm,yshift=-\tcboxedtitleheight/2,yshifttext=-\tcboxedtitleheight/2},
	minipage boxed title=1.5cm,
	boxed title style={%
			colback=white,
			size=fbox,
			boxrule=1pt,
			boxsep=2pt,
			underlay={%
					\coordinate (dotA) at ($(interior.west) + (-0.5pt,0)$);
					\coordinate (dotB) at ($(interior.east) + (0.5pt,0)$);
					\begin{scope}
						\clip (interior.north west) rectangle ([xshift=3ex]interior.east);
						\filldraw [white, blur shadow={shadow opacity=60, shadow yshift=-.75ex}, rounded corners=2pt] (interior.north west) rectangle (interior.south east);
					\end{scope}
					\begin{scope}[gray!80!black]
						\fill (dotA) circle (2pt);
						\fill (dotB) circle (2pt);
					\end{scope}
				},
		},
	#1,
}
%================================
% EXAMPLE BOX
%================================
\newtcbtheorem[number within=section]{Example}{Example}
{%
	colback = myexamplebg
	,breakable
	,colframe = myexamplefr
	,coltitle = myexampleti
	,boxrule = 1pt
	,sharp corners
	,detach title
	,before upper=\tcbtitle\par\smallskip
	,fonttitle = \bfseries
	,description font = \mdseries
	,separator sign none
	,description delimiters parenthesis
}
{ex}
%================================
% THEOREM BOX
%================================

\tcbuselibrary{theorems,skins,hooks}
\newtcbtheorem[number within=section]{Theorem}{Theorem}
{%
	enhanced,
	breakable,
	colback = mytheorembg,
	frame hidden,
	boxrule = 0sp,
	borderline west = {2pt}{0pt}{mytheoremfr},
	sharp corners,
	detach title,
	before upper = \tcbtitle\par\smallskip,
	coltitle = mytheoremfr,
	fonttitle = \bfseries\sffamily,
	description font = \mdseries,
	separator sign none,
	segmentation style={solid, mytheoremfr},
}
{th}

%%%%%%%%%%%%%%%%%%%%%%%%%%%%%%%%%%%%%%%%%%%%%%%%%%%%%%%%%%%%%%%%%
% SELF MADE COMMANDS
%%%%%%%%%%%%%%%%%%%%%%%%%%%%%%
\newcommand{\sol}{\setlength{\parindent}{0cm}\textbf{\textit{Løsning:}}\setlength{\parindent}{1cm}}
%%%%%%%%%%%%%%%%%%%%%%%%%%%%%%%%%
\usepackage[tmargin=2cm,rmargin=1in,lmargin=1in,margin=0.85in,bmargin=2cm,footskip=.2in]{geometry}\pagestyle{fancy}
\lhead{Minrui Kevin Zhou 2.b}
\rhead{H11: Kernefysik}

\title{H11: Kernefysik\\
{\Large \textbf{2.b Fys A}}}
\author{Kevin Zhou}
\date{\today}

\begin{document}
\maketitle
\begin{question}{Lungeundersøgelse}{}
  Ved en lungeundersøgelse inhaleres en gas, som indeholder isotopen $\ce{^{133}Xe}$.
  \begin{itemize}
    \item[a.] Opstil reaktionsskemaet for henfaldet af $\ce{^{133}Xe}$.
  \end{itemize}
  Da $\ce{^{133}Xe}$ har en forholdsvis kort halveringstid, skal en kilde med $\ce{^{133}Xe}$ anvendes inden 10 dage efter produktionen. 
Ved produktionen af en kilde er aktiviteten af $\ce{^{133}Xe}$ i kilden $740 \;\unit{MBq} $. 
\begin{itemize}
  \item[b.] Bestem massen af den resterende mængde $\ce{^{133}Xe}$ i kilden 10 dage efter
produktionen. 
\end{itemize}
\end{question}
\sol \\
\textbf{a.}
Vi aflæser først på et henfladskort, at isotopen henfalder med $\beta$-henfald.
Reaktionsskemaet for henfaldet af $\ce{^{133}Xe} $ ses nedenfor. 
\[
\ce{^{133}_{54}Xe} \to \ce{^{133}_{55}Cs} + \ce{^0_{-1}e} + \bar{\nu} 
\] 
\textbf{b.}
Fra Ptable slår vi halveringstiden op til at være
\[
T_{\frac{1}{2}}&=4,53 \cdot 10^5 \;\unit{s} \\ 
\] 
Vi finder først antallet af kerner til start.
\begin{equation*}
\begin{split}
  N_0&=\frac{A}{k}\\ 
  &=\frac{740 \;\unit{MBq} \cdot T_{\frac{1}{2}}}{\ln(2)}\\
  &=\frac{740 \cdot 10^6 \;\unit{Bq} \cdot 4,53 \cdot 10^5 \;\unit{s} }{\ln(2)}\\ 
  &=4,836202316 \cdot 10^{14}
\end{split}
\end{equation*}
Vi finder nu antallet af kerner efter 10 dage.
Imidlertid ser vi, at $10 \;\unit{d} = 10 \cdot 24 \cdot 60^2 \;\unit{s} =864000 \;\unit{s} $.
\begin{equation*}
\begin{split}
  N(8,64 \cdot 10^5 \;\unit{s} )&=N_0 \cdot \left(\frac{1}{2}\right)^{\frac{8,64 \cdot 10^5 \;\unit{s} }{T_{\frac{1}{2}}}}\\ 
  &=N_0 \cdot \left(\frac{1}{2}\right)^{\frac{8,64 \cdot 10^5 \;\unit{s} }{4,53 \cdot 10^5 \;\unit{s} }}\\ 
  &=1,28930164 \cdot 10^{14}
\end{split}
\end{equation*}
Ved opslag på Ptable ved vi, at hvert enkelt atom har massen $132,905910722 \;\unit{u} $.
Altså er massen af de resterende atomer 
\begin{equation*}
\begin{split}
  m_{\text{samlet} }&= N \cdot m_{\text{atom} }\\ 
  &=1,28930164 \cdot 10^{14} \cdot 132,905910722 \;\unit{u}\\ 
  &= 1,28930164 \cdot 10^{14}\cdot 132,905910722 \;\unit{u} \cdot 1,6605 \cdot 10^{-24} \;\unit{g/u} \\ 
  &\approx 2,85 \cdot 10^{-9} \;\unit{g} \\ 
  &= 28,5 \;\unit{ng} 
\end{split}
\end{equation*}
Altså er massen af den resterende mængde \ce{^{133}Xe} i kilden 10 dage efter produktion $28,5 \;\unit{ng} $.
\begin{question}{Måling af stålpladers tykkelse}{}
  På stålvalseværket DanSteel A/S i Frederiksværk kontrolleres de fremstillede stålpladers tykkelse ved hjælp af strålingen fra en radioaktiv kilde.
  Man bruger en \ce{^{137}Cs}-kilde, der ved installationen i 1992 havde aktiviteten $31,4 \;\unit{GBq} $.
  \begin{itemize}
    \item[a.] Hvor længe varer det, før aktiviteten er faldet til $15,0 \;\unit{GBq} $.
  \end{itemize}
  Kilden udsender $\gamma$-stråling.
Når $\gamma$-strålingen passerer gennem stof, absorberes en del af strålingen.
Tællehastigheden $I(x)$ efter passage af stoftykkelsen $x$ er
$$
I(x)=I_0 \cdot\left(\frac{1}{2}\right)^{\frac{x}{x_{\frac{1}{2}}}}
$$
hvor $I_0$ er tællehastigheden, når der ikke er nogen stålplade mellem kilde og detektor. Halveringstykkelsen for $\gamma$-strålingen i stål er $x_{\frac{1}{2}}=1,25 \mathrm{~cm}$.
\begin{itemize}
  \item[b.]  Bestem $I_0$ når det oplyses, at for en stålplade med tykkelsen $2,3 \mathrm{~cm}$ måles tællehastigheden $I$ til $3,57 \cdot 10^4 \mathrm{~s}^{-1}$.
\end{itemize}
\end{question}
\sol \\
\textbf{a.}
Aktiviteten aftager eksponentielt med tiden. Der gælder da
\begin{equation*}
\begin{split}
  A(t)=A_0 \cdot \left(\frac{1}{2}\right)^{\frac{t}{T_{\frac{1}{2}}}} \iff t= T_{\frac{1}{2}} \cdot \log_{\frac{1}{2}} \left(\frac{A(t)}{A_0}\right) 
\end{split}
\end{equation*}
hvor $T_{\frac{1}{2}}$ er halveringstiden, $A_0$ er aktiviteten ved $t=0$ og $A(t)$ er aktiviteten ved tiden $t$.
Vi regner nu tiden ud.
\begin{equation*}
\begin{split}
  t&= T_{\frac{1}{2}} \cdot \log_{\frac{1}{2}} \left(\frac{A(t)}{A_0}\right) \\ 
  &=30,06 \;\unit{y} \cdot \log_{\frac{1}{2}} \left(\frac{15,0 \;\unit{GBq} }{31,4 \;\unit{GBq} }\right) \\ 
  &\approx 32,0 \;\unit{y} 
\end{split}
\end{equation*}
Altså varer det $32,0$ år før aktiviteten er faldet til $15,0 \;\unit{GBq} $. \\[1ex]
\textbf{b.}
Fra udtrykket for tællehastigheden $I(x)$ får vi, at 
\begin{equation*}
\begin{split}
  I(x)=I_0 \cdot\left(\frac{1}{2}\right)^{\frac{x}{x_{\frac{1}{2}}}} \iff I_0=\frac{I(x)}{\left(\frac{1}{2}\right)^{\frac{x}{x_{\frac{1}{2}}}} }
\end{split}
\end{equation*}
Vi kan da nemt bestemme $I_0$ med de givne oplysninger: 
\begin{equation*}
\begin{split}
  I_0&=\frac{I(x)}{\left(\frac{1}{2}\right)^{\frac{x}{x_{\frac{1}{2}}}} }\\ 
  &=\frac{3,57 \cdot 10^4 \;\unit{s^{-1}} }{\left(\frac{1}{2}\right)^{\frac{2,3 \;\unit{cm} }{1,25 \;\unit{cm} }}}\\ 
  &\approx 1,27 \cdot 10^5 \;\unit{s^{-1}}  
\end{split}
\end{equation*}
\end{document}
